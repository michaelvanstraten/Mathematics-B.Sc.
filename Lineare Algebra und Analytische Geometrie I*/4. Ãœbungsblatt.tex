\documentclass{problemset}

\Lecture{Lineare Algebra und Analytische Geometrie I*}
\Problemset{4}
\author{Michael van Straten}

\begin{document}
\maketitle
\begin{problem}[Eigenschaften der Kompositionen]{7 Punkte}
Gegeben seien nichtleere Mengen $X$, $Y$ und $Z$ und Abbildungen $f : X \rightarrow Y$ und $g : Y \rightarrow Z$.
Zeigen oder widerlegen Sie die folgenden Aussagen:
\begin{enumerate}
	\item Sind $f$ und $g$ injektiv, so ist auch $g \circ f$ injektiv,
	\item sind $f$ und $g$ surjektiv, so ist auch $g \circ f$ surjektiv,
	\item sind $f$ und $g$ bijektiv, so ist auch $g \circ f$ bijektiv und ${(g \circ f )}^{−1} = f^{-1} \circ g^{-1}$,
	\item ist $g \circ f$ injektiv, so ist auch $g$ injektiv,
	\item ist $g \circ f$ injektiv, so ist auch $f$ injektiv,
	\item ist $g \circ f$ surjektiv, so ist auch $g$ surjektiv,
	\item ist $g \circ f$ surjektiv, so ist auch $f$ surjektiv.
\end{enumerate}
\begin{proof}
	$ $

	\begin{enumerate}
		\item Lass $x_1,x_2 \in X$, somit is zu zeigen das, wenn $(g \circ f)(x_1) = (g \circ f)(x_2) $ dann folgt $x_1 = x_2$.
		      Da $g$ injektiv ist gilt, wenn $y_1, y_2 \in Y$ und $g(y_1) = g(y_2)$ dann folgt $y_1 = y_2$.
		      Es ist zu bemerken das für $f$ injektiv gilt, $x_1, x_2 \in X$ und $f(x_1) = f(x_2) \Rightarrow x_1 = x_2$.
		      Da $g(y_1) = g(y_2) \Rightarrow y_1 = y_2 \Rightarrow f(x_1) = f(x_2) \Rightarrow x_1 = x_2$, folgt somit, das $g \circ f$ injektiv ist.
		      Was zu zeigen war. \checkmark
		\item Es ist zu zeigen das $\forall z \in Z$ folgt $\exists x \in X$ sodass $(g \circ f)(x) = z$.
		      Da $g$ surjektiv folgt, $\forall z \in Z$ folgt $\exists y \in X$ sodas $g(y) = z$.
		      Da $f$ surjektiv folgt, $\forall y \in Y$ folgt $\exists x \in X$ sodas $f(x) = y$.
		      Somit folgt $\forall z \in Z \Rightarrow \exists y \in Y: g(y) = z \land \forall y \in Y \Rightarrow \exists x \in X: f(x) = y$.
		      $g \circ f$ ist somit surjektiv, was zu zeigen war. \checkmark
		\item Angenommen $f$ und $g$ sind bijektiv. Das bedeutet, dass $f$ und $g$ sowohl injektiv als auch surjektiv sind.
		      Betrachten Sie $g \circ f$. Aufgrund der Injektivität von $f$ und $g$ ist $g \circ f$ ebenfalls injektiv (wie im Beweis für Punkt i gezeigt).
		      Aufgrund der Surjektivität von $f$ und $g$ ist $g \circ f$ auch surjektiv (wie im Beweis für Punkt ii gezeigt).
		      Daher ist $g \circ f$ bijektiv. Um die Gleichung ${(g \circ f )}^{-1} = f^{-1} \circ g^{-1}$ zu zeigen, betrachten Sie die Zusammensetzung beider Seiten mit einem beliebigen Element $x \in X$:
		      \begin{align*}
			      \left(f^{-1} \circ g^{-1}\right) \circ (g \circ f)(x) & = f^{-1}\left(g^{-1}(g(f(x)))\right)        \\
			                                                            & = f^{-1}\left((g^{-1} \circ g)(f(x))\right) \\
			                                                            & = f^{-1}\left(\text{Id}_Y(f(x))\right)      \\
			                                                            & = f^{-1}(f(x))                              \\
			                                                            & = x.
		      \end{align*}
		      Somit ist ${(g \circ f )}^{-1} = f^{-1} \circ g^{-1}$. \checkmark

		\item Angenommen, $g \circ f$ ist injektiv. Wir müssen zeigen, dass $g$ injektiv ist.
		      Seien $y_1$ und $y_2$ zwei verschiedene Elemente in $Y$. Angenommen, $g(y_1) = g(y_2)$.
		      Da $g \circ f$ bijektiv ist, gibt es $x_1$ und $x_2$ in $X$ mit $f(x_1) = y_1$ und $f(x_2) = y_2$. Dann haben wir:
		      \[(g \circ f)(x_1) = g(f(x_1)) = g(y_1)\]
		      \[(g \circ f)(x_2) = g(f(x_2)) = g(y_2)\]
		      Da $g \circ f$ injektiv ist, folgt aus $(g \circ f)(x_1) = (g \circ f)(x_2)$, dass $x_1 = x_2$.
		      Aber das bedeutet $y_1 = f(x_1) = f(x_2) = y_2$, ein Widerspruch zur Annahme, dass $y_1$ und $y_2$ verschieden sind. Daher muss die Annahme, dass $g(y_1) = g(y_2)$, falsch sein, und somit ist $g$ injektiv. \checkmark
		\item Der Beweis ist ähnlich zu Teil iv) Angenommen, $g \circ f$ ist injektiv.
		      Dann müssen wir zeigen, dass $f$ injektiv ist. Seien $x_1$ und $x_2$ zwei verschiedene Elemente in $X$. Angenommen, $f(x_1) = f(x_2)$. Dann haben wir:
		      \[(g \circ f)(x_1) = g(f(x_1)) = g(f(x_2)) = (g \circ f)(x_2)\]
		      Da $g \circ f$ injektiv ist, folgt aus $(g \circ f)(x_1) = (g \circ f)(x_2)$, dass $x_1 = x_2$, was der Annahme widerspricht. Daher muss die Annahme, dass $f(x_1) = f(x_2)$, falsch sein, und somit ist $f$ injektiv. \checkmark
		\item Angenommen, $g \circ f$ ist surjektiv. Wir wollen zeigen, dass $g$ surjektiv ist.
		      Sei $z \in Z$ beliebig. Da $g \circ f$ surjektiv ist, existiert ein $x \in X$ mit $g \circ f(x) = z$.
		      Das bedeutet, $g(f(x)) = z$. Somit existiert ein $y \in Y$ mit $f(x) = y$.
		      Somit haben wir $g(y) = z$, was die Surjektivität von $g$ zeigt. \checkmark
		\item Angenommen, $g \circ f$ ist surjektiv. Wir wollen zeigen, dass $f$ surjektiv ist.
		      Sei $y \in Y$ beliebig. Da $g$ surjektiv ist, existiert für jedes $z \in Z$ ein $y \in Y$ mit $g(y) = z$.
		      Da $g \circ f$ surjektiv ist, existiert ein $x \in X$ mit $g \circ f(x) = z$.
		      Das bedeutet, $g(f(x)) = z$, und somit haben wir $g(y) = z$.
		      Das zeigt, dass $y$ die Funktion $f(x)$ erreicht, was die Surjektivität von $f$ zeigt. \checkmark
	\end{enumerate}
\end{proof}
\end{problem}

\begin{problem}[Verlustfreie Kompression]{4 Punkte}
Sei $B := \{0, 1\}$. Dadurch ist für ein $n \in \mathbb{N}$ die Menge der Bitstrings der Länge $n$ gegeben durch
\[B^n = \bigtimes_{i=0}^{n} B = \underbrace{B \times \ldots \times B}_{n-\text{mal}}.\]
Die Menge der Bitstrings deren Länge kleiner als $n$ ist definieren wir durch
\[B^{<n} = \bigcup_{i=1}^{n-1} B_i = B \cup B^2 \cup \ldots \cup B^{n-1}.\]
Um alle möglichen Daten mit $n$ Bits verlustfrei zu komprimieren, benötigte man nun eine Kompressionsabbildung $C : B^n \rightarrow B^{<n}$ und eine Dekompressionsabbildung $D : B^{<n} \rightarrow B^n$ mit der Eigenschaft
\[D \circ C = \text{Id}_{B^n}.\]
Nutzen Sie die folgenden Schritte um zu zeigen, dass ein solches Paar $(C, D)$ nicht existieren kann:
\begin{enumerate}
	\item Berechnen Sie die Kardinalitäten $|B^n|$ und $|B^{<n}|$.
	\item Zeigen Sie, dass $D \circ C$ injektiv ist und folgern Sie, dass damit auch $C$ selbst injektiv sein muss.
	\item Zeigen Sie, dass $C$ nicht injektiv sein kann.
\end{enumerate}

\begin{proof}
	$ $

	\begin{enumerate}
		\item Kardinalität von $B^n$: \[
			      \mid B^n \mid = \underbrace{\mid B \mid \cdot \ldots \cdot \mid B \mid }_{n - \text{mal}} = {|B|}^n = 2^n,
		      \]
		      Kardinalität von $B^{<n}$: \[
			      \mid B^{<n} \mid = \sum_{i=1}^{n-1} \mid B^i \mid = \sum_{i=1}^{n-1} 2^i = 2^n - 1.
		      \]
		\item Für die Abbildungen $Id_{B^n}$ gilt durch die Definition der Identitätsabbildungen das, $\forall b \in B^n: Id_{B^n}(x) = x$.
		      Es folgt somit $\forall x_1, x_2 \in B^n \land Id_{B^n}(x_1) = Id_{B^n}(x_2)$ folgt $x_1 = x_2$.
		      Somit muss $D \circ C$ injektiv sein.
		      Wenn $C$ nicht injektiv wäre folgte $\exists b_1, b_2 \in B: C(b_1) = C(b_2) \land x_1 \ne x_2$.
		      Dies ist allerdings im wiederspruch mit der injektivät von $D \circ C$ da $\exists b_1, b_2 \in B: (D \circ C)(b_1) = (D \circ C)(b_2) \land x_1 \ne x_2$
		\item
		      Die Abbildungen $C$ bildet von einer Menge mit Tupeln von Länge $n$ zu einer Menge mit Tupeln mit einer maximalen Länge $n-1$ ab.
		      Somit müssen mindestens zwei Elemente aus $B^n$ auf dasselbe element in $B^{<n}$ abbilden.
		      Somit kann die Abbildungen $C$ nicht injektiv sein, was zu zeigen war. \checkmark
	\end{enumerate}
\end{proof}

\end{problem}

\begin{problem}{4 Punkte}
Sei $G := \{e, a, b, c\}$, $|G| = 4$ und $(G, \cdot)$ eine Gruppe mit neutralem Element $e$. Zeigen Sie, dass
\begin{enumerate}
	\item $a \cdot b, b \cdot a \in \{e, c\}$,
	\item $a \cdot b = b \cdot a$,
	\item $G$ kommutativ ist.
\end{enumerate}

\begin{proof}
	$ $

	\begin{enumerate}
		\item Da $e$ eindeutig bestimmt ist, und seine Anwendung kommutativ, kann hier zwischen zwei fällen unterschieden werden.
		      Entweder ist $b$ das inverse element zu $a$ und umgekehrt, in diesem Fall ist $a \cdot b$ und $b \cdot c$ gleich $e$ und somit in $\{e,c\}$.
		      Oder $a \cdot b$ und $b \cdot a$ sind nicht gleich dem neutralen Element und müssen somit laut Gruppen Axiomen ein weiteres Element der Gruppe bilden.
		      In diesem Fall ist sind die einzigen Elemente, welche nicht dem neutralen Element entsprechen das Element $c$.
		      Das Element $c$ ist in der Menge $\{e, c\}$ und somit ist auch in diesem fall das Resultat der Operation in der Menge $\{e,c\}$.
		      \checkmark
		\item Wie aus i) bereits bekannt ist entweder $b$ das inverse Element zu $a$ und umgekehrt oder $a \cdot b$ und $b \cdot c$ resultieren in dem Element $c$.
		      In beiden Fällen ist die Operation kommutativ und somit ist $a \cdot b$, in dieser Gruppe, immer gleich $b \cdot a$.
		      \checkmark
		\item
		      Aus ii) wissen wir bereits das die Operation über die Elemente $a,b,e$ kommutativ ist.
		      Es ist noch zu beweisen, dass die Operation $x \in G$ $c \cdot x$ kommutativ ist.
			      [no time left] ...
	\end{enumerate}
\end{proof}
\end{problem}

\begin{problem} {5 Punkte}
Sei $(G, \cdot)$ eine Gruppe mit neutralem Element $e \in G$. Zeigen Sie, dass
\begin{enumerate}
	\item Falls $g \cdot g = e$ für alle $g \in G$, dann ist $G$ abelsch.
	\item Für $a \in G$ ist $(G, \circ_a)$ ebenfalls eine Gruppe, wobei $x \circ_a y := x \cdot a \cdot y$ für alle $x, y \in G$.
\end{enumerate}
\begin{proof}
	$ $

	\begin{enumerate}
		\item Aus $g \cdot g = e$ für alle $g \in G$ folgt: \begin{align*}
			       & (x \cdot y) \cdot (x \cdot y) = e \tag{$x \cdot y \in G \Longrightarrow \exists g \in G: g \cdot g = e$} \\
			       & \Longrightarrow x \cdot (x \cdot y) \cdot (x \cdot y) = x \cdot e                                        \\
			       & \Longrightarrow x \cdot (x \cdot y) \cdot (x \cdot y) = x                                                \\
			       & \Longrightarrow (x \cdot x) \cdot y \cdot (x \cdot y)) = x                                               \\
			       & \Longrightarrow e \cdot y \cdot (x \cdot y) = x                                                          \\
			       & \Longrightarrow y \cdot (x \cdot y) = x                                                                  \\
			       & \Longrightarrow y \cdot y \cdot (x \cdot y) = y \cdot x                                                  \\
			       & \Longrightarrow (y \cdot y) \cdot (x \cdot y) = y \cdot x                                                \\
			       & \Longrightarrow e \cdot (x \cdot y) = y \cdot x                                                          \\
			       & \Longrightarrow x \cdot y = y \cdot x
		      \end{align*}
		      somit ist $(G, \cdot)$ abelsch. \checkmark

		\item Sei $a \in G$. Wir wollen zeigen, dass $(G, \circ_a)$ eine Gruppe ist.
		      \begin{enumerate}
			      \item \textbf{Abgeschlossenheit:} Für alle $x, y \in G$ ist $x \circ_a y = x \cdot a \cdot y \in G$, da $G$ unter der Verknüpfung $\cdot$ abgeschlossen ist.

			      \item \textbf{Assoziativität:} Seien $x, y, z \in G$. Dann ist
			            \begin{align*}
				            (x \circ_a y) \circ_a z & = (x \cdot a \cdot y) \cdot a \cdot z \\
				                                    & = x \cdot a \cdot (y \cdot a \cdot z) \\
				                                    & = x \circ_a (y \circ_a z).
			            \end{align*}

			      \item \textbf{Neutrales Element:} Das neutrale Element von $(G, \circ_a)$ ist $a^{-1}$, denn für jedes $x \in G$ haben wir
			            \[ x \circ_a a^{-1} = x \cdot a \cdot a^{-1} = x \cdot e = x. \]

			      \item \textbf{Inverse Elemente:} Für jedes $x \in G$ ist $x^{-1} = a^{-1} \cdot x^{-1} \cdot a^{-1}$ das Inverse von $x$ bezüglich $\circ_a$, denn
			            \[ x \circ_a x^{-1} = x \cdot a \cdot a^{-1} \cdot x^{-1} \cdot a^{-1} = x \cdot e \cdot x^{-1} \cdot a^{-1} =  x \cdot  x^{-1} \cdot a^{-1} = e \cdot a^{-1} = \cdot a^{-1}. \]

		      \end{enumerate}
		      Damit ist gezeigt, dass $(G, \circ_a)$ eine Gruppe ist. \checkmark
	\end{enumerate}
\end{proof}
\end{problem}

\end{document}
