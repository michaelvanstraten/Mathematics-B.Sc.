\documentclass{exam}

\usepackage{amssymb, amsmath, amsthm}
\usepackage[german]{babel}
\usepackage{hyperref}
\usepackage{enumitem}
\usepackage{tabularx}

\title{Lineare Algebra und analytische Geometrie I - 3. Übungsblatt}
\author{Michael van Straten}
\date{\today}

\newtheorem{definition}{Definition}

\begin{document}
\maketitle

\section*{Aufgabe 1:}
Geben Sie alle Äquivalenzrelationen, die zugehörigen Äquivalenzklassen und die Mächtigkeit der
zugehörigen Quotientenmenge auf der Menge $\{1, 2, 3\}$ an.

\begin{center}
	\begin{tabular}{ | c | c | c | }
		\hline
		\textbf{Äquivalenzrelationen}                                                                 & \textbf{Äquivalenzklassen} & \textbf{Mächtigkeit der Quotientenmenge} \\
		\hline
		$a \sim b \Leftrightarrow a = b$                                                              &
		\begin{array}{ccc}
			$[1]_\sim := \{b \in \{1,2,3\} \mid b \sim a\} = \{1\}$ \\
			$[2]_\sim := \{b \in \{1,2,3\} \mid b \sim a\} = \{2\}$ \\
			$[3]_\sim := \{b \in \{1,2,3\} \mid b \sim a\} = \{3\}$
		\end{array} & \[
			| A/{\sim} := \{[a]_{\sim\!} \mid a \in A\}| = 3
		\]                                                                                                                                                                    \\
		\hline
		$a \sim b \Leftrightarrow a \equiv_1 b                                                        &
	\end{tabular}
\end{center}

\pagebreak

\section*{Aufgabe 2:}

\pagebreak

\section*{Aufgabe 3:}
Seien $X$ und $Y$ Mengen und $f : X \rightarrow Y$. Beweisen Sie
\begin{enumerate}[label=\roman*)]
	\item für $M \subseteq X$ gilt $M \subseteq f^{-1}(f(M))$,
	\item falls $f$ injektive gilt i) sogar Gleichheit,
	\item für $N \subseteq Y$ gilt $f(f^{-1}(N)) \subseteq N$,
	\item falls $f$ surjektiv gilt iii) sogar Gleichheit,
\end{enumerate}
\begin{proof} \\\
	\begin{enumerate}[label=\roman*)]
		\item
		      Sei $x \in M$ beliebig aber fest, da $M \subseteq X \Rightarrow x \in X$.
		      Lass $A = f(M) = \{f(\tilde{x}) \mid \tilde{x} \in M\}$, somit ist $f(x) \in A$.
		      Lass nun $U = f^{-1}(A) = \{\tilde{x} \in X \mid f(\tilde{x}) \in A\}$.
		      Somit ist $x \in U$, da $x \in X$ und $f(x) \in A$.
		      Da $x$ beliebig gewählt wurde, ist somit $M \subseteq U$. \checkmark
		\item
		      Wenn $f$ injektiv dann existieren keine zwei $x_1, x_2$ für die gilt $f(x_1) = f(x_2) \land x_1 \not= x_2$.
		      Somit muss es für jedes $y \in A$, ein wohl bestimmtes $x$ geben für das gilt $x \in X \land x \in M \land f(x) = y$.
		      Da jedes $x \in U$ somit auch in $M$ ist, muss $U \subseteq M$ oder anders $M = U$. \checkmark
		\item
		      Lass $U = f^{-1}(N) = \{x \in X \mid f(x) \in N\}$, somit gilt $\forall x \in U \Rightarrow f(x) \in N$. \\
		      Lass jetzt $A = f(U) = \{f(x) \mid x \in U\}$, da $\forall x \in U$ gilt für $f(x) \in N$. \\
		      Gilt $\forall y \in f(f^{-1}(N)) \Rightarrow y \in N$, was zu zeigen war. \checkmark
		\item
		      Wenn $f$ surjektive dann gilt $\forall y \in N \Rightarrow \exists x \in X$ sodass $f(x) = y$. \\
		      Somit $\exists x \in U: f(x) \in N$, daraus folgt Axiomatisch aus der Definition von A $\forall y \in N: \exists y \in A$. \checkmark
	\end{enumerate}
\end{proof}

\pagebreak

\section*{Aufgabe 4:}

\end{document}

