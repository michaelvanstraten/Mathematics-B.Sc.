\documentclass{../problemset}

\Lecture{Lineare Algebra und Analytische Geometrie I}
\Problemset{7}
\DoOn{3. Dezember 2023}
\author{Your Name}

\begin{document}
\maketitle

\begin{problem}[Unterräume von Vektorräumen]
Sei $K$ ein Körper, $V$ ein $K$-Vektorraum, und seien $U_1$, $U_2$ Unterräume von $V$. Zeigen Sie, dass $U_1 \cup U_2$ genau dann ein Unterraum von $V$ ist, wenn $U_1 \subseteq U_2$ oder $U_2 \subseteq U_1$ gilt.
\end{problem}

\begin{problem}[Lineare Unabhängigkeit in Vektorräumen]
Sei $K$ ein Körper, $V$ ein $K$-Vektorraum. Zeigen oder widerlegen Sie:
\begin{enumerate}
    \item Falls $v_1, \ldots, v_n \in V$ linear unabhängig sind und $\alpha_2, \ldots, \alpha_n \in K$, dann sind die Vektoren $v_1, v_2 + \alpha_2v_1, \ldots, v_n + \alpha_nv_1$ linear unabhängig.
    \item Falls $v_1, v_2, v_3 \in V$ linear unabhängig sind und $1 + 1 \neq 0$ in $K$, dann sind auch die folgenden Vektoren linear unabhängig: $v_{12} := v_1 + v_2$, $v_{23} := v_2 + v_3$, $v_{13} := v_1 + v_3$.
    \item Falls $v_1, v_2, v_3 \in V$ linear unabhängig sind und $1 + 1 = 0$ in $K$, dann sind auch die folgenden Vektoren linear unabhängig: $v_{12} := v_1 + v_2$, $v_{23} := v_2 + v_3$, $v_{13} := v_1 + v_3$.
    \item Für einen Unterkörper $K'$ von $K$ und linear unabhängige Vektoren $v_1, \ldots, v_n$ im $K'$-Vektorraum $V$, sind $v_1, \ldots, v_n$ auch linear unabhängig im $K$-Vektorraum $V$.
\end{enumerate}
\end{problem}

\begin{problem}[Lineare Abhängigkeit bei Funktionen]
Sei $V = \text{Abb}(\mathbb{R}^2, \mathbb{R})$ der $\mathbb{R}$-Vektorraum aller Abbildungen von $\mathbb{R}^2$ nach $\mathbb{R}$, und seien $f_1, f_2, f_3, f_4 \in \text{Abb}(\mathbb{R}^2, \mathbb{R})$ definiert durch
\[ f_1(x, y) := \max(x, y), \quad f_2(x, y) := \min(x, y), \quad f_3(x, y) := x, \quad f_4(x, y) := y. \]
Bestimmen Sie eine Basis von $U = \text{Span}\{f_1, f_2, f_3, f_4\} \subset V$.
\end{problem}

\pagebreak

\begin{problem}[Unterräume im $\mathbb{R}^3$]
Geben Sie für jeden der folgenden Unterräume des $\mathbb{R}$-Vektorraums $\mathbb{R}^3$ eine Basis an:
\begin{enumerate}
    \item $U_1 = \{(x_1, x_2, x_3) \in \mathbb{R}^3 \mid x_1 + x_2 + x_3 = 0\}$
    \item $U_2 = \text{Span}\{(2, -3, 1), (1, 1, 3), (-8, 17, 1)\}$
    \item $U_1 \cap U_2$
\end{enumerate}
\end{problem}
\end{document}
