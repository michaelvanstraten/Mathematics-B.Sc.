\documentclass{problemset}

\Lecture{Lineare Algebra und Analytische Geometrie I}
\Problemset{6}
\DoOn{26. November 2023 23:59 Uhr}
\author{Michael van Straten}

\begin{document}
\maketitle

\begin{problem}[Vektorraum über einem Ring]{5 Punkte}
Sei $(K, +, \cdot)$ ein Körper und sei $(R, \oplus, \otimes)$ ein Ring.
Die neutralen Elemente bzgl. $\cdot$ und $\otimes$ seien mit $1_K$ und $1_R$ bezeichnet.
Weiterhin sei $\varphi: K \to R$ ein einserhaltender Homomorphismus von Ringen, d.h., für alle $\alpha, \beta \in K$ gilt
\[
    \varphi(\alpha) \oplus \varphi(\beta) = \varphi(\alpha + \beta),
\]
\[
    \varphi(\alpha) \otimes \varphi(\beta) = \varphi(\alpha \cdot \beta),
\]
\[
    \varphi(1_K) = 1_R. \label{def:presevesone}
\]
Zeigen Sie, dass $(R, \oplus, \odot)$ einen Vektorraum über $K$ bildet, wobei
die Skalarmultiplikation wie folgt definiert ist:
\[
    \odot: K \times R \to R, (\alpha, x) \mapsto \varphi(\alpha) \otimes x. \label{def:scalar}
\]
\begin{proof}
    $(R, \oplus, \odot)$ bildet einen Vektorraum über $K$

    Um zu zeigen, das $(R, \oplus, \odot)$ einen Vektorraum über $K$ bildet muss
    gezeigt werden, dass $(R, \oplus)$ eine kommutative Gruppe bildet. Sowie das
    die folgenden Eigenschaften der Skalarmultiplikation erfüllt sind: \begin{align}
        \alpha \odot (u \oplus v) = (\alpha \odot u) \oplus (\alpha \odot v) \\
        (\alpha + \beta) \odot v = (\alpha \odot v) \oplus (\beta \odot v)   \\
        (\alpha \cdot \beta) \odot v = \alpha \odot (\beta \odot v)          \\
        1 \odot v = v
    \end{align}

    Da $(R, \oplus, \otimes)$ ein Ring ist, muss $(R, \oplus)$ eine kommutative
    Gruppe bilden, laut den Kriterien der Ring Struktur. \checkmark

    \textbf{1}: \begin{align*}
        \alpha \odot (u \oplus v) & \stackrel{\ref{def:scalar}}{=} \varphi(\alpha) \otimes (u \oplus v)     \\
                                  & = (\varphi(\alpha) \otimes u) \oplus (\varphi(\alpha) \otimes v)        \\
                                  & \stackrel{\ref{def:scalar}}{=} (\alpha \odot u) \oplus (\alpha \odot v)
    \end{align*}

    \textbf{2}: \begin{align*}
        (\alpha + \beta) \odot v & \stackrel{\ref{def:scalar}}{=} \varphi(\alpha + \beta) \otimes v       \\
                                 & = (\varphi(\alpha) \oplus \varphi(\beta)) \otimes v                    \\
                                 & = (\varphi(\alpha) \otimes v) \oplus (\varphi(\beta) \otimes v)        \\
                                 & \stackrel{\ref{def:scalar}}{=} (\alpha \odot v) \oplus (\beta \odot v)
    \end{align*}

    \textbf{3}: \begin{align*}
        (\alpha \cdot \beta) \odot v & \stackrel{\ref{def:scalar}}{=} \varphi(\alpha \cdot \beta) \otimes v \\
                                     & = \varphi(\alpha) \otimes \varphi(\beta) \otimes v                   \\
                                     & = \varphi(\alpha) \otimes (\varphi(\beta) \otimes v)                 \\
                                     & = \varphi(\alpha) \otimes (\beta \odot v)                            \\
                                     & = \alpha \odot (\beta \odot v)                                       \\
    \end{align*}

    \textbf{4}: Da $\varphi$ ein einserhaltender Homomorphismus ist, gilt folgendes $\varphi(1_K) \odot v \stackrel{\ref{def:presevesone}}{=} 1_R \otimes v = v$. \checkmark
\end{proof}
\end{problem}

\begin{problem}[Existenz einer Vektorraumstruktur]{5 Punkte}
Zeigen Sie, dass es keine Abbildung $\otimes: \mathbb{C} \times \mathbb{R} \to \mathbb{R}$ gibt, für die $(\mathbb{R}, +_{\mathbb{R}}, \otimes)$ ein Vektorraum über $\mathbb{C}$ ist und $\otimes|_{\mathbb{R} \times \mathbb{R}} = \cdot_{\mathbb{R}}$. Dabei sind $+_{\mathbb{R}}$ und $\cdot_{\mathbb{R}}$ die üblichen Operationen auf $\mathbb{R}$.

\textbf{Hinweis:} Zeigen Sie zunächst die Nullteilerfreiheit von $\otimes$, d.h., für $z \in \mathbb{C}$ und $v \in \mathbb{R}$ mit $z \otimes v = 0_{\mathbb{R}}$ gilt $z = 0_{\mathbb{C}}$ oder $v = 0_{\mathbb{R}}$.

\textbf{Hinweis:} Betrachten Sie $(z \otimes v - (z \cdot_{\mathbb{C}} v)) \otimes v$ für $v \in \mathbb{R}$, $v \neq 0_{\mathbb{R}}$ und $z \in \mathbb{C} \setminus \mathbb{R}$.

\begin{proof}
    Angenommen, eine solche Abbildung existierte, dann würde die Vektorraumstruktur vollständig durch den Wert von \(i \cdot 1\) bestimmt werden. Denn für jede von Null verschiedene reale Zahl \(\alpha\) und jede komplexe Zahl \(a+bi\) gilt
    \[
        (a+bi) \cdot \alpha = a \cdot \alpha + b \cdot (i \cdot \alpha) = a \alpha + b(i \cdot (\alpha \cdot 1)) = a \alpha + b \alpha (i \cdot 1).
    \]
    Angenommen, \(i \cdot 1 = r\). Dann wäre \((r-i) \cdot 1 = 0\), was im
    Widerspruch zu den Eigenschaften eines Vektorraums steht, da \(r-i \neq 0\) und
    \(1 \neq \mathbf{0}\). Daher existiert keine solche Vektorraumstruktur.
\end{proof}
\end{problem}

\begin{problem}[Beispiele für Untervektorräume]{5 Punkte}
\begin{enumerate}
    \item Sind die folgenden Teilmengen Untervektorräume von $\mathbb{R}^2$?
          \begin{enumerate}[label=\alph*)]
              \item $W_2 := \{(x_1, x_2) \in \mathbb{R}^2 \mid a_1x_1 + a_2x_2 = b\}$, $a_1, a_2, b \in \mathbb{R}$,
              \item $W_5 := \{(x_1, x_2) \in \mathbb{R}^2 \mid x_1x_2 \geq 0\}$.
          \end{enumerate}
    \item Sei $x \in \mathbb{R}$. Bezeichne $\mathbb{R}[x] := P[x]$ den Vektorraum der
          Polynome über $\mathbb{R}$. Sind die folgenden Teilmengen Untervektorräume von
          $R[x]$?
          \begin{enumerate}[label=\alph*)]
              \item $U_1 := \{p(x) \in R[x] \mid p(0) = 0\}$,
              \item $U_2 := \{p(x) \in R[x] \mid p(0) = 0 \text{ und } p(1) = 0\}$,
              \item $U_3 := \{p(x) \in R[x] \mid p(0) = 2\}$.
          \end{enumerate}
\end{enumerate}
\begin{proof}
    Um zu beweisen, das $U \subseteq V$ ein Untervektorraum von $V$ über $K$ ist, müssen die folgenden drei Eigenschaften gelten \begin{align}
        U \neq \emptyset,                       \\
        x, y \in U \Longrightarrow x + y \in U, \\
        a \in K \land x \in U \Longrightarrow a \odot v \in U.
    \end{align}

    \begin{enumerate}
        \item Untervektorräume von $\mathbb{R}^2$

              \begin{enumerate}[label=\alph*)]
                  \item $W_2$ ist ein Untervektorraum von $\mathbb{R}^2$ da:

                        \textbf{1}: $(0, 0) \in \mathbb{R}^2$ und $0 \in \mathbb{R}$ somit $a_1 \cdot 0 + a_2 \cdot 0 = 0 \Longrightarrow (0, 0) \in W_2$ was impliziert $W_2 \neq \emptyset$. \checkmark

                        \textbf{2}: $(x_1, x_2), (y_1, y_2) \in \mathbb{R}^2$ und $(x_1, x_2) + (y_1+y_2) = (x_1 + y_1, x_2 + y_2)$ sowie \[
                            a_1(x_1+y_1) + a_2(x_2+y_2) = b \in \mathbb{R},
                        \] da $a_1,a_2,x_1,x_2,y_1,y_2 \in \mathbb{R}$ und $\mathbb{R}$ Körper somit
                        geschlossen unter Addition und Multiplikation.

                        \textbf{3}: Nehme $s \in \mathbb{R}$ und $(x_1, x_2) \in \mathbb{R}^2$ somit folgt $s \odot (x_1, x_2) = (s \cdot x_1, s \cdot x_2)$ sowie \[
                            a_1(s \cdot x_1) + a_2(s \cdot x_2) = b \in \mathbb{R},
                        \] da $a_1,a_2, s,x_1,x_2 \in \mathbb{R}$ und $\mathbb{R}$ Körper somit
                        geschlossen unter Addition und Multiplikation.
                  \item $W_5$ ist kein Untervektorraum von $\mathbb{R}^2$ da:

                        \textbf{2}: $(1, 2) \in W_5$ und $(-2, -1) \in W_5$ aber $(1, 2) + (-2, -1) = (-1, 1) \not\in W_5$
              \end{enumerate}
        \item Untervektorräume von $\mathbb{R}^2$

              \begin{enumerate}
                  \item [a), b)] $U_1$ und $U_2$ sind Untervektorräume von $\mathbb{R}[x]$ da:

                        \textbf{1}: \textbf{Nullpolynom} $\in \mathbb{R}[x]$ und $\textbf{Nullpolynom}(x) = 0 \; \forall x \in \mathbb{R}$ somit $U_1, U_2 \neq \emptyset$. \checkmark

                        \textbf{2}: Nehme $p_1(x), p_2(x) \in \mathbb{R}[x]$ mit $p_1(a) = p_2(a) = 0, \forall a \in N \subseteq \mathbb{R}$.
                        Da $p_1(x)$ sowie $p_2(x)$ an jeder stelle, $a \in N$ gleich null ist und $0 + 0 = 0$ ist das Polynom $p_1(x) + p_2(x)$ ebenfalls an jeder stelle $a \in N$ gleich null.
                        Im fall $U_1$ entspricht $N = \set{0}$ und im Fall $U_2$ entspricht $N = \set{0, 1}$.
                        Somit ist die Vektoraddition von zwei Polynomen, die an denselben Stellen $N$ gleich null sind geschlossen ($U_1, U_2$). \checkmark

                        \textbf{3}: Nehme $a \in \mathbb{R}$ $p(x) \in \mathbb{R}[x]$ mit $p(b)= 0, \forall b \in N \subseteq \mathbb{R}$.
                        Da für alle $b \in N$ $p(b) = 0$ und $a * 0 = 0$ $\forall a \in \mathbb{R}$, ist auch das Polynom $a \odot p(x)$ an allen stellen $b \in N$ gleich null.
                        Im fall $U_1$ entspricht $N = \set{0}$ und im Fall $U_2$ entspricht $N = \set{0, 1}$.
                        Somit ist die Skalarmultiplikation eines Polynoms, welches an den Stellen $N$ gleich null ist geschlossen ($U_1, U_2$). \checkmark
                  \item [c)] $U_3$ ist kein Untervektorraum von $\mathbb{R}[x]$ da:

                        \textbf{2}: Nehme $p_1(x), p_2(x) \in U_3$ da $p_1(0) = p_2(0) = 2$ und $2 + 2 = 4$
                        ist das Polynom $p_1(x) + p_2(x)$ an der Stelle null, vier und nicht zwei, somit ist $p_1(x) + p_2(x) \not \in U_3$
                        woraus folgt $U_3$ ist kein Untervektorraum von $\mathbb{R}[x]$.
              \end{enumerate}
    \end{enumerate}

\end{proof}
\end{problem}

\begin{problem}[Eigenschaften von Unterräumen]{5 Punkte}
Sei $K$ ein Körper, $V$ ein $K$-Vektorraum und $U_1, U_2 \subseteq V$.
\begin{enumerate}
    \item Zeigen Sie eine der folgenden Aussagen:
          \begin{enumerate}[label=\alph*)]
              \item $U_1 \subseteq \operatorname{Span} U_1$,
              \item $U_1 \supseteq \operatorname{Span} U_1$, falls $U_1$ Untervektorraum von $V$,
              \item $\operatorname{Span} U_1 = \operatorname{Span} \operatorname{Span} U_1$,
              \item $\operatorname{Span} U_1 \subseteq \operatorname{Span} U_2$, falls $U_1 \subset U_2$.
          \end{enumerate}
    \item Zeigen oder widerlegen Sie zwei der folgenden Aussagen:
          \begin{enumerate}[label=\alph*)]
              \item $\operatorname{Span} U_1 \cap \operatorname{Span} U_2 \subseteq \operatorname{Span}(U_1 \cap U_2)$,
              \item $\operatorname{Span} U_1 \cap \operatorname{Span} U_2 \supseteq \operatorname{Span}(U_1 \cap U_2)$,
              \item $\operatorname{Span} U_1 \cup \operatorname{Span} U_2 \subseteq \operatorname{Span}(U_1 \cup U_2)$,
              \item $\operatorname{Span} U_1 \cup \operatorname{Span} U_2 \supseteq \operatorname{Span}(U_1 \cup U_2)$.
          \end{enumerate}
    \item Für $M, N \subseteq V$ ist $M + N := \{v_1 + v_2 \mid v_1 \in M, v_2 \in N\}$.
          Zeigen oder widerlegen

          Sie zwei der folgenden Aussagen:
          \begin{enumerate}[label=\alph*)]
              \item $\operatorname{Span}(U_1 + U_2) \subseteq \operatorname{Span} U_1 + \operatorname{Span} U_2$,
              \item $\operatorname{Span}(U_1 + U_2) \supseteq \operatorname{Span} U_1 + \operatorname{Span} U_2$,
              \item $\operatorname{Span}(U_1 + U_2) \supseteq \operatorname{Span} U_1 + \operatorname{Span} U_2$, falls $U_1$ und $U_2$ Untervektorräume von $V$.
          \end{enumerate}
\end{enumerate}
\begin{proof}
    Eigenschaften von Untervektorräumen
    \begin{enumerate}
        \item Zu zeigen ist das für $U_1 \subseteq V$ gilt $U_1 \subseteq$ Span $U_1$.

              Betrachten wir die definition der Menge Span $U_1 := \set{\lambda_1 v_1 +
                      \ldots + \lambda_n v_n \mid \lambda_{n \in \mathbb{N}} \in K \; v_{n \in
                              \mathbb{N}} \in U_1}$.

              Da $V$ ein Vektorraum ist $\exists 0 \in K$, sodass $\forall v \in V$ gilt $0
                  \times v = 0$, sowie ein $1 \in K$ sodass $\forall v \in V$ gilt $1 \times v =
                  v$.

              Nehmen wir uns einen beliebigen $v_u \in U_1$, somit existiert eine
              Linearkombination in Span $U_1$ mit \[
                  \lambda_1 v_1 + \ldots + \lambda_n v_n \mid \lambda_u = 1 \land \lambda_{n \ne u} = 0 \; v_{n \in \mathbb{N}} \in U_1 = 0 \times v_1 + \ldots + 1 \times v_u + \ldots 0 \times v_n = v_u.
              \]

              Da \(v_u\) beliebig gewählt wurde, gilt dies für jeden \(v \in U_1\), was zeigt
              das $U_1 \subseteq$ Span $U_1$. \checkmark
        \item Wiederlegung zweier Aussagen

              \textbf{a)}

              Lass \(U_1, U_2 \in R^2\) zwei triviale Basen mit \(U_1 = \left\{\begin{pmatrix}
                  1 \\ 0
              \end{pmatrix},
              \begin{pmatrix}
                  0 \\ 1
              \end{pmatrix}\right\}\) und \(U_2 = \left\{\begin{pmatrix}
                  2 \\ 0
              \end{pmatrix},
              \begin{pmatrix}
                  0 \\ 2
              \end{pmatrix}\right\}\).

              Bemerke das \(\operatorname{Span} U_1\) und \(\operatorname{Span} U_2\) gleich
              \(R^2\) da \(U_1, U_2\) Basen von \(R^2\) und \(U_1 \cap U_2 = \emptyset \).

              Somit ist der Schnitt von \(\operatorname{Span} U_1 = \operatorname{Span} U_2 =
              R^2\) gleich \(R^2\) und \(\operatorname{Span} \emptyset = \emptyset\).

              Da \(R^2 \not \subseteq \emptyset\), trivialerweise, ist die Aussagen
              \(\operatorname{Span} U_1 \cap \operatorname{Span} U_2 \subseteq
              \operatorname{Span}(U_1 \cap U_2)\) widerlegt. \checkmark

              \textbf{d)}

              Lass \(U_1, U_2 \in R^2\) mit \(U_1 = \left\{\begin{pmatrix}
                  1 \\0
              \end{pmatrix}\right\}\) und \(U_2 = \left\{\begin{pmatrix}
                  0 \\1
              \end{pmatrix}\right\}\).
              Somit bildet \(U_1 \cup U_2\) eine triviale Basis des \(R^2\) mit \(\operatorname{Span} U_1 \cup U_2 = R^2\).

              Es ist jedoch leicht zu sehen das der Vektor \(\begin{pmatrix}
                  1 \\ 1
              \end{pmatrix}\) nicht in der Vereinigung von \(\operatorname{Span} U_1\) und \(\operatorname{Span} U_2\) ist.
              \textbf{Begründung}: Naja, der \(\operatorname{Span} U_1\) und der \(\operatorname{Span} U_2\) bilden für sich die \(X\) und \(Y\)
              Achsen des \(R^2\) ab und somit kann ein Vektor in deren Vereinigung liegt entweder nur auf der \(X\) oder nur auf der \(Y\) Achse liegen.
              Der Vektor \(\begin{pmatrix}
                  1 \\ 1
              \end{pmatrix}\) liegt hingegen auf der Winkelhalbierenden von Q1 und kann somit nicht auf einer der beiden Achsen liegen. \checkmark
        \item Beweis oder Wiederlegung zweier Aussagen
              \begin{enumerate}[label=\alph*)]
                  \item $\operatorname{Span}(U_1 + U_2) \subseteq \operatorname{Span} U_1 + \operatorname{Span} U_2$ da

                  \item $\operatorname{Span}(U_1 + U_2) \not\supseteq \operatorname{Span} U_1 + \operatorname{Span} U_2$ da:

                        Nehme $v \in V$ und $U_1 = \set{v}$ $U_2 = \set {-v}$, somit ist $U_1 + U_2 =
                            \emptyset = \operatorname{Span} \emptyset$ aber $\operatorname{Span} U_1 +
                            \operatorname{Span} U_2 \neq \emptyset$. \checkmark
              \end{enumerate}
    \end{enumerate}
\end{proof}
\end{problem}
\end{document}
