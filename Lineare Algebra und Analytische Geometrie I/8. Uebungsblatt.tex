\documentclass{problemset}

\Lecture{Lineare Algebra und Analytische Geometrie I}
\Problemset{8}
\DoOn{10. Dezember 2023}
\author{Michael van Straten}

\begin{document}
\maketitle

\begin{problem}[Basen endlichdimensionaler Vektorräume]{3 Punkte}
Sei $n \in \mathbb{N}$, $V$ ein $K$-Vektorraum mit $\dim_K(V) = n$ und $B = \{v_1, \ldots, v_n\} \subseteq V$. Zeigen Sie, dass die folgenden Aussagen äquivalent sind:
\begin{enumerate}
    \item $B$ ist eine Basis.
    \item $B$ ist linear unabhängig.
    \item $B$ ist ein Erzeugendensystem.
\end{enumerate}

\begin{proof}
    $ $

    \begin{itemize}
        \item [	,,i) $\Longrightarrow$ ii)'':]
              Da $B$ eine Basis ist, ist $B$ linear unabhängig und ein Erzeugendensystem.
        \item [,,ii) $\Longrightarrow$ iii)'':]
              Angenommen $B$ ist kein Erzeugendensystem.

              Da $B$ linear unabhängig ist kann $B$ laut des Basisergänzungssatzes zu einem
              Erzeugendensystem ergänzt werden. Ausgehend hat $B$ $n$ Elemente, nach der
              Ergänzung von $B$ zu einem Erzeugendensystem hat $B$ mindestens $n + 1$
              Elemente. Somit haben wir ein linear unabhängiges Erzeugendensystem, eine
              Basis, mit einer Mindestsanzahl an $n + 1$ Elementen. Daraus schließen wir das
              $\dim_K(V) \ge n + 1$, dies steht aber im Wiederspruch mit der Annahme das
              $\dim_K(V) = n$. Somit muss unsere Annahme das $B$ kein Erzeugendensystem ist
              falsch zein. \checkmark
        \item [	,,iii) $\Longrightarrow$ i)'':] Angenommen $\dim_K(V) = n$
              \begin{itemize}
                  \item [Fall 1:]
                        $B$ ist linear unabhängig $\Rightarrow$ $B$ ist eine Basis von $V$ \checkmark
                  \item [Fall 2:]
                        $B$ ist linear abhängig $\Rightarrow$ $\exists w \in B$ mit $w = \lambda_1 v_1 + \lambda_2 v_2 + ... + \lambda_{n-1} v_{n - 1}$ $v_i \in B \setminus \set{w}$

                        Für denn Fall das $B \setminus \set w$ nicht linear unabhängig ist wieder hole
                        den Prozess.

                        In beiden fällen, $B \setminus \set w$ linear unabhängig, $B \setminus \set w$
                        linear abhängig, erhalten wir eventuell ein Erzeugendensystem welches Lineare
                        unabhängig ist und weniger als $n$ Elemente hat.

                        Dies steht allerdings im Widerspruch mit der Annahme das $\dim_K(V) = n$, was
                        impliziert das $B$ nicht linear abhängig sein kann.
              \end{itemize}

              Da wir Fall 2 somit ausschließen können, folgt $B$ ist eine Basis. \checkmark
    \end{itemize}
\end{proof}
\end{problem}

\begin{problem}[Vektoren mit 0 an der i-ten Stelle]{3 Punkte}
Sei $K \in \{ \mathbb{R}, \mathbb{C} \}$ und $n \in \mathbb{N}$ mit $n > 1$. Weiter seien $\bar{e}_i := (1, \ldots, 1, 0, 1, \ldots, 1) \in K^n$, $i \in \{1, \ldots, n\}$ die Vektoren mit 0 an der $i$-ten Stelle und 1 sonst. Berechnen Sie $\dim_K(\operatorname{Span}\{\bar{e}_i \mid i \in \{1, \ldots, n\}\})$.

\begin{proof}
    Wir wählen die Vektoren $\bar{e}_i$ als potenzielle Basis für $\operatorname{Span}\{\bar{e}_i \mid i \in \{1, \ldots, n\}\}$ und überprüfen deren lineare Unabhängigkeit.

    Angenommen, es existieren Skalare $\alpha_i \in K$ mit $\sum_{i=1}^{n} \alpha_i
        \bar{e}_i = 0$. Dies führt zu folgendem linearen Gleichungssystem:
    \begin{align}
        0 + \alpha_2 + \alpha_3 + \ldots + \alpha_n                & = 0             \\
        \alpha_1 + 0 + \alpha_3 + \ldots + \alpha_n                & = 0             \\
        \vdots                                                          \tag*{}      \\
        \alpha_1 + \alpha_2 + \alpha_3 + \ldots + 0 + \alpha_n     & = 0 \tag{n - 1} \\
        \alpha_1 + \alpha_2 + \alpha_3 + \ldots + \alpha_{n-1} + 0 & = 0 \tag{n}
    \end{align}

    Daraus ergibt sich durch Subtraktion der Gleichungen $(i)-(i+1)$:
    \begin{align*}
        -\alpha_1 + \alpha_2     & = 0 \implies \alpha_1 = \alpha_2     \\
        -\alpha_2 + \alpha_3     & = 0 \implies \alpha_2 = \alpha_3     \\
        \vdots                                                          \\
        -\alpha_{n-1} + \alpha_n & = 0 \implies \alpha_{n-1} = \alpha_n
    \end{align*}

    Durch sukzessives Einsetzen in die vorherige Gleichungskette erhalten wir
    $\alpha_1 = \alpha_2 = \ldots = \alpha_n = 0$.

    Dies zeigt, dass die Vektoren $\bar{e}_i$ linear unabhängig sind und somit eine
    Basis bilden. Da die Basis $n$ Elemente hat, gilt
    $\dim_K(\operatorname{Span}\{\bar{e}_i \mid i \in \{1, \ldots, n\}\}) = n$.
\end{proof}
\end{problem}

\begin{problem}[Unterraum von Polynomen]{6 Punkte}
Bezeichne $R\leq 3[t]$ den Vektorraum der Polynome über $\mathbb{R}$ mit Grad höchstens drei. Sei weiter $U := \{p(t) \in R\leq 3[t] \mid p(0) = p(1) = 0\}$.
\begin{enumerate}
    \item Zeigen Sie, dass $U$ ein Unterraum von $R\leq 3[t]$ ist.
    \item Bestimmen Sie eine Basis von $U$ und berechnen Sie $\dim_{\mathbb{R}}(U)$.
    \item Geben Sie Polynome an, welche die in ii) angegebene Basis von $U$ zu einer
          Basis von $R\leq 3[t]$ erweitern.
\end{enumerate}
\begin{proof}
    $ $

    \begin{enumerate}
        \item Um zu zeigen, dass $U$ ein Unterraum von $\mathbb{R}_{\leq 3}[t]$ ist, müssen
              wir zeigen, dass $U$ unter Vektoraddition und Skalarmultiplikation
              abgeschlossen ist.

              \textbf{Vektoraddition}:

              Seien $p(x), q(x) \in U$. Da $p(x), q(x)$ an den Stellen $x = 0, x = 1$ gleich
              0 sind, folgt $p(x) + q(x)$ ebenfalls an den Stellen $x = 0, x = 1$ gleich 0.
              Somit ist $U$ unter Vektoraddition geschlossen. \checkmark

              \textbf{Skalarmultiplikation}:

              Sei $p(x) \in U$ und $\alpha \in \mathbb{R}$. Da $\alpha \cdot 0 = 0$ ist, ist
              auch $\alpha p(x)$ an jeder Stelle, an der $p(x)$ gleich null ist, ebenfalls
              gleich null. Somit ist $U$ unter Skalarmultiplikation geschlossen. \checkmark
        \item Betrachten wir einen Vektor $p(t)$ in $\mathbb{R}_{\leq 3}[t]$, der die Form
              $p(t) = \alpha_0 + \alpha_1t + \alpha_2t^2 + \alpha_3t^3$ hat. Unter
              Berücksichtigung der Bedingungen des Unterraums $U$ ergibt sich die Struktur
              eines Vektors in $U$ als $(-\alpha_2 - \alpha_3)t + \alpha_2t^2 + \alpha_3t^3$.

              Nun wählen wir zwei Vektoren aus $U$ als potenzielle Basis und überprüfen, ob
              sie linear unabhängig sind und ob $\operatorname{Span} B = U$ gilt.

              \[
                  B := \{2t - t^2 - t^3, t^2 - t^3\}
              \]

              Wir setzen

              \[
                  \lambda_1 \cdot (2t - t^2 - t^3) + \lambda_2 \cdot (t^2 - t^3) = 0
              \]

              \textbf{Setze $t = 2$}:

              \begin{align*}
                  \lambda_1 \cdot (4 - 4 - 8) + \lambda_2 \cdot (4 - 8) & = 0            \\
                  -8 \lambda_1 - 4 \lambda_2                            & = 0            \\
                  \lambda_2                                             & = -2 \lambda_1
              \end{align*}

              \textbf{Setze $t = -1$}:

              \begin{align*}
                  \lambda_1 \cdot (-2 - 1 + 1) + \lambda_2 \cdot (1 + 1) & = 0         \\
                  -2 \lambda_1 + 2 \lambda_2                             & = 0         \\
                  \lambda_1                                              & = \lambda_2
              \end{align*}

              Da $\lambda_1 = \lambda_2$ und $\lambda_2 = -2 \lambda_1$ folgt, dass
              $\lambda_1 = \lambda_2 = 0$.

              Daher ist die Menge $B$ linear unabhängig.

              Wählen wir nun einen generischen Vektor aus $U$ und sehen, ob er sich durch $B$
              darstellen lässt.

              \[
                  (-\alpha_2 - \alpha_3)t + \alpha_2t^2 + \alpha_3t^3 = \lambda_1 \cdot (2t - t^2 - t^3) + \lambda_2 \cdot (t^2 - t^3)
              \]

              \textbf{Setze $t = 2$}:
              \begin{align*}
                  2(-\alpha_2 - \alpha_3) + 4\alpha_2 + 8\alpha_3 & = -8 \lambda_1 - 4 \lambda_2 \\
                  2\alpha_2 + 6\alpha_3                           & =                            \\
                  \alpha_2 + 3\alpha_3                            & = -4 \lambda_1 - 2 \lambda_2
              \end{align*}

              \textbf{Setze $t = -1$}:
              \begin{align*}
                  (-1)(-\alpha_2 - \alpha_3) + \alpha_2 - \alpha_3 & = -2 \lambda_1 + 2 \lambda_2 \\
                  2\alpha_2                                        & =                            \\
                  \alpha_2                                         & = - \lambda_1 + \lambda_2
              \end{align*}

              Somit erhalten wir \[
                  \lambda_1 = - \frac{1}{2}\alpha_2 - \frac{1}{2}\alpha_3
              \]
              \[
                  \lambda_2 = \frac{1}{2}\alpha_2 - \frac{1}{2}\alpha_3
              \]

              Somit ist $\operatorname{Span} B = U$

              Somit ist $\dim_{\mathbb{R}}(U) = 2$, da $B$ zwei Elemente enthält.
        \item Um die Basis $B$ von $U$ zu einer Basis von $\mathbb{R}_{\leq 3}[t]$ zu
              erweitern, wählen wir $1, t \in \mathbb{R}_{\leq 3}[t]$. Damit ist $B \cup \{1,
                  t\}$ die erweiterte Basis.
    \end{enumerate}
\end{proof}
\end{problem}

\begin{problem}[Endliche Körper und Vektorräume]{8 Punkte}
\begin{enumerate}
    \item Sei $(L, +, \cdot)$ ein Körper und $K \subseteq L$ ein Unterkörper. Zeigen Sie,
          dass $(L, +, \cdot)$ ein $K$-Vektorraum ist.
    \item Betrachten Sie die Menge $\mathbb{Z}_4 := \{0, 1, 2, 3\}$ sowie
          $\operatorname{mod}_4 : \mathbb{Z} \to \mathbb{Z}_4$, $z \mapsto y$ mit $\{y\}
              = [z] \equiv 4 \cap \mathbb{Z}_4$, das heißt $\operatorname{mod}_4(z)$ wählt
          den (eindeutigen) Repräsentanten von $[z] \equiv 4$ aus der in $\mathbb{Z}_4$
          ist. Seien $+_4 : \mathbb{Z}_4 \times \mathbb{Z}_4 \to \mathbb{Z}_4$ und
          $\cdot_4 : \mathbb{Z}_4 \times \mathbb{Z}_4 \to \mathbb{Z}_4$ definiert durch
          \[ x +_4 y := \operatorname{mod}_4(x + y), \quad x \cdot_4 y := \operatorname{mod}_4(x \cdot y). \]
          Zeigen Sie, dass $(\mathbb{Z}_4, +_4, \cdot_4)$ kein Körper ist.
    \item Sei $F_4 := \{0, 1, a, b\}$ und seien $+_F4 : F_4 \times F_4 \to F_4$ und
          $\cdot_F4 : F_4 \times F_4 \to F_4$ definiert durch:
          \[ \begin{array}{c|cccc}
                  +_{F_4} & 0 & 1 & a & b \\
                  \hline
                  0       & 0 & 1 & a & b \\
                  1       & 1 & 0 & b & a \\
                  a       & a & b & 0 & 1 \\
                  b       & b & a & 1 & 0 \\
              \end{array} \quad
              \begin{array}{c|cccc}
                  \cdot_{F_4} & 0 & 1 & a & b \\
                  \hline
                  0           & 0 & 0 & 0 & 0 \\
                  1           & 0 & 1 & a & b \\
                  a           & 0 & a & b & 1 \\
                  b           & 0 & b & 1 & a \\
              \end{array} \]
          Sei weiter $F_2 := \{0, 1\} \subseteq F_4$. Sie können ohne Beweis annehmen,
          dass $(F_4, +_{F_4}, \cdot_{F_4})$ ein Körper ist und $F_2$ ein Unterkörper von
          $F_4$ ist. Geben Sie eine Basis von $F_4$ als Vektorraum über $F_2$ an und
          berechnen Sie $\dim_{F_2}(F_4)$.
    \item Berechnen Sie alle Nullstellen des Polynoms $X^2 + X + 1$ über $F_2$ und $F_4$.
\end{enumerate}
\begin{proof}
    $ $

    \begin{enumerate}
        \item Sei $(L, +, \cdot)$ ein Körper und $K \subseteq L$ ein Unterkörper. Wir wollen
              zeigen, dass $(L, +, \cdot)$ ein $K$-Vektorraum ist.

              \begin{enumerate}
                  \item \textbf{Abgeschlossenheit unter der Addition:} Für alle $u, v \in L$ ist auch $u + v \in L$.

                        Dies folgt direkt aus der Definition der Addition im Körper $L$.

                  \item \textbf{Assoziativität der Addition:} Für alle $u, v, w \in L$ gilt $(u + v) + w = u + (v + w)$.

                        Dies folgt direkt aus der Assoziativität der Addition im Körper $L$.

                  \item \textbf{Existenz des neutralen Elements:} Es gibt ein neutrales Element $0 \in L$ bezüglich der Addition, so dass für alle $u \in L$ gilt $u + 0 = u$.

                        Dies folgt direkt aus der Existenz des neutralen Elements in $L$.

                  \item \textbf{Existenz des inversen Elements:} Für jedes $u \in L$ gibt es ein inverses Element $-u \in L$ bezüglich der Addition, so dass $u + (-u) = 0$.

                        Dies folgt direkt aus der Existenz des inversen Elements in $L$.

                  \item \textbf{Kommutativität der Addition:} Für alle $u, v \in L$ gilt $u + v = v + u$.

                        Dies folgt direkt aus der Kommutativität der Addition in $L$.

                  \item \textbf{Abgeschlossenheit unter der skalaren Multiplikation:} Für alle $a \in K$ und $u \in L$ ist auch $a \cdot u \in L$.

                        Dies folgt aus der Tatsache, dass $K$ ein Unterkörper von $L$ ist.

                  \item \textbf{Assoziativität der skalaren Multiplikation:} Für alle $a, b \in K$ und $u \in L$ gilt $(a \cdot b) \cdot u = a \cdot (b \cdot u)$.

                        Dies folgt aus der Assoziativität der Multiplikation in $K$.

                  \item \textbf{Distributivität des Skalars bezüglich der Vektoraddition:} Für alle $a \in K$ und $u, v \in L$ gilt $a \cdot (u + v) = a \cdot u + a \cdot v$.

                        Dies folgt aus der Distributivität der Multiplikation in $K$.

                  \item \textbf{Distributivität des Skalars bezüglich der Skalarmultiplikation:} Für alle $a, b \in K$ und $u \in L$ gilt $(a + b) \cdot u = a \cdot u + b \cdot u$.

                        Dies folgt ebenfalls aus der Distributivität der Multiplikation in $K$.

                  \item \textbf{Existenz des neutralen Elements:} Es gibt ein neutrales Element $1 \in K$ bezüglich der skalaren Multiplikation, so dass für alle $u \in L$ gilt $1 \cdot u = u$.

                        Dies folgt direkt aus der Definition des neutralen Elements in $K$.
              \end{enumerate}

              Da alle zehn Axiome erfüllt sind, ist $(L, +, \cdot)$ ein $K$-Vektorraum.

        \item Betrachten wir die Multiplikationstabelle für $\field{Z}_4$
              \[
                  \begin{array}{c|cccc}
                      \cdot_{\field{Z}_4} & 0 & 1 & 2 & 3 \\
                      \hline
                      0                   & 0 & 0 & 0 & 0 \\
                      1                   & 0 & 1 & 2 & 3 \\
                      2                   & 0 & 2 & 0 & 2 \\
                      3                   & 0 & 3 & 2 & 1 \\
                  \end{array}
              \]

              Wir bemerken wir das es für $2 \in \field{Z}_4$ keinen Multiplikativen inverses
              $2^{-1}$ gibt mit welchem $2 \cdot 2^{-1} = 1$.

              Somit ist $\field{Z}_4$ kein Körper. \checkmark
        \item Wählen wir $B := \set{a, b}$ als eine potenzielle Basis für $\field{F}_4$.

              Zu zeigen ist das $B$ linear unabhängig über $\field{F}_2$ ist sowie das
              $\operatorname{Span} B = \field{F}_4$.

              Da $\field{F}_2$ lediglich zwei Elemente besitzt, gibt es nur 4 möglich linear
              Kombinationen der Elemente aus $B$.

              \textbf{Fall 1: $\lambda_1 = 0, \lambda_2 = 0$}
              \[
                  0 \cdot a + 0 \cdot b = 0 + 0 = 0
              \]
              \textbf{Fall 2: $\lambda_1 = 1, \lambda_2 = 0$}
              \[
                  1 \cdot a + 0 \cdot b = a + 0 = a \neq 0
              \]
              \textbf{Fall 3: $\lambda_1 = 0, \lambda_2 = 1$}
              \[
                  0 \cdot a + 1 \cdot b = 0 + b = b \neq 0
              \]
              \textbf{Fall 4: $\lambda_1 = 1, \lambda_2 = 1$}
              \[
                  1 \cdot a + 1 \cdot b = a + b = 1 \neq 0
              \]

              Somit folgt aus $\lambda_1 \cdot a + \lambda_2 \cdot b = 0$ das $\lambda_1 =
                  \lambda_2 = 0$, was impliziert $B$ ist Lineare unabhängig. \checkmark

              Um zu zeigen das $\operatorname{Span} B = \field{F}_4$, können wir zeigen das
              es für alle $v \in \field{F}_4$ $\lambda_1, \lambda_2 \in \field{F}_2$ mit
              $\lambda_1 \cdot a + \lambda_2 \cdot b = v$.

              \textbf{Fall $v = 0 \Rightarrow \lambda_1 = 0, \lambda_2 = 0$}:
              \[
                  0 \cdot a + 0 \cdot b = 0 + 0 = 0
              \]
              \textbf{Fall $v = 1 \Rightarrow \lambda_1 = 1, \lambda_2 = 1$}:
              \[
                  1 \cdot a + 1 \cdot b = a + b = 1
              \]
              \textbf{Fall $v = a \Rightarrow \lambda_1 = 1, \lambda_2 = 0$}:
              \[
                  1 \cdot a + 0 \cdot b = a + 0 = a
              \]
              \textbf{Fall $v = b \Rightarrow \lambda_1 = 0, \lambda_2 = 1$}:
              \[
                  0 \cdot a + 1 \cdot b = 0 + b = b
              \]

              Somit haben wir gezeigt das $B$ linear unabhängig ist und das
              $\operatorname{Span} B = \field{F}_4$, was zeigt $B$ ist Basis.
        \item Da $\field{F}_2$ und $\field{F}_4$ jeweils nur wenige endliche Elemente haben
              lässt sich per hand überprüfen ob diese als Nullstellen für das Polynom $X^2 +
                  X + 1$ infragekommen.

              \textbf{$\field{F}_2$:}
              \begin{itemize}
                  \item [$X = 0$:]
                        \[
                            0^2 + 0 + 1 = 0 \cdot 0 + 0 + 1 = 1 \neq 0
                        \]

                        \textit{keine Nullstelle}

                  \item [$X = 1$:]
                        \[
                            1^2 + 1 + 1 = 1 + 1 + 1 = 0 + 1 = 1 \neq 0
                        \]

                        \textit{keine Nullstelle}

              \end{itemize}

              Somit hat das Polynom $X^2 + X + 1$ keine Nullstellen über $\field{F}_2$.

              \textbf{$\field{F}_4$:}

              Für $X = 0$ und $X = 1$ wissen wir bereits das sie keine Nullstellen über
              $\field{F}_2$ sind und da $\field{F}_2$ unterkörper von $\field{F}_4$ sind
              diese auch keine Nullstellen in $\field{F}_4$.
              \begin{itemize}
                  \item [$X = a$:]
                        \[
                            a^2 + a + 1 = a \cdot a + a + 1 = b + a + 1 = 1 + 1 = 0
                        \]

                        \textit{Nullstelle} \checkmark

                  \item [$X = b$:]
                        \[
                            b^2 + b + 1 = b \cdot b + b + 1 = a + b + 1 = 1 + 1 = 0
                        \]

                        \textit{Nullstelle} \checkmark

              \end{itemize}

              Somit hat das Polynom $X^2 + X + 1$ zwei Nullstellen über $\field{F}_4$ mit
              $x_0 \in \set{a,b}$.
    \end{enumerate}
\end{proof}
\end{problem}

\end{document}
