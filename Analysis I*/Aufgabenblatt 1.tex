\documentclass{exam}

\usepackage{amsthm, amsmath, amsfonts}
\usepackage[german]{babel}

\title{Analysis I* - Aufgabenblatt 1}
\author{Michael van Straten}
\begin{document}
\maketitle
\section*{1. Summenformeln}
Beweisen Sie folgende Summenformeln fur alle $n \in \mathbb{N}$:
\begin{enumerate}
	\item[a)]
	      \[
		      \sum_{k = 1}^{n} k^3 = \left(\sum_{k = 1}^{n}k\right)^2,
	      \]
	\item[b)]
	      \[
		      \sum_{k = 1}^{n-1} k^2(n-k)^2 = \frac{n(n^4-1)}{30}.
	      \]
\end{enumerate}
\begin{proof}
	Für alle $n \in \mathbb{N}$ gilt $\sum_{k = 1}^{n} k^3 = \left(\sum_{k = 1}^{n}k\right)^2$.
	\begin{enumerate}
		\item[a)] \underline{Anfang}: $A(0)$ ist war: $Leere\ Summe = 0 = (Leere\ Summme)^2$
		\item[b)] \underline{Schritt}: Sei $n \ge n_0$. Angenommen $A(n)$ sei schon bewiesen.
		      Dann ist \begin{align}
			      \sum_{k = 1}^{n + 1} k^3 & = \sum_{k = 1}^{n}k^3 + (n + 1)^3                                           \\
			                               & = \left(\sum_{k = 1}^{n}k\right)^2 + (n + 1)^3 \tag{Gausische summenformel} \\
			                               & = \left(\frac{n(n+1)}{2}\right)^2 + (n + 1)^3                               \\
			                               & = \frac{n^2(n+1)^2}{4} + (n + 1)^3                                          \\
			                               & = \frac{n^2(n+1)^2}{4} + \frac{4(n + 1)^3}{4}                               \\
			                               & = \frac{n^2(n+1)^2 + 4(n + 1)^3}{4}                                         \\
			                               & = \frac{(n+1)^2(n^2 + 4(n + 1))}{4}                                         \\
			                               & = \frac{(n+1)^2(n^2 + 4n + 4))}{4}                                          \\
			                               & = \frac{(n+1)^2(n +2)^2}{4}                                                 \\
			                               & = \left(\frac{(n+1)(n +2)}{2}\right)^2                                      \\
			                               & = \left(\sum_{k=1}^{n+1}k\right)^2
		      \end{align}
	\end{enumerate}
	Dann folgt aus a), b), dass $A(n)$ für alle $n \ge n_0$ wahr ist.
\end{proof}
\begin{proof}
	Für alle $n \in \mathbb{N}$ gilt $\sum_{k = 1}^{n - 1} k^2(n - k)^2 = \frac{n(n^4 - 1)}{30}$.
	\begin{enumerate}
		\item[a)] \underline{Anfang}: $A(1)$ ist war: $Leere\ Summe = \frac{1(1^4 - 1)}{30} = 0$
		\item[b)] \underline{Schritt}: Sei $n \ge n_0$. Angenommen $A(n)$ sei shon beweisen: Dann ist
		      \begin{align}
			      \sum_{k = 1}^{n} k^2(n + 1 - k)^2 & = \sum_{k = 1}^{n} k^2(n - k + 1)^2                                                      \\
			                                        & = \sum_{k = 1}^{n} k^2((n - k)^2 + 2(n-k)+ 1)                                            \\
			                                        & = \sum_{k = 1}^{n} k^2(n - k)^2 + 2k^2(n-k) + k^2                                        \\
			                                        & = \sum_{k = 1}^{n} k^2(n - k)^2 + \sum_{k=1}^{n} 2k^2(n-k) + k^2                         \\
			                                        & = \frac{n(n^4 - 1)}{30} + \sum_{k=1}^{n} 2k^2(n-k) + k^2                                 \\
			                                        & = \frac{n(n^4 - 1)}{30} + \sum_{k=1}^{n} 2nk^2 - \sum_{k=1}^{n} k^3 + \sum_{k=1}^{n} k^2 \\
			                                        & = \frac{n(n^4 - 1)}{30} + 2n\sum_{k=1}^{n} k^2 - \sum_{k=1}^{n} k^3 + \sum_{k=1}^{n} k^2 \\
			                                        & = \frac{n(n^4 - 1)}{30} + 3n\sum_{k=1}^{n} k^2 - \sum_{k=1}^{n} k^3                      \\
			                                        & = \frac{n(n^4 - 1)}{30} + 3n\sum_{k=1}^{n} k^2 - \sum_{k=1}^{n} k^3                      \\
		      \end{align}
	\end{enumerate}
\end{proof}

\section*{Aufgabe 2:}
Wir ergänzen nun die Definition 2.9 des Binomialkoeffizienten $n \choose k$ mit $n \in \mathbb{N}$ und $k \in \mathbb{Z}$ aus der Vorlesung zu $x \choose v$ mit $x \in \mathbb{R}$ und $v \in \mathbb{N}$:
\[
	\binom{x}{v} := \prod_{j=1}^{v}\frac{x-j+1}{j} = \frac{x(x-1)\cdot\ldots\cdot(x-v+1)}{v!}.
\]
Es sein $x \in \mathbb{R}$, $v,l,k \in \mathbb{N}$ und $l \le k$. Beweisen Sie die folgenden Formeln für die
Binomialkoeffizienten $\binom{x}{v}$:
\begin{proof}
	a)
	\begin{displaymath}
		\binom{-x}{v} = (-1)^v\binom{x+v-1}{v}
	\end{displaymath}
	\begin{align}
		\binom{-x}{v} & = \frac{(-x)(-x-1)\cdot\ldots\cdot(-x-v+1)}{v!}                                                 \\
		              & = \frac{(-1)(x)(-1)(x+1)\cdot\ldots\cdot(-1)(x+v-1)}{v!}                                        \\
		              & = \frac{\overbrace{(-1)\cdot\ldots\cdot(-1)}^{v\text{ mal}}(x)(x+1)\cdot\ldots\cdot(x+v-1)}{v!} \\
		              & = \frac{(-1)^v(x)(x+1)\cdot\ldots\cdot(x+v-1)}{v!}                                              \\
		              & = (-1)^v\frac{(x)(x+1)\cdot\ldots\cdot(x+v-1)}{v!}                                              \\
		              & = (-1)^v\frac{(x+v-1)\cdot\ldots\cdot(x+1)(x)}{v!}                                              \\
		              & = (-1)^v\binom{x+v-1}{v}
	\end{align}
\end{proof}
\begin{proof}
	b)
	\begin{displaymath}
		\binom{x+1}{v+1} &= \binom{x}{v}\frac{x+1}{v+1} \\
	\end{displaymath}
	\begin{align}
		\binom{x+1}{v+1} & = \frac{(x+1)x(x-1)\cdot\ldots\cdot(x-v+1)}{(v+1)!}        \\
		                 & = \frac{(x+1)x(x-1)\cdot\ldots\cdot(x-v+1)}{(v+1)v!}       \\
		                 & = \frac{x+1}{v+1} \frac{x(x-1)\cdot\ldots\cdot(x-v+1)}{v!} \\
		                 & = \frac{x+1}{v+1} \binom{x}{v}                             \\
		                 & = \binom{x}{v} \frac{x+1}{v+1}
	\end{align}

\end{proof}
\begin{proof}
	c)
	\begin{displaymath}
		\binom{x}{v+1} = \binom{x}{v}\frac{x-v}{v+1}
	\end{displaymath}
	\begin{align}
		\binom{x}{v+1} & = \prod_{j=1}^{v+1}\frac{x-j+1}{j}                    \\
		               & = \prod_{j=1}^{v}\frac{x-j+1}{j}\frac{x-(v+1)+1}{v+1} \\
		               & = \binom{x}{v}\frac{x-v-1+1}{v+1}                     \\
		               & = \binom{x}{v}\frac{x-v}{v+1}
	\end{align}
\end{proof}
\begin{proof}
	d)
	\begin{displaymath}
		\binom{x}{k}\binom{k}{l} = \binom{x}{l}\binom{x-l}{k-l}
	\end{displaymath}
	\begin{align}
		\binom{x}{k}\binom{k}{l} & = \frac{x(x-1)\cdot\ldots\cdot(x-k+1)}{k!}\frac{k(k-1)\cdot\ldots\cdot(k-l+1)}{l!}
	\end{align}
\end{proof}

\section*{Aufgabe 3:}
Beweisen Sie 3.14 a) und b) aus der Vorlesung:
\newline\newline
Für alle $n,m \in \mathbb{Z}$ und $x \in \mathbb{R}$ bzw. $x \in \mathbb{R}^* = \mathbb{R}\setminus\{0\}$ im Falle negativer Potenzen gilt
\begin{proof}
	a)
	\begin{displaymath}
		x^nx^m = x^{n+m}
	\end{displaymath}
	\textbf{Fall $n,m \ge 0$}
	\begin{align}
		x^nx^m & = \overbrace{x \cdot\ldots\cdot x}^{n\text{ mal}} \cdot x^m                                                                           \\
		       & = \overbrace{x \cdot\ldots\cdot x}^{n\text{ mal}} \cdot \overbrace{x \cdot\ldots\cdot x}^{m\text{ mal}}                               \\
		       & = \underbrace{\overbrace{x \cdot\ldots\cdot x}^{n\text{ mal}} \cdot \overbrace{x \cdot\ldots\cdot x}^{m\text{ mal}}}_{n+m\text{ mal}}
	\end{align}
\end{proof}
\begin{proof}
	b)
	\begin{displaymath}
		{(x^n)}^m = x^{nm}
	\end{displaymath}
\end{proof}

\end{document}
