\documentclass{../problemset}

\Lecture{Analysis I} % the lecture
\Problemset{4} % the problemset number
\DoOn{21.11.2023} % the date it is due on
\author{Michael van Straten}


\setlist[enumerate, 1]{label=\alph*)}

\begin{document}
\maketitle

% Problem 1 %
\begin{problem}{Ungleichungen mit Wurzeln und zwei Anwendungen}{8 Punkte}
% problem description %
\begin{enumerate}
	% subproblem 1a %
	\item Beweisen Sie: Für alle natürlichen Zahlen $n \geq 2$ gilt
	      \[
		      \sum_{k=1}^{n} \frac{1}{\sqrt{k}} > \sqrt{n}.
	      \]

	      % subproblem 1b %
	\item Zeigen Sie mithilfe von \textbf{a)}, dass \[
		      \sum_{k=1}^{\infty} \frac{1}{\sqrt{k}}
	      \] bestimmt gegen $+\infty$ divergiert.

	      % subproblem 1c %
	\item Beweisen Sie: Für alle natürlichen Zahlen $n \geq 1$ gilt
	      \[
		      \sqrt{n} \leq 1 + 2\sqrt{n}.
	      \]

	      % subproblem 1d %
	\item Zeigen Sie mithilfe von c), dass
	      \[
		      \lim_{{n\to\infty}} \frac{n}{\sqrt{n}} = 1.
	      \]
\end{enumerate}

\begin{proof}
	\begin{enumerate}
		\item \textit{Beweis durch Induktion}, angenommen \textbf{a)} ist bereits für $n=2$ bewiesen, so folgt: \begin{align*}
			      \sum_{k=1}^{n+1} \frac{1}{\sqrt{k}} = \sum_{k=1}^{n} \frac{1}{\sqrt{k}} + \frac{1}{\sqrt{n + 1}} & > \sqrt{n} + \frac{1}{\sqrt{n + 1}}                                              \\
			                                                                                                       & = \sqrt{n+1} \frac{1}{\sqrt{n+1}} \left(\sqrt{n} + \frac{1}{\sqrt{n + 1}}\right) \\
			                                                                                                       & = \sqrt{n+1} \left(\frac{\sqrt{n}\sqrt{n+1} + 1}{n + 1}\right)                   \\
			                                                                                                       & > \sqrt{n+1} \left(\frac{\sqrt{n}\sqrt{n} + 1}{n + 1}\right)                     \\
			                                                                                                       & = \sqrt{n+1} \left(\frac{n + 1}{n + 1}\right)                                    \\
			                                                                                                       & = \sqrt{n+1} (1)                                                                 \\
			                                                                                                       & = \sqrt{n+1} \tag*{\checkmark}.                                                  \\
		      \end{align*}
		\item Da $\sum_{k=1}^{n} \frac{1}{\sqrt{k}} > \sqrt{n}$ reicht es zu zeigen das $\sqrt{n}$ divergiert und somit transitive auch $\sum_{k=1}^{n} \frac{1}{\sqrt{k}}$.
		      Es ist zu zeigen das für alle $M > 0$ ein $N \in \mathbb{N}$ existiert sodas $\forall n \ge N$ $\mid \sqrt{n} \mod > M$.
		      Setze $N = {(M + 1)}^2$, somit ist $\forall n \ge N$ $\sqrt{n} \ge \sqrt{{(M + 1)}^2} = M+1 > M$.
		      Somit divergiert $\sqrt{n}$ bestimmt gegen $+\infty$ und transitive auch $\sum_{k=1}^{n} \frac{1}{\sqrt{k}}$.
		      \checkmark
	\end{enumerate}
\end{proof}


\end{problem}

% Problem 2 %
\begin{problem}{Cauchy-Folgen}{6 Punkte}
Sei $0 < q < 1$ und $(a_n)$ eine Folge mit $|a_{n+1} - a_n| \leq q |a_n - a_{n-1}|$ für alle $n > n_0$.

% subproblem 2a %
\item Zeigen Sie, dass dann $(a_n)$ eine Cauchy-Folge ist.

% subproblem 2b %
\item Zeigen Sie, dass aus $|a_{n+1} - a_n| < |a_n - a_{n-1}|$ ab einem $n_0$ nicht notwendigerweise folgt, dass $(a_n)$ eine Cauchy-Folge ist.
\end{problem}

% Problem 3 %
\begin{problem}{Eine Folge mit Wurzeln}{6 Punkte}
Seien $c > 1$ und $x_0 = 1$ und $(x_n)$ die folgende rekursiv definierte Folge:
\[
	x_{n+1} := \sqrt{c} \cdot x_n, \quad n \in \mathbb{N}.
\]

Zeigen Sie, dass die Folge $(x_n)$ gegen $c$ konvergiert. Hinweis: Der Beweis kann ähnlich zu dem Beweis von Stz 8.8 geführt werden.
\end{problem}

% Problem 4 %
\begin{problem}{\(\sqrt{2}\) als dyadischer Bruch}{4 Sonderpunkte}
% subproblem 4.1 %
\item Finden Sie die ersten vier Terme $x_0$, $x_1$, $x_2$, $x_3$ unserer Folge $(x_n)$ aus 8.8 (siehe auch 1.6 im Skript) für die Quadratwurzel aus Zwei als exakte dyadische (=2-adische) Brüche. Einen Beweis brauchen Sie nur für $x_0$, $x_1$, $x_2$ zu führen; für $x_3$ genügt das Resultat.

% subproblem 4.2 %
\item Kleine Forschungsaufgabe: Finden Sie im Internet oder berechnen Sie mit einem Computer \(\sqrt{2}\) als dyadischen Bruch mit einer Genauigkeit von 20 binären Stellen hinter dem Punkt.
\end{problem}

\end{document}
