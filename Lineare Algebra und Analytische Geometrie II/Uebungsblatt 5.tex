\documentclass{problemset}

\Lecture{Lineare Algebra und Analytische Geometrie II}
\Problemset{5}
\DoOn{19. Mai 2024}
\author{Michael van Straten}

\begin{document}
\maketitle

\begin{problem}[Simultane Diagonalisierbarkeit]{5 Punkte}
Sei $n, k, \ell \in \mathbb{N}$, $K$ ein Körper und $A, B \in K^{n,n}$. Die
Eigenwerte von $A$ seien mit $L := \{\lambda_1, \ldots, \lambda_k\}$ und die
Eigenwerte von $B$ mit $M := \{\mu_1, \ldots, \mu_\ell\}$ bezeichnet. Zeigen
Sie:
\begin{enumerate}
    \item \label{eigenspace:invariant} Gilt $[A, B] := AB - BA = 0$, so sind die Eigenräume von $A$
          invariant unter $x \mapsto Bx$.
    \item \label{common-eigenvectors} Ist $[A, B] = 0$ und $B$ diagonalisierbar, dann gilt für jeden
          Eigenwert $\lambda \in L$ von $A$, dass
          \[
              \operatorname{Eig}(A, \lambda) = (\operatorname{Eig}(B, \mu_1) \cap \operatorname{Eig}(A, \lambda)) \oplus \cdots \oplus (\operatorname{Eig}(B, \mu_\ell) \cap \operatorname{Eig}(A, \lambda)),
          \]
          wobei die $\mu_1, \ldots, \mu_\ell$ die Eigenwerte von $B$ bezeichnen.
    \item Sind $A$ und $B$ diagonalisierbar, so gilt $[A, B] = 0$ genau dann, wenn es ein
          $T \in \operatorname{GL}_n(K)$ und Diagonalmatrizen $D_A, D_B \in K^{n,n}$
          gibt, welche
          \[
              A = T D_A T^{-1} \quad \text{und} \quad B = T D_B T^{-1}.
          \]
\end{enumerate}

\begin{proof}
    \leavevmode
    \begin{enumerate}
        \item Sei \((\lambda, x)\) ein Eigenpaar von der Matrix \(A\) so gilt
              \begin{equation*}
                  A x = \lambda x \Rightarrow B A x = B \lambda x = \lambda B x.
              \end{equation*}
              Da \(AB = BA\) ist, folgt nun das
              \begin{equation*}
                  A B x = \lambda B x,
              \end{equation*} was jedoch impliziert das \(Bx \in \Eig(A, \lambda)\).

        \item Für die Inklusionsrichtung \(\supseteq\) sei zunächst
              \begin{equation*}
                  v \in \mathop{\bigoplus}_{i=1}^l \Eig(B, \mu_i) \cap \Eig(A, \lambda).
              \end{equation*}
              So folgt das
              \begin{equation*}
                  v = \sum_{i = 1}^{l} v_i \quad \text{mit } v_i \in \Eig(A, \lambda).
              \end{equation*}
              Da \(\Eig(A, \lambda)\) ein Untervektorraum ist, folgt das \(v \in \Eig(A, \lambda)\) ist.

              Für die Inklusionsrichtung \(\subseteq\) sei \(v \in \Eig(A, \lambda)\). Da
              \(B\) diagonalisierbar ist, folgt das
              \begin{equation*}
                  v = \sum_{i = 1}^l v_i \quad \text{mit } v_i \in \Eig(B, \mu_i)
              \end{equation*} ist.

              Da \(v\) im Eigenraum zum wert \(\lambda\) von \(A\) ist, folgt somit das
              \begin{equation}
                  \lambda v = \lambda \sum_{i = 1}^l v_i = A \sum_{i = 1}^l v_i = \sum_{i = 1}^l A v_i \label{linear-combination-eigenvectors}
              \end{equation} ist.
              Aus dem kommutieren von \(A\) und \(B\) folgt, mit \autoref{eigenspace:invariant} das
              \begin{equation*}
                  A v_i \in \Eig(B, \mu_i)
              \end{equation*} ist.

              Stellen wir jetzt \autoref{linear-combination-eigenvectors} nach \(\lambda v_k
              - Av_k\), mit \(1 \le k \le l\), um so erhalten wir
              \begin{equation*}
                  \lambda v_k - Av_k = \sum_{\substack{i = 1 \\ i \neq k}}^{l} A v_k - \lambda v_i.
              \end{equation*}
              Aus der Direktheit der Summe der Eigenräume von \(B\) folgt das
              \begin{equation*}
                  \lambda v_k - Av_k = 0
              \end{equation*}
              was impliziert das \(v_k \in \Eig(A, \lambda)\) ist.

              Die Direktheit der Summe kann aus der Direktheit der Summe über die Eigenräume
              von \(B\) hergeleitet werden.

        \item Aus \autoref{common-eigenvectors} lässt sich mit der Diagonalisierbarkeit von
              \(A\) und \(B\) folgern, dass eine Basis \(B\) aus gemeinsamen Eigenvektoren
              gefunden werden kann.

              Konstruieren wir somit eine Matrix
              \begin{equation*}
                  T = \begin{pmatrix}
                      v_i
                  \end{pmatrix} \quad \text{mit } v_i \text{ Basisvektor von } B,
              \end{equation*} so folgt
              \begin{equation*}
                  T^{-1}AT = T^{-1} \begin{pmatrix}
                      \lambda_j v_i
                  \end{pmatrix} \quad \text{mit } 1 \le j \le k \text{ Eigenwert zu } v_i
              \end{equation*} und
              \begin{equation*}
                  T^{-1} \begin{pmatrix}
                      \lambda_j v_i
                  \end{pmatrix} = \begin{pmatrix}
                      \lambda_j e_i
                  \end{pmatrix} \quad \text{wo } e_i \text{ standart Basisvektor}.
              \end{equation*}

              Bemerken wir das \( \begin{pmatrix} \lambda_j e_i \end{pmatrix} \eqcolon D_A
              \), \(T\) die gesuchten Eigenschaften für die Matrix \(A\) erfüllen.

              Analog kann gezeigt werden das eine Matrix \(D_B\) existiert, für die dasselbe
              gilt.
    \end{enumerate}

\end{proof}
\end{problem}

\begin{problem}[Bidualraum]{4 Punkte}
Sei $K$ ein Körper und $V$ ein $K$-Vektorraum. Für jedes $v \in V$ sei $\Phi_v : V^* \to K$ durch $\Phi_v(f) := f(v)$ definiert. Zeigen Sie:
\begin{enumerate}
    \item Für alle $v \in V$ ist $\Phi_v \in V^{**} := (V^*)^*$ und die Abbildung $\Phi_V
              : V \to V^{**}, \, v \mapsto \Phi_v$ ist linear und injektiv.
\end{enumerate}
Sei nun $\dim(V) \leq \infty$. Zeigen Sie:
\begin{enumerate}[resume]
    \item Der Bidualraum $V^{**}$ ist isomorph zu $V$.
    \item Für einen weiteren endlich dimensionalen Vektorraum $W$ und ein $f \in L(V, W)$
          gilt
          \[
              f = \Phi_W^{-1} \circ f^{**} \circ \Phi_V,
          \]
          wobei $f^{**} := (f^*)^*$ die biduale Abbildung zu $f$ bezeichnet.
\end{enumerate}
\end{problem}

\begin{problem}[Lemma 2.13, Lemma 2.16]{6 Punkte}
Sei $K$ ein Körper und $V, W$ und $X$ Vektorräume über $K$ und $f \in L(V, W)$. Zeigen Sie:
\begin{enumerate}
    \item Für die duale Abbildung $f^*$ von $f$ gilt $f^* \in L(W^*, V^*)$.
    \item Ist $g \in L(W, X)$, dann gilt $(g \circ f)^* \in L(X^*, V^*)$ und $(g \circ
              f)^* = f^* \circ g^*$.
    \item Ist $f$ injektiv/surjektiv, dann ist $f^*$ surjektiv/injektiv.
    \item Ist $f$ bijektiv, dann ist $f^*$ bijektiv und es gilt $(f^*)^{-1} =
              (f^{-1})^*$.
    \item Ist $W = V$ und $f$ nilpotent vom Grade $k \in \mathbb{N}$, so ist $f^*$
          ebenfalls nilpotent vom Grade $k$.
    \item Ist $W = V$, $M \subseteq V^*$ ein unter $f^*$ invarianter Unterraum, dann ist
          $M^0$ invariant unter $f$.
\end{enumerate}
\end{problem}

\end{document}
