\documentclass{problemset}

\Lecture{Lineare Algebra und Analytische Geometrie II}
\Problemset{6}
\DoOn{26. Mai 2024}
\author{Michael van Straten}

\begin{document}
\maketitle

\begin{problem}[Rechenregeln und Definition des Annihilators \textit{[LM21, Aufgabe 11.5, 16.2]}]{7 Punkte}
Sei $K$ ein Körper, $V$ ein endlichdimensionaler $K$-Vektorraum, $U_1, U_2$
Untervektorräume von $V$ und $M$ ein Untervektorraum von $V^*$.
\begin{enumerate}
    \item Zeigen Sie:
          \begin{enumerate}
              \item $(U_1 + U_2)^\circ = U_1^\circ \cap U_2^\circ$.
              \item $(U_1 \cap U_2)^\circ = U_1^\circ + U_2^\circ$.
              \item $U_2^\circ \subseteq U_1^\circ$, falls $U_1 \subseteq U_2$.
          \end{enumerate}
    \item Konstruieren Sie ausgehend von einer Basis von $M$ eine Basis von $M^\circ
              \subseteq V$ und eine Basis von $M^\circ \subseteq (V^*)^*$.
          \begin{hint}
              Hier sind die zwei unterschiedlichen Sichtweisen auf die Definition des
              Annihilators zu einem Unterraum eines Dualraums gemeint.
          \end{hint}
    \item Zeigen Sie, dass $\dim(V) = \dim(M) + \dim(M^\circ)$.
\end{enumerate}

\begin{proof}
    \leavevmode
    \begin{enumerate}
        \item
              \leavevmode
              \begin{enumerate}
                  \item Sei zunächst \(f \in (U_1 + U_2)^\circ\) so folgt das für alle \(v = u_1 +
                        u_2\), mit \(u_1 \in U_1\) und \(u_2 \in U_2\) folgt dass,
                        \begin{equation}
                            f(v) = 0
                        \end{equation} ist.

                        Da die Null in der Vereinigung \(U_1 \cup U_2\) ist, folgt somit das auch für
                        alle \(v \in U_1\) und \(v \in U_2\) gilt das \(f(v) = 0\) ist. So folgt das
                        \(f \in U_1^\circ \cap U_2^\circ\).

                        Sei nun \(f \in U_1^\circ \cap U_2^\circ\) so folgt das \(f(v) = 0\) für alle
                        \(v \in U_1 \cup U_2\). Sei indessen \(v = u_1 + u_2\) mit \(u_1 \in U_1\) und
                        \(u_2 \in U_2\) so folgt das
                        \begin{equation}
                            f(v) = f(u_1 + u_2) = f(u_1) + f(u_2) = 0 + 0 = 0,
                        \end{equation} womit folgt das \(f \in (U_1 + U_2)^\circ\) ist.

                  \item Sei zunächst \(f \in (U_1 \cap U_2)^\circ\) so folgt das für alle \(v \in U_1
                        \cap U_2\) \(f(v) = 0\) ist. Definieren wir jetzt
                        \begin{equation*}
                            g(v) \coloneq \left\{\begin{array}{lr}
                                0    & \text{für } v \in U_1 \\
                                f(v) & \text{sonst}
                            \end{array}\right., \quad
                            h(v) \coloneq \left\{\begin{array}{lr}
                                0    & \text{für } v \in U_2 \cup V \setminus U_1 \\
                                f(v) & \text{sonst}
                            \end{array}\right.
                        \end{equation*}

                        Es ist somit leicht zu sehen das \(g \in U_1^\circ\) und \(h \in U_2^\circ\)
                        ist.

                        So folgt für \(v \in U_1 \cap U_2\) das \(f(v) = 0 = g(v) + h(v)\). Für \(v \in
                        U_1 \setminus U_2\) folgt das \(f(v) = 0 + h(v) = g(v) + h(v)\). Für \(v \in
                        U_2 \setminus U_1\) folgt das \(f(v) = g(v) + 0 = g(v) + h(v)\). Für \(v \in V
                        \setminus U_1 \cup U_2\) folgt \(f(v) = g(v) + 0 = g(v) + h(v)\). So folgt das
                        \(f \in U_1^\circ + U_2^\circ\).

                        Sei nun \(f \coloneq g + h \in U_1^\circ + U_2^\circ\) mit \(g \in U_1^\circ\)
                        und \(h \in U_2^\circ\). Für \(v \in U_1 \cap U_2\) folgt somit das \(g(v) =
                        0\) und \(h(v) = 0\) und somit auch \(f(v) = 0\), was impliziert das \(f \in
                        (U_1 \cap U_2)^\circ\).

                  \item Sei \(f \in U_2^\circ\) so folgt das für alle \(v_2 \in U_2\) \(f(u_2) = 0\) ist.
                        Da \(U_1 \subseteq U_2\) folgt das auch für alle \(u_1 \in U_1\) \(f(u_1) = 0\)
                        ist.
              \end{enumerate}

        \item \label{basis:annihilator} Sei \(B_M^* \coloneq \set{v_1^*, \ldots, v_k^*}\) eine Basis von \(M \subseteq
              V^*\), so lassen sich linear unabhängige Vektoren \(v_{k+1}^*, \ldots, v_n^*\)
              finden die vereinigt mit \(B_M^*\) eine Basis \(B^*\) von \(V^*\) bildet. Zudem
              existiert eine eindeutige Basis \(B\) von \(V\) mit \(B^*\) Dualer-Basis zu
              \(B\).

              Betrachten wir jetzt einen allgemeinen Vektor \(v = \lambda_1 v_1 + \ldots
              \lambda_n v_n \in M^\circ\) so muss für jedes \(f \in M\) gelten das \(f(v) =
              0\). Betrachten wir nun für jeden Basisvektor \(v_i^*\) der Basis \(B_M^*\) die
              Gleichung
              \begin{equation*}
                  v_i^* \sum_{j=1}^n\lambda_j v_j = 0
              \end{equation*}
              so folgt das \(\lambda_j = 0\) für \(1 \le j \le k\) mit \(k = \dim(M)\).

              Somit lässt sich jedes \(v \in M^\circ\) als eine Linearkombination der linear
              unabhängigen Vektoren \(v_{k+1}, \ldots, v_n\) schreiben, was diese zu einer
              Basis von \(M^\circ \subseteq V\) macht.

              Betrachten wir nun zu \(B^*\) die Dualer-Basis \({B^*}^*\) mit \({v^*}^* =
              \lambda_1 {v_1^*}^* + \cdots + \lambda_n {v_n^*}^* \in M^\circ \subseteq
              {V^*}^* \quad {v_i^*}^* \in {B^*}^*\). So folgt unter anderem aus \({v^*}^* \in
              M^\circ\) das
              \begin{equation*}
                  {v^*}^* v_j^* = \sum_{i=1}^{n} \lambda_i {v_i^*}^* v_j^* = 0 \quad \text{für } v_j^* \in B_M^* \Rightarrow \lambda_j = 0 \quad 1 \le j \le \dim(B_M^*).
              \end{equation*}

              Somit lässt sich jedes \({v^*}^* \in M^\circ\) als eine Linearkombination der
              linear unabhängigen Vektoren \[
                  {v_{k+1}^*}^*, \ldots, {v_n^*}^*
              \] schreiben, was diese zu einer Basis von \(M^\circ \subseteq {V^*}^*\) macht.

        \item Aus \autoref{basis:annihilator} wissen wir bereits, dass für \(k = \dim(M)\)
              eine Basis des Annihilatorraums von \(M\) mit \(n - k\) linear unabhängigen
              Basisvektoren existiert, wobei \(n = \dim(V) = \dim(V^*)\) ist. Daraus folgt,
              dass
              \begin{equation*}
                  \dim(V) = n = k + (n - k) = \dim(M) + \dim(M^\circ).
              \end{equation*}
    \end{enumerate}
\end{proof}
\end{problem}

\begin{problem}{4 Punkte}
Sei $(U, W)$ ein duales Raumpaar von endlichdimensionalen Vektorräumen über
demselben Körper. Zeigen Sie, dass $\dim(U) = \dim(W)$.
\end{problem}

\begin{problem}[\textit{[LM21, Aufgaben 16.1, 16.3, 16.4]}]{4 Punkte}
Sei $K$ ein Körper, $V$ ein $K$-Vektorraum, $f \in L(V, V)$, $\lambda \in K$
und $g := f - \lambda \operatorname{Id}_V$. Zeigen Sie:
\begin{enumerate}
    \item Ist $v \in V \setminus \{0\}$ vom Grad $m \in \mathbb{N}$ bezüglich $f$, dann
          gilt
          \[
              K_1(f, v) \subsetneq K_2(f, v) \subsetneq \ldots \subsetneq K_m(f, v) = K_{m+1}(f, v).
          \]
    \item Ein Unterraum von $V$ ist genau dann $f$-invariant, wenn er $g$-invariant ist.
    \item Für $d \in \mathbb{N}$ und $v \in V$ gilt $K_d(f, v) = K_d(g, v)$ und der Grad
          von $v$ bezüglich $f$ ist gleich dem Grad von $v$ bezüglich $g$.
\end{enumerate}
\end{problem}

\end{document}
