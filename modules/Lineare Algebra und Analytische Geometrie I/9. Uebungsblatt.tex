\documentclass{problemset}

\usepackage{multicol}

\Lecture{Lineare Algebra und Analytische Geometrie I}
\Problemset{9}
\DoOn{17. Dezember 2023}
\author{Michael van Straten}

\begin{document}
\maketitle

\begin{problem}[Charakterisierungen direkter Summen]{5 Punkte}
\begin{enumerate}
    \item Sei $V$ ein Vektorraum, $r \in \mathbb{N}$ und $U_1, \ldots, U_r \subseteq V$
          Untervektorräume von $V$. Zeigen Sie, dass folgende Aussagen äquivalent sind:
          \begin{enumerate}
              \item Für $u \in U_1 + \ldots + U_r$ existieren eindeutig bestimmte $u_i \in U_i$, $i
                        \in \{1, \ldots, r\}$ mit $u = u_1 + \ldots + u_r$.
              \item Für $u_1 \in U_1, \ldots, u_r \in U_r$ mit $u_1 + \ldots + u_r = 0$ gilt $u_1 =
                        \ldots = u_r = 0$.
              \item Für $i \in \{1, \ldots, r\}$ gilt $U_i \cap (U_1 + \ldots + U_{i-1} + U_{i+1} +
                        \ldots + U_r) = \{0\}$.
          \end{enumerate}
    \item Finden Sie Gegenbeispiele, die zeigen, dass die Bedingungen aus i) zu keiner
          der folgenden Aussagen äquivalent sind:
          \begin{multicols}{2}
              \begin{enumerate}[resume*]
                  \item $U_1 \cap \ldots \cap U_k = \{0\}$.
                  \item $U_i \cap U_j = \{0\}$, $i, j \in \{1, \ldots, k\}$, $i \neq j$.
              \end{enumerate}
          \end{multicols}
\end{enumerate}
\begin{proof}
    $ $

    \begin{enumerate}
        \item
              \begin{description}
                  \item[\qouted{a) $\Longrightarrow$ b)}]

                        Da $0$ im Untervektorraum $U_1 + \ldots + U_r$ enthalten ist und die
                        Nullvektorenaddition offensichtlich den Nullvektor ergibt, folgt aus der
                        Eindeutigkeit der Darstellung, dass für beliebige Vektoren $u_1 \in U_1,
                            \ldots, u_r \in U_r$ mit $u_1 + \ldots + u_r = 0$ die Gleichheit $u_1 = \ldots
                            = u_r = 0$ gilt. \checkmark

                  \item[\qouted{$\neg$ c) $\Longrightarrow \neg$ b)}]

                        Angenommen, es existiert ein $u_i \in U_i$ mit
                        \[
                            u_i = u_1 + \ldots + u_{i-1} + u_{i+1} + \ldots u_r \neq 0.
                        \]
                        Dies würde implizieren, dass $0 = u_1 + \ldots + u_{i-1} - u_i + u_{i+1} +
                            \ldots + v_r$ mit $u_i \neq 0$ ist.

                        Somit ergibt sich aus $\neg c$ und $\neg b$ die Gültigkeit von $b \Rightarrow
                            c$.

                  \item[\qouted{c) $\Longrightarrow$ a)}]

                        Wählen wir ein $u \in U_1 + \ldots + U_r$ und prüfen, ob es eindeutig bestimmt
                        ist:
                        \begin{align*}
                            u & = u_1 + \ldots + u_r                  \\
                            u & = \tilde{u}_1 + \ldots + \tilde{u}_r.
                        \end{align*}
                        Daraus folgt:
                        \begin{align*}
                            0                             & = u_1 - \tilde{u}_1 + \ldots + u_r - \tilde{u}_r  \\
                            \Rightarrow \tilde{u}_1 - u_1 & = u_2 - \tilde{u}_2 + \ldots + u_r - \tilde{u}_r.
                        \end{align*}
                        Da $U_i \cap (U_1 + \ldots + U_{i-1} + U_{i+1} \ldots + U_r) = \{0\}$, folgt $\tilde{u}_1 - u_1 = 0$.
                        Somit erhalten wir erneut:
                        \[
                            0 = u_2 - \tilde{u}_2 + \ldots + u_r - \tilde{u}_r.
                        \]
                        Diesen Schritt wiederholen wir $r-1$ mal.

                        Daraus folgt $u_i = \tilde{u}_i$, was wiederum bedeutet, dass $u$ eindeutig
                        bestimmt ist. \checkmark
              \end{description}
    \end{enumerate}
\end{proof}
\end{problem}

\begin{problem}[Gemeinsame Komplemente von Untervektorräumen]{8 Punkte}
Sei $V$ ein endlich erzeugter $K$-Vektorraum und seien $U_1$ und $U_2$ Unterräume von $V$ mit $\dim(U_1) = \dim(U_2)$. Zeigen Sie, dass $U_1$ und $U_2$ ein gemeinsames Komplement haben, d.h. es gibt einen Unterraum $W$ von $V$ mit $V = U_1 \oplus W = U_2 \oplus W$.
\end{problem}

\begin{problem}[Schnitte mit linearen Hyperebenen]{7 Punkte}
Sei $n \in \mathbb{N}$ und $V$ ein Vektorraum mit $\dim(V) = n$. Zeigen Sie:
\begin{enumerate}
    \item Sei $H \subseteq V$ ein Unterraum mit $\dim(H) = n - 1$ (ein solcher Unterraum
          heißt lineare Hyperebene von $V$) und $U \subseteq V$ ein weiterer Unterraum,
          so gilt $\dim(U \cap H) \geq \dim(U) - 1$.
    \item Seien $k \in \mathbb{N}$ und $H_1, \ldots, H_k \subseteq V$ Unterräume mit
          $\dim(H_1) = \ldots = \dim(H_k) = n - 1$, so gilt $\dim(H_1 \cap \ldots \cap
              H_k) \geq n - k$.
    \item Sei $k \in \mathbb{N}$, $U \subseteq V$ ein Unterraum mit $\dim(U) = n - k$, so
          existieren Unterräume $H_1, \ldots, H_k \subseteq V$ mit $\dim(H_1) = \ldots =
              \dim(H_k) = n - 1$ und $U = H_1 \cap \ldots \cap H_k$.
\end{enumerate}

\begin{proof}
    $ $
    \begin{enumerate}
        \item Aus der Vorlesung ist uns bekannt, dass $\dim(U+H) = \dim(H) + \dim(U) - \dim(U
                  \cap H)$ gilt. Dies lässt sich auch als \[
                  \dim(U \cap H) + \dim(U + H) - n = \dim(U) - 1
              \] formulieren.

              Da $\max(\dim(U + H)) = n$, wobei $\dim(V) = n$, ergibt sich $\max(\dim(U + H)
                  - n) = 0$. Daher folgt $\dim(U \cap H) \ge \dim(U \cap H) + \dim(U + H) - n$,
              und daraus ergibt sich $\dim(U \cap H) \geq \dim(U) - 1$.

        \item Angenommen, die Induktionsbehauptung gilt für $k-1$ Hyperebenen.

              Sei $L = \bigcap_{i=1}^{k-1} H_i$. Dann folgt
              \begin{align*}
                  n \ge \dim(L + H_k) & = \dim L + \dim H_k \annotated{Induktions Hypothese}{\ge} (n - (k-1)) + (n-1) - \dim(L \cap H_k) \\
                                      & \Rightarrow \dim(L \cap H_k) \ge n - k + 1 + n - 1 - n = n - k.
              \end{align*}

    \end{enumerate}
\end{proof}
\end{problem}

\end{document}
