\documentclass{exam}

\usepackage{amsmath}

\title{Lineare Algebra und Analytische Geometrie I - 1. Übungsblatt}
\author{Michael van Straten}

\AtBeginEnvironment{align}{\setcounter{equation}{0}}

\begin{document}
\maketitle
\section*{Aufgabe 1: Mengenoperationen und Gleichheit von Mengen}
Gegeben seien die Mengen $M$, $N$ und $L$. Zeigen Sie:
\begin{enumerate}
    \item[i)] $M \cap (N \cap L) = (M \cap N) \cap L$,
    \item[ii)] $M \cap (N \cup L) = (M \cap N) \cup (M \cap L)$,
    \item[iii)] $M \setminus (N \cap L) = (M \setminus N) \cup (M \setminus L)$.
\end{enumerate}
\textbf{Lösungen:}
\begin{enumerate}
    \item[i)]
          \begin{align*}
              M \cap (N \cap L)                     & \Leftrightarrow \tag{Umformung zu Aussagenlogik}                \\
              x \in M \land (x \in N \land x \in L) & \Leftrightarrow \tag{Assoziativgesetze [LM21, Aufgabe 2.1 (a)]} \\
              (x \in M \land x \in N) \land x \in L & \Leftrightarrow \tag{Zurückformung zu Mengen}                   \\
              (M \cap N) \cap L
          \end{align*}
    \item[ii)]
          \begin{align*}
              M \cap (N \cup L)                                    & \Leftrightarrow \tag{Umformung zu Aussagenlogik}                 \\
              x \in M \land (x \in N \lor x \in L)                 & \Leftrightarrow \tag{Distributivgesetze [LM21, Aufgabe 2.1 (c)]} \\
              (x \in M \land x \in N) \lor (x \in M \land x \in L) & \Leftrightarrow \tag{Zurückformung zu Mengen}                    \\
              (M \cap N) \cup (M \cap L)
          \end{align*}
    \item[iii)]
          \begin{align*}
              M \setminus (N \cap L)                                     & \Leftrightarrow \tag{Umformung zu Aussagenlogik}                 \\
              x \in M \land \neg (x \in N \land x \in L)                 & \Leftrightarrow \tag{\textit{De Morganschen Gesetze}}            \\
              x \in M \land (x \notin N \lor x \notin L)                 & \Leftrightarrow \tag{Distributivgesetze [LM21, Aufgabe 2.1 (c)]} \\
              (x \in M \land x \notin N) \lor (x \in M \land x \notin L) & \Leftrightarrow \tag{Zurückformung zu Mengen}                    \\
              (M \setminus N) \cup (M \setminus L)
          \end{align*}
\end{enumerate}
\section*{Aufgabe 2: Äquivalenz von Aussage}
Gegeben sein Mengen $A$ und $B$ Zeigen Sie die Äquivalenz der folgenden Aussagen:
\begin{enumerate}
    \item[i)] $ A \subseteq B $,
    \item[ii)] $ A \cup B = B $,
    \item[iii)] $ A \cap B = A $.
\end{enumerate}
\textbf{Lösungen:}
\begin{enumerate}
    \item $ A \subseteq B \Leftrightarrow A \cup B = B $ \\
          Gehen wir davon aus, dass $A$ eine Teilmenge von $B$ ist.
          Somit gibt es für jedes Element in $A$ ein Element in $B$.
          Schauen wir uns nun die Vereinigung von $A$ und $B$ an.
          Da jedes Element von $A$ ebenfalls in $B$ ist, muss die Vereinigung von $A$ und $B$ gleich $B$ sein.
          Wenn jedoch die Vereinigung von $A$ und $B$ gleich $B$ ist, muss $A$ eine Teilmenge von $B$ sein.
    \item $ A \cup B = B \Leftrightarrow A \cap B = A $ \\
          Wenn $ A \cup B $ gleich der Menge $B$ ist, muss nach 1. $A$ eine Teilmenge von $B$ sein.
          Da $A$ eine Teilmenge von $B$ ist, muss jedes Element in $A$ auch in $B$ enthalten sein.
          Somit ist der Durchschnitt von $A$ und $B$ gleich $A$ nach Definition des Durchschnitts zweier Mengen.
          Ist jedoch der Durchschnitt von $A$ und $B$ gleich $A$, so muss $A$ eine Teilmenge von $B$ sein,
          da für jedes Element in $A$ es ein Element in $B$ nach Definition der Teilmenge geben muss.
\end{enumerate}
\end{document}

