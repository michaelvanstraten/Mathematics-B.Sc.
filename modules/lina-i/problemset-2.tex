\documentclass{exam}

\usepackage{amssymb, amsmath, amsthm}
\usepackage[german]{babel}
\usepackage{hyperref}
\usepackage{tabularx}

\title{Lineare Algebra und analytische Geometrie I - 2. Übungsblatt}
\author{Michael van Straten}
\date{\today}

\newtheorem{definition}{Definition}

\begin{document}
\maketitle

\section*{Aufgabe 1:}
\begin{enumerate}
    \item [i)] Es sei $A := \{\emptyset, \pi\}$, $B := \mathcal{P}(A)$ und $C := \mathcal{P}(B)$. Entscheiden Sie den Wahrheitsgehalt von jeder der folgenden Aussagen. \newline
          \renewcommand{\arraystretch}{1.5}
          \begin{tabularx}{\textwidth}{ X X X }
              a) $\{\pi\} \in A$ (falsch)       & d) $\{\pi\} \in B$ (wahr)           & g) $\{\{ \pi \}\} \in C$ (wahr)                     \\
              b) $\{\pi\} \subseteq A$ (wahr)   & e) $\emptyset \subseteq B$ (wahr)   & h) $\{A\} \in C$ (wahr)                             \\
              c) $A \subseteq \{\pi\}$ (falsch) & f) $\{\{\pi\}\} \subseteq B$ (wahr) & i) $\{\emptyset, \{\emptyset\}\} = B \cap C$ (wahr)
          \end{tabularx}

    \item [ii)] Sein $M$ und $N$ Mengen. Zeigen Sie, dass \begin{enumerate}
              \item [a)] $\mathcal{P}(M) \cap \mathcal{P}(N) \subseteq \mathcal{P}(M \cap N)$,
              \item [b)] $\mathcal{P}(M) \cup \mathcal{P}(N) \subseteq \mathcal{P}(M \cup N)$.
          \end{enumerate}
          \begin{definition}
              \begin{displaymath}
                  \mathcal{P}(M) := \{X\,|\,X \subseteq M\}
              \end{displaymath}
          \end{definition}

          \begin{proof}
              $\mathcal{P}(M) \cap \mathcal{P}(N) \subseteq \mathcal{P}(M \cap N)$
              \newline
              Lass $X$ eine beliebige Menge sein so das $X \subseteq M$ und $X \subseteq N$.
              Laut Definition der Potenzmenge ist somit $X$ in der Menge $\mathcal{P}(M)$ sowie $\mathcal{P}(N)$.
              Da $X$ eine Teilmenge von $M$ sowie $N$ ist, ist $X$ auch eine Teilmenge von dem Schnitt $M \cap N$.
              Daher $X$ somit eine Teilmenge von $M \cap N$ ist, ist $X$, laut Definition der Potenzmenge, ebenfalls in der Menge $\mathcal{P}(M \cap N)$ enthalten.
              Da die Menge $X$ beliebig gewählt wurde, ist somit $\mathcal{P}(M) \cap \mathcal{P}(N)$ eine Teilmenge von $\mathcal{P}(M \cap N)$, was zu zeigen war.
          \end{proof}
          \begin{proof}
              $\mathcal{P}(M) \cup \mathcal{P}(N) \subseteq \mathcal{P}(M \cup N)$
              \newline
              Lass $X$ eine beliebige Menge sein so das $X \subseteq M$ oder $X \subseteq N$.
              Laut Definition der Potenzmenge ist somit $X$ in der Menge $\mathcal{P}(M)$ oder in $\mathcal{P}(N)$.
              Da $X$ eine Teilmenge von $M$ oder $N$ ist, ist $X$ auch eine Teilmenge von der Vereinigung $M \cup N$.
              Daher $X$ somit eine Teilmenge von $M \cup N$ ist, ist $X$, laut Definition der Potenzmenge, ebenfalls in der Menge $\mathcal{P}(M \cup N)$ enthalten.
              Da $X$ beliebig gewählt wurde, ist somit $\mathcal{P}(M) \cup \mathcal{P}(N)$ eine Teilmenge von $\mathcal{P}(M \cup N)$, was zu zeigen war.
          \end{proof}

          In dem fall $\mathcal{P}(M) \cap \mathcal{P}(N) \subseteq
          \mathcal{P}(M \cap N)$, gilt sogar $\mathcal{P}(M) \cap
          \mathcal{P}(N) = \mathcal{P}(M \cap N)$.
\end{enumerate}

\pagebreak

\section*{Aufgabe 2:}
Sei $M$ eine Menge und $R$ eine reflexive, symmetrische und antisymmetrische Relation auf $M$. Zeigen Sie, dass \[
    R = Id_M := \{(x,y) \in M \times M\,|\,x=y \}.
\]
\begin{proof} $R$ ist reflexive, symmetrische und antisymmetrische:
    \begin{enumerate}
        \item [a)] \underline{reflexive}: Für alle $x \in M$ gilt $x \sim x$: \newline
              Für jedes $x \in M$ gilt $x \sim x$ da $x$ immer gleich $x$ ist.
              \newline\newline
              Somit ist $R$ reflexive.
        \item [b)] \underline{symmetrische}: Für alle $x, y \in M$ gilt $x \sim y \Leftrightarrow y \sim x$: \newline
              Für jedes $x,y \in M$ existiert ein geordnetes paar $(x,y)$ in der Menge $M \times M$ sowie ein paar $(y, x)$ für welches gilt $x = y$ und $y = x$.
              Und somit sind sowohl die Relation $x \sim y$ und $y \sim x$ in $R$. Daraus folgt, $R$ ist symmetrisch.
        \item [c)] \underline{antisymmetrische}: Für alle $x, y \in M$ gilt $x \sim y$ und $y \sim x$ $\Rightarrow x = y$: \newline
              Aus \textit{b} wissen wir bereits das zwei solche paar $(x,y)$ und $(y, x)$ in $R$ existiert und das $x = y$ ist.
              Somit ist $R$ antisymmetrische.
    \end{enumerate}
\end{proof}

\pagebreak

\section*{Aufgabe 3:}
Sei $M$ eine Menge und $\le_{M}$ eine partielle Ordnung auf $M$. Zeigen Sie, dass \[
    \le_{M^2} := \left\{((x_1,x_2), (y_1,y_2))\in M^2 \times M^2\ \begin{array}{|ccc}
        x_1 \neq y_1 \text{ und } x_1 \le_{M} y_1 \\
        \text{ oder }                             \\
        x_1 = y_1 \text{ und } x_2 \le_{M} y_2
    \end{array} \right\}
\] eine partielle Ordnung auf $M^2$ definiert.
\begin{proof} $\le_{M^2}$ definiert eine partielle Ordnung \newline
    Um zu zeigen das $\le_{M^2}$ eine partielle auf $M^2$ definiert müssen die folgenden drei eigenschaften erfüllt sein.
    \begin{enumerate}
        \item [a)] \underline{reflexive}: Für alle $x \in M^2$ gilt $x \sim x$. \newline
              Es ist zu beweisen, dass die Relation $(x_1, x_2) \sim (x_1, x_2)$ existiert.
              Laut Voraussetzung von $\le_{M^2}$ können wir denn fall $x_1 \neq y_1$ ausschließen da $x_1$ immer gleich $x_1$.
              Es muss also nur die Relation $x_2 \le_M x_2$ existierten.
              Da $\le_M$ eine partielle Ordnung definiert muss allerdings für alle $x \in M$ $x \le_M x$ gelten.
              Damit existiert die Relation $(x_1, x_2) \le_{M^2} (x_1, x_2)$ für alle $(x_1,x_2) \in M^2$.
              \newline\newline
              Somit ist $\le_{M^2}$ reflexive.
        \item [b)] \underline{antisymmetrische}: Für alle $x, y \in M^2$ gilt, wenn $x \sim y$ und $y \sim x$ $\Rightarrow x = y$ \newline
        \item [c)] \underline{transitive}: Für alle $x,y,z \in Q$ gilt, wenn $x \sim y$ und $y \sim z$ $\Rightarrow x \sim z$ \newline
    \end{enumerate}
\end{proof}

\pagebreak

\section*{Aufgabe 4:}
Wir betrachten die folgende Relation $\sim$ auf $Q := \mathbb{Z} \times \mathbb{Z}\setminus\{0\}$ \[
    \sim := \{((n_1,d_1),(n_2,d_2)) \in Q \times Q\ |\ n_1d_2 = n_2d_1\}.
\]
Beweise oder widerlegen Sie die folgenden Aussagen:
\begin{enumerate}
    \item [i)]
          \renewcommand{\arraystretch}{1.5}
          \begin{tabularx}{\textwidth}{ X X X }
              a) $(3,6) \sim (9,18)$ & c) $(3,6) \sim (-3, 6)$   & e) $(3, 6) \sim (-3, -6)$ \\
              b) $(0,3) \sim (0,-2)$ & d) $(-4,12) \sim (1, -3)$ & f) $(1, 10) \sim (10, 1)$
          \end{tabularx}
    \item [ii)] Für alle $(n,d) \in Q$ und $k \in \mathbb{Z}\setminus\{0\}$ gilt $(n,d) \sim (kn,kd)$
    \item [iii)] $\sim$ ist eine partielle Ordnung auf $Q$
    \item [iv)] $\sim$ ist eine Äquivalenzrelation auf $Q$
\end{enumerate}
\begin{proof} \ \newline
    \begin{enumerate}
        \item [i)]
              \renewcommand{\arraystretch}{1.5}
              \begin{tabularx}{\textwidth}{ X X X }
                  a) $3*18 = 6*9 = 54$ (wahr)  & c) $3*6 = 6*(-3)$ (falsch)  & e) $3*(-6) = 6 * (-3)$ (wahr) \\
                  b) $0*(-2) = 3*0 = 0$ (wahr) & d) $(-4)*(-3)= 12*1$ (wahr) & f) $1*10 = 10*1$ (wahr)
              \end{tabularx}
        \item [ii)] Für ein $(n,d) \in Q$ und $k \in \mathbb{Z}\setminus\{0\}$ gilt $(n,d) \not\sim (kn,kd)$, daraus folgt
              \begin{align}
                  nkd & \neq dkn                            \\
                  nd  & \neq dn \tag{da $k \neq 0$}         \\
                  nd  & \neq nd \tag{\textbf{Widerspruch!}}
              \end{align}
              somit gilt für alle $(n,d) \in Q$ und $k \in \mathbb{Z}\setminus\{0\}$ $(n,d) \sim (kn,kd)$.
        \item [iii)] $\sim$ ist eine partielle Ordnung auf $Q$: \newline
              Damit $\sim$ eine partielle auf Q definiert muss es die folgenden drei Eigenschaften erfüllen.
              \begin{enumerate}
                  \item [a)] \underline{reflexive}: Für alle $x \in Q$ gilt $x \sim x$: \newline
                        Für alle $(n_1,d_1) \in Q$ gilt $(n_1, d_1) \sim (n_1, d_1)$ da \begin{align}
                            n_1*d_1 = d_1*n_1 \\
                            n_1*d_1 = n_1*d_1
                        \end{align} somit ist $\sim$ reflexive. \checkmark
                  \item [b)] \underline{antisymmetrische}: Für alle $x, y \in Q$ gilt, wenn $x \sim y$ und $y \sim x$ $\Rightarrow x = y$: \newline
                        Aus \underline{4. i) a)} wissen wir das, dass paar $((3,6),(9,18))$ in $\sim$ enthalten ist, wenn nun $((9,18),(3,6))$ Teil von $\sim$ ist muss laut Voraussetzung $(3,6) = (9,18)$ sein.
                        $9*6 = 18 * 3 = 54$, somit ist $(9,18) \sim (3,6)$.
                        Da $(3,6) \neq (9,18)$ ist $\sim$ somit nicht antisymmetrische.
                  \item [c)] \underline{transitive}: Für alle $x,y,z \in Q$ gilt, wenn $x \sim y$ und $y \sim z$ dann $x \sim z$ \newline
                        Lass $(n_1,d_1), (n_2,d_2), (n_3, d_3) \in Q$ für die gilt $n_1*d_2 = d_1*n_2$ und $n_2*d_3=d_2*n_3$.
                        Für Transitivität ist zu beweisen das $n_1*d_3=d_1*n_3$, starten wir mit $(n_1, d_1) \sim (n_2, d_2)$ so gilt: \[
                            n_1*d_2 = d_1*n_2.
                        \]
                        Da $n_2*d_3=d_2*n_3$ können wir beiden Seiten der zwei
                        Gleichungen mit ein an der multiplizieren und erhalten \[
                            n_1*n_2*d_2*d_3 = d_1*d_2*n_2*n_3.
                        \]
                        Beide Seiten beinhalten den Term $n_2*d_2$ und da die
                        Ausgangsmenge die $0$ nicht enthielt können diese
                        weggekürzten. Wir erhalten den Term \[
                            n_1*d_3 = d_1*n_3
                        \] was zu zeigen war. \checkmark
              \end{enumerate}
              Die Relation erfüllt somit nicht die Eigenschaften einer partiellen Ordnung.
        \item [iv)] $\sim$ ist eine Äquivalenzrelation auf $Q$ \newline
              Damit $\sim$ eine Äquivalenzrelation auf $Q$ definiert muss sie die folgenden drei Eigenschaften haben.
              \begin{enumerate}
                  \item [a)] \underline{reflexive}: Für alle $x \in Q$ gilt $x \sim x$ \newline
                        Bereits bewiesen in \underline{4. iii) a)}
                  \item [b)] \underline{symmetrisch}: Für alle $x,y \in Q$ gilt, wenn $x \sim y$ dann $y \sim x$: \newline
                        Angenommen wir haben zwei paare $(n_1,d_1),(n_2,d_2) \in Q$ für die gilt $n_1*d_2=d_1*n_2$.
                        Somit ist zu beweisen das dann auch \[
                            n_2*d_1 = d_2*n_1
                        \] gilt. \newline\newline Laut assoziative gesetzt ist
                           $a*b = b*a$ und somit dürfen wir auf beiden Seiten
                           $d_k$ mit $n_k$ vertauschen.
                        \[
                            d_2*n_1=n_2*d_1.
                        \]
                        Da $a = b$ dasselbe ist, wie $b = a$ dürfen wir beide
                        Seiten vertauschen \[
                            n_2*d_1 = d_2*n_1.
                        \]
                        Somit existiert die Relation $(n_2,d_2) \sim (n_1,
                        d_1)$, was zu zeigen war.
                  \item [c)] \underline{transitive}: Für alle $x,y,z \in Q$ gilt, wenn $x \sim y$ und $y \sim z$ dann $x \sim z$ \newline
                        Bereits bewiesen in \underline{4. iii) c)}
              \end{enumerate}
              Da $\sim$ reflexive, symmetrisch und transitive ist, ist sie eine Äquivalenzrelation.
    \end{enumerate}
\end{proof}

\end{document}
