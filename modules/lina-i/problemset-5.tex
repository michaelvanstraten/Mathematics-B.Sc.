\documentclass{problemset}

\Lecture{Lineare Algebra und Analytische Geometrie I}
\Problemset{5}
\DoOn{19. November 2023 23:59 Uhr}
\author{Michael van Straten}

\begin{document}
\maketitle

\begin{problem}[Untergruppenkriterien; Kern als Untergruppe]{5 Punkte}
Sei $(G, \cdot)$ eine Gruppe und $U \subseteq G$ nicht leer. Ist auch $(U, \cdot)$ eine Gruppe, so heißt sie Untergruppe von $(G, \cdot)$.

\begin{enumerate}
    \item Zeigen Sie die Äquivalenz der folgenden Aussagen:
          \begin{enumerate}[label=\alph*)]
              \item $(U, \cdot)$ ist eine Untergruppe von $(G, \cdot)$,
              \item Für alle $a, b \in U$ gilt $ab^{-1} \in U$,
              \item Die Relation $\sim_U := \{(a, b) \in G \times G \mid ab^{-1} \in U\}$ ist eine
                    Äquivalenzrelation auf $G$.
          \end{enumerate}
    \item Sei $(H, +)$ eine weitere Gruppe und $f: G \to H$ ein Gruppenhomomorphismus.
          Zeigen Sie, dass $(\ker(f), \cdot)$ eine Untergruppe von $(G, \cdot)$ ist.
\end{enumerate}

\begin{proof}
    \begin{enumerate}
        \item \textit{Äquivalenz von Aussagen:}
              \begin{enumerate}[label=\alph*)]
                  \item \textit{a)} $\Longrightarrow$ \textit{b)}: \\
                        $(U, \cdot)$ ist eine Untergruppe von $(G, \cdot)$, wenn sie in Hinsicht auf das inverse Element und die Operationen $\cdot$ geschlossen ist.
                        Es folgt
                        \begin{align}
                            a, b \in U & \Longrightarrow a \cdot b \in U \quad \text{(Geschlossenheit)}, \\
                            b \in U    & \Longrightarrow b^{-1} \in U \quad \text{(Inverse)}.
                        \end{align}
                        Mit $a \in U$ und $b \in U$ aus (2) folgt, dass auch $b^{-1} \in U$ ist und aus (1) folgt, dass somit auch $a \cdot b^{-1} \in U$. \checkmark
                  \item \textit{a)} $\Longleftarrow$ \textit{b)}: \\
                        Angenommen, $\forall a,b \in U$ gilt $ab^{-1} \in U$. \\
                        \\
                        Somit muss auch das neutrale Element in $U$ sein, da wenn $a=b$, dann $aa^{-1} \in U$. \\
                        \\
                        Setze $a=e$ und $b=a$. Somit muss auch das inverse Element zu $a$ in $U$ sein, da $ea^{-1} = a^{-1} \in U$.
                        Da jedoch auch inverse Elemente in $U$ sind, muss $U$ somit auch abgeschlossen sein, da $a,b^{-1} \in U \Longrightarrow a{(b^{-1})}^{-1} = ab \in U$. \\
                        \\
                        Somit sind alle Bedingungen für eine Untergruppe erfüllt und die beiden Aussagen sind äquivalent. \checkmark
                  \item \textit{b)} $\Longleftrightarrow$ \textit{c)}: \\
                        \textbf{Reflexivität}: $(a, a) \in \sim_U$ \\
                        $\forall (a,a) \in G \times G$ gilt durch das Kriterium, dass $aa^{-1} \in U$ (wie in \textit{ii)} bewiesen), da das neutrale Element in $U$ ist. \checkmark \\
                        \textbf{Symmetrie}: $(a, b) \in \sim_U \Longrightarrow (b,a) \in \sim_U$ \\
                        Hier folgt wiederum direkt aus dem Kriterium, dass wenn $ab^{-1}$ in $U$ ist, auch $ba^{-1}$ in $U$ ist (wie in \textit{ii)} bewiesen). \checkmark \\
                        \textbf{Transitivität}: $(a, b) \in \sim_U \land (b,c) \in \sim_U \Longrightarrow (a,c) \in \sim_U$ \\
                        Angenommen, $(a, b) \in \sim_U$ und $(b, c) \in \sim_U$. Das bedeutet, dass $ab^{-1} \in U$ und $bc^{-1} \in U$. Wir wollen zeigen, dass dann auch $ac^{-1} \in U$.
              \end{enumerate}
        \item \textit{$\ker(f)$ ist eine Untergruppe:}
              Wie bereits in \textit{i.b} bewiesen, müssen wir lediglich zeigen, dass $\forall a,b \in \ker(f) \Rightarrow ab^{-1} \in \ker(f)$, um zu zeigen, dass $\ker(f)$ eine Untergruppe von $G$ ist.

              Nehmen wir zwei Elemente aus $\ker(f)$, $a$ und $b$. Somit ist laut Definition
              von Homomorphismen $f(a \cdot b) = f(a) + f(b)$. Somit folgt
              \begin{align}
                  f(b^{-1})         & = f(b)^{-1} \quad \text{(Homomorphismen bilden inverse auf inverse ab)}                     \\
                  f(b \cdot b^{-1}) & = f(b) + f(b)^{-1}                                                                          \\
                  f(e_G)            & = e_H + f(b)^{-1}                                                                           \\
                  f(e_G)            & = e_H + e_H \quad \text{(Homomorphismen bilden neutrale Elemente auf neutrale Elemente ab)}
              \end{align}
              Somit muss $b^{-1}$ in $\ker(f)$ sein, da $f(b^{-1}) = e_H$. Folglich folgt
              \begin{align}
                  f(a \cdot b^{-1}) = f(a) + f(b^{-1}) = e_H + e_H = e_H
              \end{align}
              und somit ist auch $ab^{-1} \in \ker(f)$, woraus folgt, dass $\ker(f)$ eine Untergruppe von $G$ ist.
    \end{enumerate}
\end{proof}
\end{problem}

\begin{problem}[Der erste Isomorphiesatz am Beispiel]{4 Punkte}
Sei $x \in \mathbb{R}$ und $n \in \mathbb{N}$.

Wir betrachten die Abbildung $d: (P_{\le n }[x], +) \to (P_{\le n-1}[x], +)$
zwischen den additiven Gruppen der Polynome vom Grad kleiner gleich $n$ und
Grad kleiner gleich $n - 1$, die für jedes $p(x) = a_0 + a_1x + a_2x^2 + \ldots
    + a_nx^n \in P_{\le n}[x]$, $a_j \in \mathbb{R}$, $j \in \{1, \ldots, n\}$
gegeben ist durch \[
    d(p(x)) = a_1 + 2a_2x + \cdots + na_nx^{n-1} \in P_{\le n-1}[x].
\]

Sei weiter $P'_{\le n}[x] := \{p(x) \in P_{\le n}[x] \mid p(x) = a_1x + \ldots
    + a_nx^n, a_j \in \mathbb{R}, j \in \{1, \ldots, n\}\}$.

\begin{enumerate}
    \item Zeigen Sie, dass $d$ ein surjektiver Gruppenhomomorphismus ist.
    \item Berechnen Sie $\ker(d)$ und zeigen Sie, dass $\ker(d) \cap P'_{\le n}[x] =
              \{0\}$, wobei $0$ das neutrale Element in $(P_{\le n}[x], +)$ bezeichnet.
    \item Zeigen Sie, dass $(P'_{\le n}[x], +)$ eine Untergruppe von $(P_n[x], +)$ ist.
    \item Zeigen Sie, dass $D := d|_{P'_n[x]}$ bijektiv ist und geben Sie die inverse
          Abbildung $I := D^{-1}: P_{n-1}[x] \to P'_n[x]$ von $D$ an.
\end{enumerate}

\begin{proof}
    \begin{enumerate}
        \item \textit{$d$ ist ein surjektiver Gruppenhomomorphismus} \\
              \textbf{$d$ ist ein Gruppenhomomorphismus}: \\
              Zu zeigen ist, dass $\forall p(x),q(x) \in P_{\le n}[x]$ gilt $d(p(x) + q(x)) = d(p(x)) + d(q(x))$. \\

              Notiere das:
              \begin{align}
                  p(x) & = a_0 + a_1x^1 + a_2x^2 + \cdots + a_nx^n, \\
                  q(x) & = b_0 + b_1x^1 + b_2x^2 + \cdots + b_nx^n.
              \end{align}

              Somit ist $p(x) + q(x) =$ \[
                  (a_0+b_0) + (a_1+b_1)x^1 + (a_2 + b_2)x^2 + \cdots + (a_n+b_n)x^n,
              \] und $d(p(x) + q(x)) =$ \[
                  (a_1+b_1) + 2(a_2 + b_2)x + \cdots + n(a_n+b_n)x^{n-1}.
              \]

              Betrachten wir nun die Abbildung von $d(p(x)) + d(q(x))$:
              \begin{align}
                   & d(a_0 + a_1x^1 + a_2x^2 + \cdots + a_nx^n) + d(b_0 + b_1x^1 + b_2x^2 + \cdots + b_nx^n) \\
                   & = a_1 + 2a_2x + \cdots + na_nx^{n-1} + b_1 + 2b_2x + \cdots + nb_nx^{n-1}               \\
                   & = a_1 + b_1 + 2a_2x + 2b_2x + \cdots + na_nx^{n-1} + nb_nx^{n-1}                        \\
                   & = (a_1 + b_1) + 2(a_2 + b_2)x + \cdots + n(a_n+b_n)x^{n-1}                              \\
                   & = d(p(x) + q(x)).
              \end{align}

              Somit ist gezeigt, dass $d(p(x))$ ein Gruppenhomomorphismus ist. \checkmark
              \textbf{$d$ ist surjektiv}: \\ Zu zeigen ist, dass $\forall p_{\le n-1}(x) \in
                  P_{\le n-1}[x]$ ein $p(x) \in P_{\le n}[x]$ existiert, sodass $d(p(x)) = p_{\le
                          n-1}(x)$. \\\\ Ein solches $p(x)$ zu finden ist gleichbedeutend damit, für
              jedes Polynom vom Grad $n-1$ ein Polynom des Grades $n$ in $P_{\le n}[x]$ zu
              finden, da $d(p(x))$ von Polynomen Grad $n$ zu Polynomen Grad $n-1$ abbildet.
              Da $P_{\le n}[x]$ jedoch die Menge der Polynome vom Grad $n$ ist, lässt sich
              für jedes Polynom aus der Menge $P_{\le n - 1}$ ein Polynom des Grades $n+1$ in
              $P_{\le n}[x]$ finden, sodass $d(p(x)) = p_{\le n -1}(x)$. Somit ist $d$
              surjektiv. \checkmark
        \item \textit{Der $\ker(d)$ und seine Schnitte} \\
              Es ist leicht zu sehen, dass der $\ker(d)$ gleich der Menge
              \[
                  \{p(x) \in P_{\le n} \mid a_0 \in \mathbb{R} \land a_j = 0, j \in \{1, \ldots, n\}\}
              \]
              ist, da $\forall p(x) \in \ker(d)$ gilt
              \[
                  d(p(x)) = 0 + 0 + \cdots + 0 = 0 = \text{neutrales Element aus } P_{\le n-1}.
              \]
              Somit ist jedoch auch leicht zu sehen, dass $\ker(d) \cap P'_{\le n}[x] =
                  \{0\}$, da
              \begin{align}
                  \ker(d) \cap P'_{\le n}[x] & = \{p(x) \in P_{\le n} \mid (a_0 \in \mathbb{R} \land a_j = 0) \land (a_0 = 0 \land a_j \in \mathbb{R}), j \in \{1, \ldots, n\}\} \\
                                             & = \{p(x) \in P_{\le n} \mid (a_0 = 0 \land a_j = 0, j \in \{1, \ldots, n\}\}                                                      \\
                                             & = \{p(x) \in P_{\le n} \mid (a_j = 0, j \in \{0, \ldots, n\}\}                                                                    \\
                                             & = \{0\}.
              \end{align}
              \checkmark
        \item \textit{Die Untergruppe} \\
              Um zu beweisen, dass $P'_{\le n}[x]$ eine Untergruppe von $P_{\le n}[x]$ ist,
              müssen wir die Geschlossenheit von $P'_{\le n}[x]$ sowie seine Geschlossenheit
              in Hinsicht auf Inverse beweisen. \\ \textbf{Geschlossenheit}: \\ Es ist zu
              zeigen, dass $\forall p(x),q(x) \in P'_{\le n}[x]$ gilt $p(x) + q(x) \in
                  P'_{\le n}[x]$

              Lass soeben wieder $p(x), q(x) =$ \[
                  p(x) = a^\prime_1 x + \cdots + a^\prime_n x^n, \\
                  q(x) = b^\prime_1 x + \cdots + b^\prime_n x^n.
              \]
              Somit gilt $p(x) + q(x) =$ \[
                  a^\prime_1 x + \cdots + a^\prime_nx^n + b^\prime_1x + \cdots + b^\prime_nx^n =
                  \overbrace{(a^\prime_1+b^\prime_1)}^{\in \mathbb{R}}x + \cdots + \overbrace{(a^\prime_n+b^\prime_n)}^{\in \mathbb{R}}x^n,
              \] was jedoch wieder in $P'_{\le n}[x]$ liegt. \checkmark \\
              \textbf{Geschlossenheit unter Inversen}: \\ Da $\mathbb{R}$ ein Körper ist,
              existiert zu jedem Faktor $a_j$ ein additives inverses Element in $\mathbb{R}$,
              sodass $a_j + a_j^{-1} = 0$. Daraus folgt, dass es für jedes $p(x) \in P'_{\le
                  n}[x]$ ein $p(x)^{-1} \in P'_{\le n}[x]$ existiert, sodass $p(x) + p(x)^{-1} =
                  0 =$ neutrales Element in $P'_{\le n}[x]$ und in $P_{\le n}[x]$. \checkmark
        \item \textit{Die Bijektivität von $D$ und die Inverse Abbildung $I$} \\
              Um zu zeigen, dass $D := d|_{P'_{\le n}[x]}$ bijektiv ist, betrachten wir die Injektivität und die Surjektivität von $D$.

              \textbf{Injektivität von $D$}: \\
              Angenommen, es gibt $p_1(x), p_2(x) \in P'_{\le n}[x]$ mit $D(p_1(x)) = D(p_2(x))$. Das bedeutet, dass $d(p_1(x)) = d(p_2(x))$.

              Da $d$ surjektiv ist (wie im ersten Teil gezeigt), existieren Polynome $q_1(x),
                  q_2(x) \in P_{\le n}[x]$ mit $d(q_1(x)) = p_1(x)$ und $d(q_2(x)) = p_2(x)$.

              Da $d(p_1(x)) = d(p_2(x))$, folgt $d(q_1(x)) = d(q_2(x))$. Durch die
              Injektivität von $d$ erhalten wir $q_1(x) = q_2(x)$.

              Somit ist $p_1(x) = d(q_1(x)) = d(q_2(x)) = p_2(x)$. Daher ist $D$ injektiv.
              \checkmark

              \textbf{Surjektivität von $D$}: \\
              Angenommen, es gibt $p'(x) \in P_{\le n-1}[x]$. Da $d$ surjektiv ist, existiert ein Polynom $q(x) \in P_{\le n}[x]$ mit $d(q(x)) = p'(x)$.

              Da $q(x) \in P_{\le n}[x]$, können wir $q(x) = a_1x + \cdots + a_nx^n$
              schreiben. Dann ist $q(x) \in P'_{\le n}[x]$, und es gilt $D(q(x)) = p'(x)$.

              Daher ist $D$ surjektiv. \checkmark

              Da $D$ sowohl injektiv als auch surjektiv ist, ist $D$ bijektiv.

              Die inverse Abbildung $I := D^{-1}: P_{n-1}[x] \to P'_{\le n}[x]$ ist die
              Umkehrung von $D$. Das bedeutet, dass $I$ jedes $p'(x) \in P_{n-1}[x]$ auf das
              Polynom $q(x) \in P'_{\le n}[x]$ abbildet, für das $D(q(x)) = p'(x)$.

              Daher ist die inverse Abbildung $I$ gegeben durch $I(p'(x)) = q(x)$.
    \end{enumerate}
\end{proof}

\end{problem}

\begin{problem}[Erster Isomorphiesatz]{4 Punkte}
Seien $(G, \cdot)$ und $(H, +)$ Gruppen und $f: G \to H$ ein surjektiver
Gruppenhomomorphismus. Zeigen Sie, dass $G/\ker(f)$ isomorph zu $H$ ist. Zeigen
Sie dazu, dass
\begin{enumerate}
    \item $F: G/\ker(f) \to H$ definiert durch $F(K) := f(g)$ für ein $g \in K$ wohl definiert ist. Das heißt, der Wert von $F$ hängt nicht von der Wahl von $g \in K$ aus der Definition ab.
    \item $f = F \circ \pi$, wobei $\pi: G \to G/\ker(f)$, $g \mapsto [g]_{\sim_{\ker(f)}}$ die kanonische Abbildung eines Elements auf seine Äquivalenzklasse bezüglich $\ker(f)$ ist.
    \item $F$ bijektiv ist.
\end{enumerate}

\end{problem}

\begin{problem}[Einfache Körpererweiterungen]{6 Punkte}
Sei $(L, +, \cdot)$ ein Körper, $K \subsetneq L$, und $(K, +, \cdot)$ ebenfalls
ein Körper bezüglich der gleichen Operationen (man nennt $K$ dann Unterkörper
von $L$ analog zur Untergruppe aus Aufgabe 1). Sei weiter $j \in L \setminus K$
mit $j^2 \in K$. Zeigen Sie, dass $K[j] := \{a + bj \in L \mid a, b \in K\}$
mit den Operationen
\begin{align*}
    (a + bj) \oplus (c + dj)  & = a + c + j(b + d),       \\
    (a + bj) \otimes (c + dj) & = ac + j^2bd + j(ad + bc)
\end{align*}
einen Körper bildet.
\begin{proof}
    Um zu beweisen, dass $K[j]$ einen Körper bildet, muss gezeigt werden, dass er alle Körperaxiome erfüllt. \\

    \textbf{Assoziativität}

    Für die Operation $\oplus$:
    \begin{align}
        ((a + bj) \oplus (c + dj)) \oplus (e + fj) & = (a + c) + j(b + d) \oplus (e + fj)         \\
                                                   & = (a + c + e) + j(b + d + f)                 \\
                                                   & = (a + bj) \oplus ((c + dj) \oplus (e + fj))
    \end{align}

    Für die Operation $\otimes$:
    \begin{align}
        ((a + bj) \otimes (c + dj)) \otimes (e + fj) & = (ac + j^2bd + j(ad + bc)) \otimes (e + fj)   \\
                                                     & = ace + j^2bde + j(adf + bcf) + j^3bdf         \\
                                                     & = (ce + j^2df) + j(cf + de) \otimes (a + bj)   \\
                                                     & = (a + bj) \otimes ((c + dj) \otimes (e + fj))
    \end{align}

    \textbf{Kommutativität}

    Für die Operation $\oplus$:
    \begin{align}
        (a + bj) \oplus (c + dj) & = (a + c) + j(b + d)       \\
                                 & = (c + a) + j(d + b)       \\
                                 & = (c + dj) \oplus (a + bj)
    \end{align}

    Für die Operation $\otimes$:
    \begin{align}
        (a + bj) \otimes (c + dj) & = ac + j^2bd + j(ad + bc)   \\
                                  & = ca + j^2db + j(da + bc)   \\
                                  & = (c + dj) \otimes (a + bj)
    \end{align}

    \textbf{Neutrale Elemente}

    Für die Operation $\oplus$ ist das neutrale Element $e_\oplus = (0+0j)$, da \[
        (a + bj) \oplus (0 + 0j)   = (a + 0) + j(b + 0) = a + bj
    \]

    Für die Operation $\otimes$ ist das neutrale Element $e_\otimes = (1+0j)$, da \[
        (a + bj) \otimes (1 + 0j)  = ac + j^2b \cdot 0 + j(ad + bc) = a + bj
    \]

    \textbf{Inverse Elemente}

    Für die Operation $\oplus$ ist das inverse Element $\forall (a + bj) \in K[j]$
    gleich $(a^{-1} - b^{-1}j)$, da \[
        (a + bj) \oplus (a^{-1} - b^{-1}j) = (a + a^{-1}) + j(b - b^{-1}) = 0 + 0j
    \]

    Für die Operation $\otimes$ ist das inverse Element $\forall (a + bj) \in K[j]$
    gleich $(a{(a^2+(-j^2)b^2)}^{-1} - bj{(a^2+(-j^2)b^2)}^{-1})$.

    \textbf{Distributivgesetze}

    Für die Operation $\otimes$ über $\oplus$:
    \begin{align}
        (a + bj) \otimes ((c + dj) \oplus (e + fj)) & = (a + bj) \otimes (ce + j(d + f))                               \\
                                                    & = ace + j^2b(d + f) + j(ad + af + bce)                           \\
                                                    & = ace + j^2(bd + bf) + j(ad + af + bce)                          \\
                                                    & = (ac + j^2bd) + (ae + j^2bf) + j((ad + bc) + (af + be))         \\
                                                    & = (ac + j^2bd + j(ad + bc)) \oplus (ae + j^2bf + j(af + be))     \\
                                                    & = ((a + bj) \otimes (c + jd)) \oplus ((a + bj) \otimes (e + fj))
    \end{align}

    Für die Operation $\otimes$ über $\oplus$:
    \begin{align}
        ((c + dj) \oplus (e + fj)) \otimes (a + bj) & = (ce + j(d + f)) \otimes (a + bj)                               \\
                                                    & = ace + j^2(bd + bf) + j(ad + af + bce)                          \\
                                                    & = (ac + j^2bd) + (ae + j^2bf) + j((ad + bc) + (af + be))         \\
                                                    & = (c + dj) \otimes (a + bj) \oplus (e + fj)                      \\
                                                    & = ((c + dj) \otimes (a + bj)) \oplus ((e + fj) \otimes (a + bj))
    \end{align}

    Damit sind alle Körperaxiome erfüllt, und somit ist gezeigt, dass $K[j]$ ein
    Körper ist.
\end{proof}
\end{problem}

\end{document}

