\documentclass{problemset}

\Lecture{Lineare Algebra und Analytische Geometrie I}
\Problemset{10}
\DoOn{7. Januar 2024}
\author{Michael van Straten}

\usepackage{multicol}

\begin{document}
\maketitle

\begin{problem}[Polynomfunktionen und Koeffizientenvergleich]{7 Punkte}
Hinweis: Sei $n \in \mathbb{N}$. Im Beispiel 3.13 der Vorlesung wurde argumentiert, dass die Monomfunktionen
$p_k(x) := x^k \in \text{Abb}(\mathbb{R}, \mathbb{R})$, $k \in \{0, \ldots, n\}$ linear unabhängig sind. Die Gleichheit in $\text{Abb}(\mathbb{R}, \mathbb{R})$,
die der linearen Unabhängigkeit dabei zugrunde liegt, ist dabei die Gleichheit von Werten der
Funktionen in $\mathbb{R}$ bei allen möglichen Einsetzungen von Argumenten in $\mathbb{R}$.

Linearkombinationen von $\{p_k\}_{k\in\{0,\ldots,n\}}$ über $\mathbb{R}$ sind
Funktionen der Form $x \mapsto a_0 + a_1x + \ldots + a_nx^n$ für $a_i \in
\mathbb{R}$, $i \in \{0, \ldots, n\}$. Die Ausdrücke auf der rechten Seite
dieser Zuweisung sind Elemente des Polynomraums $\mathbb{R}[x]$ in der
Unbekannten $x$. asdad asdad Für Polynome wurde die Gleichheit allerdings über
die Gleichheit ihrer Koeffizienten definiert.

Dies führt auf die Frage, welche auch in den Übungen bereits diskutiert wurde,
ob die beiden Formen der Gleichheit kompatibel sind. Der Struktursatz für
Vektorräume liefert hier eine positive Antwort zumindest für den endlich
dimensionalen Untervektorraum der Polynome vom Grad kleiner oder gleich $n$.

Sei $n \in \mathbb{N}$, $K := \mathbb{R}$ und für $k \in \{0, \ldots, n\}$ sei
$p_k \in \text{Abb}(K, K)$ definiert durch $p(x) := x^k$. Sei weiter $U :=
\text{Span}\{p_k\}_{k\in\{0,\ldots,n\}} \subseteq \text{Abb}(K, K)$.

\begin{enumerate}
    \item Berechnen Sie ohne einen Koeffizientenvergleich zu nutzen $\dim_K(U)$
          und zeigen oder widerlegen Sie $U \cong K^{n+1}$.
    \item Wiederholen Sie i) für $K = F_3$ und $n \leq 2$.
    \item Wiederholen Sie i) für $K = F_3$ und $n > 2$.
\end{enumerate}

\begin{proof}
    $ $
    \begin{enumerate}
        \item \textbf{Annahme}: $U \cong K^{n+1}$

              Um zu zeigen das $U \cong K^{n+1}$, genügt es, $\dim_K(U) = n +
              1$ zu beweisen, da bereits in der Vorlesung bewiesen wurde, dass
              jeder Vektorraum der Dimension $n$ isomorph zum Vektorraum $K^n$
              ist.

              Unsere zu zeigende Annahme lautet daher, dass die Menge
              $\{p_k\}_{k=0}^{n}$ linear unabhängig ist, da diese Menge
              offensichtlich den Vektorraum $U$ erzeugt.

              \textbf{Annahme}: $\{p_k\}_{k=0}^{n}$ ist linear unabhängig.

              Für den Fall, dass $U = \operatorname{Span}{x^0}$ ist, ist die
              lineare Unabhängigkeit trivial.

              Nehmen wir an, dass $\{p_k\}_ck=0c^{n}$ bereits für ein $n$
              linear unabhängig ist. Wir müssen nun zeigen, dass
              $\{p_k\}_{k=0}^{n+1}$ ebenfalls linear unabhängig ist.

              Betrachten wir die Gleichung mit $a_i \in K$:
              \[
                  a_0 x^0 + a_1 x^1 + \ldots + a_{n+1} x^{n+1} = 0.
              \]

              Für den Wert $x = 0$ ergibt sich $a_0 = 0$.

              Formen wir die Gleichung nun um:
              \begin{align}
                  a_1 x^1 + a_2 x^2 + \ldots + a_{n+1}x^{n+1} = x(a_1 + a_2 x + \ldots + a_{n+1}x^n) = 0.
              \end{align}

              Diese Gleichung ist jedoch nur in zwei Fällen null, nämlich wenn
              $x = 0$ oder aufgrund der linearen Unabhängigkeit von
              $\{p_k\}_{k=0}^{n}$ $a_i = 0$ für alle $i$.

              Somit ist $\{p_k\}_{k=0}^{n+1}$ linear unabhängig.

              Daher ergibt sich eine Basis, was wiederum $\dim_K(U) = n + 1$
              für alle $n \in \nats$ impliziert. \checkmark

        \item Wiederholen Sie i) für $K = \field{F}_3$ und $n \leq 2$.

              Bemerken wir, dass alle Polynome mit einem Grad von $n \leq 2$
              die Form $a_0x^0 + a_1x^1 + a_2x^2$ haben.

              Wenn wir $x = 0$ setzen, ergibt sich $a_0 = 0$. Somit haben wir
              $a_1x^1 = - a_2x^2$. Jetzt überprüfen wir, ob es außer der
              trivialen Lösung noch weitere Werte für $a_i$ gibt, die diese
              Gleichung erfüllen.

              Wenn $a_1 = a_2$, dann ergibt sich $x^1 = - x^2$, was für $x = 1$
              nicht erfüllt ist.

              Wenn $a_1 = 1$, dann ergibt sich $x^1 = -2x^2$, was für $x = 2$
              nicht erfüllt ist.

              Wenn $a_1 = 2$, dann ergibt sich $2x^1 = -x^2$, was ebenfalls für
              $x = 2$ nicht erfüllt ist.

              Daher bleibt nur die triviale Lösung für $a_0x^0 + a_1x^1 +
              a_2x^2 = 0$ übrig, bei der $a_i = 0$ ist. Dies impliziert, dass
              die Funktionen $x^0, x^1$ und $x^2$ linear unabhängig sind und
              somit eine Basis bilden.

              Folglich gilt für $K = F_3$ und $n \leq 2$ $\dim_K(U) = n + 1$
              wie in ii sowie $U \cong K^{n+1}$.
    \end{enumerate}
\end{proof}

\end{problem}

\begin{problem}[Untervektorraum-Dimensionen von End(V)]{7 Punkte}
Seien $n, m \in \mathbb{N}$, $V$ ein $K$-Vektorraum mit $\dim_K(V) = n$ und $U \subseteq V$ ein Untervektorraum von $V$
mit $\dim_K(U) = m$. Weiter seien $W_1$ und $W_2$ Untervektorräume von $\text{End}(V) := \text{L}(V, V)$ definiert
durch
\[ W_1 := \{f \in \text{End}(V) \mid f|_U = 0\}, \quad W_2 := \{f \in \text{End}(V) \mid \text{im}(f) \subseteq U\}. \]
Berechnen Sie \begin{enumerate}
    \begin{multicols}{3}
        \item $\dim_K(\text{End}(V))$,
        \item $\dim_K(W_1)$,
        \item $\dim_K(W_2)$,
        \item $\dim_K(W_1 \cap W_2)$,
        \item $\dim_K(W_1 + W_2)$.
    \end{multicols}
\end{enumerate}

\begin{proof}
    $ $

    \begin{enumerate}
        \item $\dim_K(\text{End}(V))$,

              Für jedes $e_i$ müssen die n Komponenten von ${e_i}'$, dem Bild
              von $e_i$, angegeben werden. Es gibt n Basisvektoren $e_i$, daher
              erhält man $\dim_K(\text{End}(V)) = n \times n = n^2$.

        \item $\dim_K(W_1)$,

              $W_1$ ist der Untervektorraum von $\text{End}(V)$, der aus den linearen Abbildungen besteht, die auf $U$ verschwinden. Das bedeutet, dass die Abbildungen in $W_1$ auf $U$ den Nullvektor abbilden. Wenn $U$ eine Dimension von $m$ hat, gibt es $m$ unabhängige Basisvektoren in $U$. Für jede dieser Basisvektoren gibt es $n-m$ unabhängige Basisvektoren in $V$, die nicht in $U$ enthalten sind. Daher gibt es insgesamt $m \cdot (n-m)$ unabhängige lineare Abbildungen, die auf $U$ verschwinden, und somit ist $\dim_K(W_1) = m \cdot (n-m)$.

        \item $\dim_K(W_2)$,

              $W_2$ ist der Untervektorraum von $\text{End}(V)$, der aus den linearen Abbildungen besteht, deren Bild in $U$ enthalten ist. Das bedeutet, dass die Abbildungen in $W_2$ von $V$ nach $U$ abbilden. Da $\dim_K(U) = m$, gibt es $m$ unabhängige Basisvektoren in $U$, und es gibt $n-m$ unabhängige Basisvektoren in $V$, die nicht in $U$ enthalten sind. Daher gibt es insgesamt $m \cdot (n-m)$ unabhängige lineare Abbildungen, deren Bild in $U$ enthalten ist, und somit ist $\dim_K(W_2) = m \cdot (n-m)$.

        \item $\dim_K(W_1 \cap W_2)$,

              $W_1 \cap W_2$ ist der Schnitt der beiden Untervektorräume $W_1$ und $W_2$. Dies bedeutet, dass die Abbildungen in $W_1 \cap W_2$ sowohl auf $U$ verschwinden als auch ihr Bild in $U$ haben müssen. Daher enthält $W_1 \cap W_2$ nur die Nullabbildung. Da der Nullvektor der einzige Vektor in $W_1 \cap W_2$ ist, ist $\dim_K(W_1 \cap W_2) = 0$.

    \end{enumerate}
\end{proof}
\end{problem}

\begin{problem}[Projektionen]{6 Punkte}
Sei $V$ ein $K$-Vektorraum und $P : V \to V$ ein idempotenter Vektorraum-Homomorphismus, d.h.
$P \circ P = P$. Eine solche Abbildung $P$ heißt auch Projektion. Zeigen Sie:
\begin{enumerate}
    \item $\text{Re} : \mathbb{C} \to \mathbb{C}$ ist eine Projektion und berechnen Sie $\text{im}(\text{Re})$, $\ker(\text{Re})$, $\text{im}(\text{Re}^\perp)$ und $\ker(\text{Re}^\perp)$, wobei $\text{Re}^\perp := \text{Id}_\mathbb{C} - \text{Re}$.
    \item $\text{Id}_V - P$ ist eine Projektion.
    \item $\text{im}(P) = \ker(\text{Id}_V - P)$ und $\ker(P) = \text{im}(\text{Id}_V - P)$.
    \item $V = \ker(P) \oplus \text{im}(P)$.
\end{enumerate}

\begin{proof}
    $ $

    \begin{enumerate}
        \item $\text{Re} : \mathbb{C} \to \mathbb{C}$ ist eine Projektion und berechnen Sie $\text{im}(\text{Re})$, $\ker(\text{Re})$, $\text{im}(\text{Re}^\perp)$ und $\ker(\text{Re}^\perp)$, wobei $\text{Re}^\perp := \text{Id}_\mathbb{C} - \text{Re}$.

              Für $\operatorname{Re} := a + ib \mapsto a + i \cdot 0$, mit $z
              \in \field{C}$ und $z := a + ib$, $a,b \in \reals$ ist $
              \operatorname{Re} \circ \operatorname{Re} (a + ib) =
              \operatorname{Re}(a + i \cdot 0) = a \Rightarrow
              \operatorname{Re}$ ist eine Projektion \checkmark.

              \begin{enumerate}
                  \item $\operatorname{im}(\operatorname{Re}) = \reals$ \checkmark
                  \item $\ker(\operatorname{Re}) = \set{z \in \field{C} \mid \operatorname{Re}(z) = 0} = \set{0 + ib \mid b \in \reals}$ \checkmark
                  \item $\operatorname{im}(\operatorname{Re}^\perp) = \set{ z - \operatorname{Re}(z) \mid z \in \field{C}} = \set { a + ib - (a - i \cdot 0) \mid a,b \in \reals} = \set{ z \in \field{C} \mid \operatorname{Re}(z) = 0}$ \checkmark
                  \item $\operatorname{ker}(\operatorname{Re}^\perp) = \set{ z \in \field{C} \mid z - \operatorname{Re} = 0} = \reals $ \checkmark

              \end{enumerate}

        \item $\text{im}(P) = \ker(\text{Id}_V - P)$ und $\ker(P) = \text{im}(\text{Id}_V - P)$.

              Bemerken wir das um im Kern von $P$ zu sein die Bedingung $Id_V -
              P = 0$ gelten muss. Formen wir um erhalten wir $P = Id_V$ also
              alle Elemente aus $v \in V$ für die $P(v) = v$ dies ist
              allerdings genau die Bedingung um im Bild von $P$ zu sein ($P(v)
              = v \Rightarrow P(P(v)) = P(v)$) \checkmark.

              Damit allerdings ein $v \in V$ im Bild von $Id_V - P$ ist muss $v
              = Id_V(v) - P(v)$ gelten. Formen wir um erhalten wir \[
                  v = Id_V(v) - P(v) \Rightarrow v = v - P(v) \Rightarrow 0 = 0 - P(v) \Rightarrow P(v) = 0,
              \] was allerdings die Bedingung ist um im Kern von $P$ zu sein
                 \checkmark.

        \item $V = \ker(P) \oplus \text{im}(P)$.

              Zeigen wir zunächst das $\ker{P} \cap \im{P} = \set{0}$.

              Wie wir bereits in der vorherigen teilaufgabe bemerkt haben muss,
              für $v \in V$ im $\ker P$ gelten das $\Id_V(v) - P(v) = 0$
              kombinieren wir dies mit der Bedingung, um im Bild von $P$ zu
              sein, erhalten wir $v = P(v) = \Id{V}(v) - P(v) = 0$. Somit ist
              der einziege Vektor der diese Bedingung erfüllt $v = 0$ was
              impliziert, dass die Summer $\ker(P) \oplus \text{im}(P)$ direkt
              ist.

              Um zu Zeigen das $V = \ker(P) + \im{P}$ ist wählen wir ein
              generisches $v \in V$. Bemerken wir das \[
                  P(v - P(v)) = P(v) - P(P(v)) = P(v) - P(v) = 0
              \] also $v - P(v) \in \ker(P)$. Wählen wir nun $P(v) \in \im(P)$
                 somit erhalten wir $v - P(v) + P(v) = v$ oder $v$ ist die
                 Summe von einem Element aus dem Kern sowie dem Bild von $P$
                 \checkmark.

    \end{enumerate}
\end{proof}
\end{problem}

\end{document}
