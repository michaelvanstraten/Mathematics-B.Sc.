\documentclass{exam}

\usepackage{amssymb, amsmath, amsthm}
\usepackage[german]{babel}
\usepackage{hyperref}
\usepackage{enumitem}
\usepackage{tabularx}

\title{Lineare Algebra und analytische Geometrie I - 3. Übungsblatt}
\author{Michael van Straten}
\date{\today}

\newtheorem{definition}{Definition}

\begin{document}
\maketitle

\section*{Aufgabe 1:}
Geben Sie alle Äquivalenzrelationen, die zugehörigen Äquivalenzklassen und die Mächtigkeit der
zugehörigen Quotientenmenge auf der Menge $\{1, 2, 3\}$ an. \\
\\
\textit{Lösungen:}
\begin{center}
    \begin{tabular}{ | c | c | c | }
        \hline
        \textbf{Äquivalenzrelation}         & \textbf{Äquivalenzklassen} & $\#\{1,2,3\}/\sim$ \\
        \hline
        $a \sim b: a \equiv_1 b$            & \{1, 2, 3\}                & 1                  \\
        $a \sim b: a = b \lor a,b \not = 3$ & \{1, 2\}, \{3\}            & 2                  \\
        $a \sim b: a = b \lor a,b \not = 1$ & \{1\}, \{2, 3\}            & 2                  \\
        $a \sim b: a = b \lor a,b \not = 2$ & \{1, 3\}, \{2\}            & 2                  \\
        $a \sim b: a = b$                   & \{1\}, \{2\}, \{3\}        & 3                  \\
        \hline
    \end{tabular}
\end{center}

\pagebreak

\section*{Aufgabe 2:}
Für eine Menge $M$ und eine partielle Ordnung $\leq$ auf $M$, heißt ein Element $x \in M$ maximales Element bzgl. $\leq$, wenn für alle $y \in M$ gilt, dass aus $x \leq y$ folgt, dass $x = y$.
\\\\
Sei $m \in \mathbb{N}$ und $M$ eine Menge mit $|M| = m$. Sei außerdem $\leq$ eine partielle Ordnung auf $M$. Zeigen Sie ohne die Verwendung des Auswahlaxioms oder des Lemmas von Zorn, dass $M$ ein maximales Element besitzt.
\\\\
Sie können dafür die folgenden Schritte nutzen. Zeigen Sie:
\begin{enumerate}[label=\roman*)]
    \item
          für alle $n \in \mathbb{N}$ mit $n \leq m$ und beliebige $x_1, x_2, \ldots, x_n \in M$ gilt, dass \[
              x_1 \leq x_2 \leq \ldots \leq x_n \text{ und }x_n \leq x_1 \text{ impliziert } x_1 = x_2 = \ldots = x_n.
          \]
    \item
          falls kein maximales Element in $M$ existiert, dann gibt es für alle $n \in \mathbb{N}$ mit $n \leq m$ und beliebige $x_1, x_2, \ldots, x_n \in M$ mit $x_1 \leq x_2 \leq \ldots \leq x_n$ und $x_i \neq x_j$ falls $i \neq j,
              i, j \in \{1, 2, \ldots, n\}$ ein $x_{n+1} \in M \setminus \{x_1, x_2, \ldots, x_n\}$ mit \[
              x_j \leq x_{n+1} \text{ für alle } j \in \{1, 2, \ldots, n\}.
          \]
    \item
          die Endlichkeit von $M$ steht im Widerspruch zu ii).
\end{enumerate}
\begin{proof}
    \leavevmode
    \begin{enumerate}[label=\roman*)]
        \item
              Da $\le$ eine partielle Ordnung auf $M$ ist, ist $\le$ antisymmetrisch und transitiv.
              Da $\le$ transitiv gilt wenn $x_n \le x_1$ und $x_{n-1} \le x_n \Rightarrow x_{n-1} \le x_1$, sogleich für $x_{n-2}, \dots, x_2$.
              Draus folgt $x_n \leq x_{n-1} \le \dots \le x_1$.
              Da somit $\forall i,j \in \{1,2,\dots n\}$ $i \ne j$ $x_i,x_j \in M \Rightarrow x_i \le x_j \land x_j \le x_i$ ist,
              durch die Transitivität von $\le$ $x_i = x_j$, was zu zeigen war.
              \checkmark
        \item
              Angenommen, es gibt kein maximales Element in $M$. Dann können wir eine beliebige endliche Teilmenge von $M$ betrachten, z.B., $x_1, x_2, \ldots, x_n \in M$ mit $x_1 \leq x_2 \leq \ldots \leq x_n$ und $x_i \neq x_j$ für $i \neq j, i, j \in {1, 2, \ldots, n}$.
              Da es kein maximales Element gibt, können wir für jedes $x_i$ ein $x_{n+1} \in M$ wählen, das nicht in ${x_1, x_2, \ldots, x_n}$ enthalten ist und $x_i \leq x_{n+1}$. Dies ist möglich, da es kein maximales Element gibt, das die Auswahl von $x_{n+1}$ verhindern könnte.
              \checkmark
        \item
              Wenn $M$ endlich, existiert kein Element $x_{n+1} \in M$ mit $\forall j \in \mathbb{N} \le n_{n+1}$, da $M$ sonst nicht endlich wäre.
              Somit muss ein Element $x \in M$ existiert sodass $\forall y \in M \setminus \{x\}: y \le x$.
              \checkmark
    \end{enumerate}
\end{proof}

\pagebreak

\section*{Aufgabe 3:}
Seien $X$ und $Y$ Mengen und $f : X \rightarrow Y$. Beweisen Sie
\begin{enumerate}[label=\roman*)]
    \item für $M \subseteq X$ gilt $M \subseteq f^{-1}(f(M))$,
    \item falls $f$ injektive gilt i) sogar Gleichheit,
    \item für $N \subseteq Y$ gilt $f(f^{-1}(N)) \subseteq N$,
    \item falls $f$ surjektiv gilt iii) sogar Gleichheit,
\end{enumerate}
\begin{proof}
    \leavevmode
    \begin{enumerate}[label=\roman*)]
        \item
              Sei $x \in M$ beliebig aber fest, da $M \subseteq X \Rightarrow x \in X$.
              Lass $A = f(M) = \{f(\tilde{x}) \mid \tilde{x} \in M\}$, somit ist $f(x) \in A$.
              Lass nun $U = f^{-1}(A) = \{\tilde{x} \in X \mid f(\tilde{x}) \in A\}$.
              Somit ist $x \in U$, da $x \in X$ und $f(x) \in A$.
              Da $x$ beliebig gewählt wurde, ist somit $M \subseteq U$. \checkmark
        \item
              Wenn, $f$ injektiv dann existieren keine zwei $x_1, x_2$ für die gilt $f(x_1) = f(x_2) \land x_1 \not= x_2$.
              Somit muss es für jedes $y \in A$, ein wohl bestimmtes $x$ geben für das gilt $x \in X \land x \in M \land f(x) = y$.
              Da jedes $x \in U$ somit auch in $M$ ist, muss $U \subseteq M$ oder anders $M = U$. \checkmark
        \item
              Lass $U = f^{-1}(N) = \{x \in X \mid f(x) \in N\}$, somit gilt $\forall x \in U \Rightarrow f(x) \in N$. \\
              Lass jetzt $A = f(U) = \{f(x) \mid x \in U\}$, da $\forall x \in U$ gilt für $f(x) \in N$. \\
              Gilt $\forall y \in f(f^{-1}(N)) \Rightarrow y \in N$, was zu zeigen war. \checkmark
        \item
              Wenn, $f$ surjektive dann gilt $\forall y \in N \Rightarrow \exists x \in X$ sodass $f(x) = y$. \\
              Somit $\exists x \in U: f(x) \in N$, daraus folgt Axiomatisch aus der Definition von A $\forall y \in N: \exists y \in A$.  \\
              Somit ist $N \subseteq f(f^{-1}(N))$ oder anders $f(f^{-1}(N)) = N$. \checkmark
    \end{enumerate}
\end{proof}

\pagebreak

\section*{Aufgabe 4:}
Entscheiden Sie, ob die folgenden Funktionen injektiv, surjektiv und/oder bijektiv sind:
\begin{enumerate}[label=\roman*)]
    \item $f: \mathbb{R} \times \mathbb{R} \rightarrow \mathbb{R}, (x,y) \mapsto x^2 + y^2 -1$
    \item $g: \mathbb{R} \times \mathbb{R} \rightarrow \mathbb{R}, (x,y) \mapsto x + y$
    \item $h: \mathbb{N}_0 \rightarrow \mathbb{Z}$, \[
              n \mapsto \left\{
              \begin{array}{ll}
                  n/2      & \text{falls } n \in {0,2,4, \dots} = [0]_{\equiv_2}, \\
                  -(n+1)/2 & \text{sonst}.
              \end{array}
              \right\}
          \]
\end{enumerate}
\textit{Lösungen:}
\begin{enumerate}[label=\roman*)]
    \item Weder injektiv
          \[
              (-1,-1) \mapsto {(-1)}^2 + {(-1)}^2 - 1 \mapsto 1
              \land
              (1,1) \mapsto {(1)}^2 + {(1)}^2 - 1 \mapsto 1
              \Rightarrow
              (-1,-1) \not = (1,1)
              \Rightarrow\Leftarrow,
          \] noch surjektiv \[
              \forall (x,y) \in \mathbb{R} \times \mathbb{R}: (x,y) \mapsto x^2 + y^2 -1 \ge -1.
          \]
    \item Nicht injektiv
          \[
              (0,1) \mapsto 0 + 1 \mapsto 1
              \land
              (1,0) \mapsto 1 + 0 \mapsto 1
              \Rightarrow
              (0,1) \not = (1,0)
              \Rightarrow\Leftarrow,
          \] aber surjektiv: $(0, y \in R) \mapsto 0 + y \mapsto \mathbb{R}$.
    \item Bijektiv da
          \[
              \lvert \mathbb{N}_0 \lvert = \lvert \mathbb{Z} \lvert \Rightarrow \exists f: \mathbb{N}_0 \mapsto \mathbb{Z}.
          \]
\end{enumerate}

\end{document}

