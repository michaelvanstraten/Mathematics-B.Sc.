\documentclass{problemset}

\Lecture{Analysis I}
\Problemset{5}
\DoOn{28.11.2023}
\author{Michael van Straten}

\setlist[enumerate, 1]{label=\alph*)}
\setlist[enumerate, 2]{label=\alph*)}
\setlist[enumerate, 3]{label=\alph*)}
\setlist[enumerate, 4]{label=\alph*)}

\begin{document}
\maketitle

\begin{problem}[Aussagenlogik und Mengentheoreme]{6 Punkte}
Seien $A$, $B$, $C$ Mengen. Geben Sie, in Analogie mit Bsp Z2.4, detaillierte Beweise mithilfe der Aussagenlogik zu folgenden beiden Mengentheoremen (siehe Z1):
\begin{enumerate}
    \item $A \setminus (B \cup C) = (A \setminus B) \cap (A \setminus C)$
    \item $A \setminus (B \cup C) = (A \setminus B) \setminus C$
\end{enumerate}
\begin{proof}
    $ $
    \begin{enumerate}
        \item $A \setminus (B \cup C) = (A \setminus B) \cap (A \setminus C)$

              Sei $x$ ein beliebiges Element.
              \begin{align*}
                  x \in A \setminus (B \cup C) & \Longrightarrow x \in A \land x \not \in (B \cup C)                             \\
                                               & \Longrightarrow x \in A \land (x \not \in B \land x \not \in C)                 \\
                                               & \Longrightarrow (x \in A \land x \not \in B) \land (x \in A \land x \not \in C) \\
                                               & \Longrightarrow (x \in A \setminus B) \land (x \in A \setminus C)               \\
                                               & \Longrightarrow x \in (A \setminus B) \cap (A \setminus C) \tag{\checkmark}     \\
              \end{align*}

        \item $A \setminus (B \cup C) = (A \setminus B) \setminus C$

              Sei $x$ ein beliebiges Element.
              \begin{align*}
                  x \in A \setminus (B \cup C) & \Longrightarrow x \in A \land x \not\in (B \cup C)                 \\
                                               & \Longrightarrow x \in A \land (x \not\in B \land x \not\in C)      \\
                                               & \Longrightarrow x \in (A \setminus B) \land x \not\in C            \\
                                               & \Longrightarrow x \in (A \setminus B) \setminus C \tag{\checkmark} \\
              \end{align*}
    \end{enumerate}
\end{proof}
\end{problem}

\pagebreak

\begin{problem}[Abbildungen und Verknüpfungen]{8 Punkte}
\begin{enumerate}
    \item Seien $f: X \to Y$, $g: Y \to Z$ und $h = g \circ f$, also $h: X \to
          Z$.
          \begin{enumerate}
              \item Beweisen Sie, dass aus $h$ injektiv folgt, dass $f$
                    injektiv ist.
              \item Beweisen Sie, dass $h$ surjektiv $\Rightarrow$ $g$
                    surjektiv.
              \item Gilt $h$ surjektiv $\Rightarrow$ $f$ surjektiv? Gilt $h$
                    surjektiv $\Rightarrow$ $g$ injektiv? Wenn nicht, dann
                    geben Sie Gegenbeispiele!
              \item Seien nun noch vier Mengen $A$, $B$, $C$ und $D$ zusammen
                    mit drei Abbildungen $f: A \to B$, $g: B \to C$ und $h: C
                    \to D$ gegeben. Beweisen Sie, dass aus der Bijektivit¨at
                    von $g \circ f$ und $h \circ g$ folgt, dass $f$, $g$ und
                    $h$ alle bijektiv sind!
          \end{enumerate}

\end{enumerate}

\begin{proof} $ $
    \begin{enumerate}
        \item $h = g \circ f$ injektiv $\Rightarrow$ $f$ injektiv

              Angenommen, $f(x) = f(y)$. Um die Injektivität von $f$ zu zeigen,
              muss gezeigt werden, dass daraus $x = y$ folgt.

              Betrachten wir $g$ angewendet auf $f(x)$ und $f(y)$. Da $g$ eine
              Abbildung ist, muss sie eindeutig sein. Daher gilt:
              \[
                  g(f(x)) = g(f(y)).
              \]

              Aufgrund der Annahme $h = g \circ f$ haben wir auch $h(x) =
              h(y)$. Da $h$ injektiv ist, folgt $x = y$. Somit ist $f$
              injektiv. \checkmark
        \item $h = g \circ f$ bijektiv $\Rightarrow$ $g$ bijektiv

              Wir betrachten die zusammengesetzte Funktion \(g \circ f: X \to
              Z\) als eine surjektive Abbildung. Nach Definition bedeutet dies,
              dass für jedes \(z \in Z\) ein \(x \in X\) existiert, so dass \(g
              \circ f(x) = z\).

              Wir faktorisieren nun \(g \circ f\) durch \(Y\):

              \[ X \overset{f}{\to} Y \overset{g}{\to} Z; \]

              Daher gilt für alle \(x \in X\), dass \(f(x) \in Y\). Somit
              existiert ein \(y = f(x) \in Y\) mit \(g(y) = g(f(x)) = g \circ
              f(x) = z\).

              Da dies für jedes \(z \in Z\) gilt, ergibt sich:

              \[ g: Y \to Z \]

              Die Funktion \(g\) ist somit ebenfalls surjektiv.

        \item $h$ surjektiv $\not\Rightarrow$ $f$ surjektiv und $h$ surjektiv $\not\Rightarrow$ $g$ injektiv.

              Betrachten wir die Mengen $X = \{x_1\}$, $Y = \{y_1, y_2\}$ und
              $Z = \{z_1\}$ sowie die Funktionen $f: X \to Y$ mit $f(x_1) =
              y_1$ und $g: Y \to Z$ mit $g(y_1) = g(y_2) = z_1$.

              Es folgt, dass $g \circ f$ surjektiv ist, da für jedes $z \in Z$
              ein $x \in X$ existiert, sodass $(g \circ f)(x) = z$. Dies liegt
              daran, dass $(g \circ f)(x_1) = g(f(x_1)) = g(y_1) = z_1$ für
              jedes $z_1 \in Z$ gilt.

              Jedoch ist $f$ nicht surjektiv, da es für das Element $y_2 \in Y$
              kein $x \in X$ gibt, für das $f(x) = y_2$.

              Dies zeigt, dass aus der Surjektivität von $h$ nicht unbedingt
              folgt, dass $f$ surjektiv ist.

              Darüber hinaus zeigt das obige Beispiel, dass $g$ nicht injektiv
              sein muss, da $g(y_1) = g(y_2)$, aber $y_1 \neq y_2$.

              Somit ist gezeigt, dass die Surjektivität von $h$ nicht
              notwendigerweise die Surjektivität von $f$ impliziert, und die
              Surjektivität von $h$ nicht notwendigerweise die Injektivität von
              $g$ impliziert. \checkmark

        \item $g \circ f$ und $h \circ g$ bijektiv $\Rightarrow$ $f$, $g$ und $h$ bijektiv

              Da $g \circ f$ bijektiv ist, ist $f$ injektiv und $g$ surjektiv.
              Ebenso ist $g \circ h$ bijektiv, was bedeutet, dass $g$ injektiv
              und $h$ surjektiv ist. Daher ist $g$ bijektiv.

              Da $g$ bijektiv ist, existiert die inverse Funktion $g^{-1}$, die
              ebenfalls bijektiv ist. Somit kann $f$ als die Komposition zweier
              bijektiver Funktionen geschrieben werden: $f = g^{-1} \circ (g
              \circ f)$. Da die Komposition zweier bijektiver Funktionen wieder
              bijektiv ist, folgt, dass $f$ bijektiv ist.

              Analog dazu lässt sich zeigen, dass auch $h$ bijektiv ist: $h =
              (h \circ g) \circ g^{-1}$.

    \end{enumerate}
\end{proof}
\end{problem}

\pagebreak

\begin{problem}[Graphen]{6 Punkte}
\begin{enumerate}
    \item Seien $X := \{0, 1, 2, 3, 4\}$ und $Y := \{0, 5, 10, 15\}$, sowie $A
          := X \times Y = \{(x, y) \mid x \in X, y \in Y\}$ und $B := \{x + y
          \mid x \in X, y \in Y\}$ gegeben. Definiere dann $R \subset A \times
          B$ durch $R := \{((x, y), x + y) \mid x \in X, y \in Y\}$. Zeigen
          Sie, dass es eine bijektive Abbildung $f : A \to B$ gibt, so dass $R$
          der Graph von $f$ ist.

    \item Sei $X$ eine nichtleere Menge. Welche Eigenschaften muss $X$ haben,
          damit $X \times X$ der Graph einer Abbildung von $X$ nach $X$ ist?

\end{enumerate}

\begin{proof} $ $
    \begin{enumerate}
        \item $f: A \rightarrow B$ mit Graph von $f$ gleich $R$

              Wählen wir \(f: (x, y) \mapsto x + y\) für \(x \in X\) und \(y
              \in Y\). Es ist offensichtlich, dass der Graph von \(f\) genau
              \(R\) ist.

              Nun müssen wir zeigen, dass \(f\) bijektiv ist, das heißt, \(f\)
              injektiv und surjektiv ist.

              \textbf{\(f\) surjektiv}:

              Um zu zeigen, dass \(f\) surjektiv ist, müssen wir zeigen, dass
              für jedes \(b \in B\) ein \(a \in A\) existiert, so dass \(f(a) =
              b\). Aus der Definition der Menge \(B\) folgt leicht, dass für
              jedes \(x + y \in B\) ein \((x, y) \in A\) existiert, wobei
              \(f((x, y)) = x + y\).

              \textbf{\(f\) injektiv}:

              Annahme: \(f\) ist injektiv, das heißt, es existieren \((x_1,
              y_1)\) und \((x_2, y_2)\) in \(A\) mit \(f((x_1, y_1)) = f((x_2,
              y_2))\) und \((x_1, y_1) = (x_2, y_2)\).

              Um zu zeigen, dass für alle \(x_1, x_2 \in X := \{0, 1, 2, 3,
              4\}\) und \(y_1, y_2 \in Y := \{0, 5, 10, 15\}\) gilt: Wenn \(x_1
              + y_1 = x_2 + y_2\), dann folgt \(x_1 = x_2\) und \(y_1 = y_2\),
              können wir dies durch Fallunterscheidung demonstrieren.

              Es gibt nur eine begrenzte Anzahl von Kombinationen für \(x_1,
              x_2, y_1, y_2\) in den gegebenen Mengen \(X\) und \(Y\). Wir
              können die möglichen Fälle durchgehen:

              1. \(x_1 = 0\): In diesem Fall ist \(y_1\) beliebig, und wir haben \(0 + y_1 = x_2 + y_2\). Die einzige Möglichkeit für \(x_2\) und \(y_2\), um dies zu erfüllen, ist \(x_2 = 0\) und \(y_2 = y_1\). Daher folgt \(x_1 = x_2\) und \(y_1 = y_2\).

              2. \(x_1 = 1\): In diesem Fall ist \(y_1\) beliebig, und wir haben \(1 + y_1 = x_2 + y_2\). Die einzige Möglichkeit für \(x_2\) und \(y_2\), um dies zu erfüllen, ist \(x_2 = 1\) und \(y_2 = y_1\). Daher folgt \(x_1 = x_2\) und \(y_1 = y_2\).

              3. \(x_1 = 2\): Ebenso wie in den vorherigen Fällen folgt \(x_1 = x_2\) und \(y_1 = y_2\).

              4. \(x_1 = 3\): Wiederholung des obigen Arguments.

              5. \(x_1 = 4\): Wiederholung des obigen Arguments.

              Da in jedem Fall \(x_1 = x_2\) und \(y_1 = y_2\) folgt, wenn
              \(x_1 + y_1 = x_2 + y_2\), haben wir gezeigt, dass die gegebene
              Aussage für alle \(x_1, x_2 \in X\) und \(y_1, y_2 \in Y\) gilt.

              Dann ergibt sich:

              \[
                  x_1 + y_1 = x_2 + y_2 \Rightarrow x_1 = x_2 \land y_1 = y_2
              \]

              Damit ist gezeigt, dass \(f\) bijektiv ist.

        \item Eigenschaften einer Abbildung mit Graphen $X \times X$

              Damit der Graph einer Abbildung, bezeichnet als $f: X \rightarrow
              X$, gleich dem kartesischen Produkt $X \times X$ ist, muss die
              Menge $X$ einelementig sein. Andernfalls wäre die Abbildung nicht
              eindeutig definiert, was bedeutet, dass für einen bestimmten
              $x$-Wert mehr als genau ein Wert $f(x)$ existieren würde.

              Um dies zu verdeutlichen, betrachten wir ein Beispiel:
              Angenommen, $X = \{a, b\}$. Dann ist das kartesische Produkt $X
              \times X = \{(a, a), (a, b), (b, a), (b, b)\}$. In diesem Fall
              kann jedes Element von $X \times X$ als Paar $(x, y)$ dargestellt
              werden, wobei $x$ und $y$ Elemente von $X$ sind. Wenn $X$ mehr
              als ein Element enthält, besteht die Möglichkeit, dass
              verschiedene Paare unterschiedlichen Abbildungswerten zugeordnet
              werden können, was der Voraussetzung einer Abbildung
              widerspricht.

              Daher muss die Menge $X$ einelementig sein, damit der Graph der
              Abbildung mit $X \times X$ übereinstimmt.

    \end{enumerate}
\end{proof}

\end{problem}

\pagebreak

\begin{problem}[Injektive Abbildungen]{4 Sonderpunkte}
Seien $A$, $B$ nichtleere Mengen und $f : A \to B$, $g : B \to A$ injektive Abbildungen. Für jede Teilmenge $C \subset A$ sei die Menge $F(C)$ definiert durch
\[ F(C) := A \setminus g(B \setminus f(C)). \]

\begin{enumerate}
    \item Nehmen Sie zunächst an, es gebe eine nichtleere Teilmenge $C \subset
          A$ mit $F(C) = C$. Zeigen Sie, dass dann eine bijektive Abbildung von
          $A$ nach $B$ existiert.
    \item Lassen Sie nun die Zusatzannahme aus a) fallen, und beweisen Sie,
          dass solch eine Menge $C \subset A$ mit $F(C) = C$ existiert. (Dann
          haben Sie zusammen mit Ihrem Resultat aus a) bewiesen: Wenn jede von
          zwei Mengen $A$ und $B$ injektiv in die jeweils andere abgebildet
          werden kann, dann existiert eine Bijektion von $A$ auf $B$.)
\end{enumerate}

\begin{proof}
    \begin{enumerate}
        \item
    \end{enumerate}
\end{proof}
\end{problem}

\end{document}
