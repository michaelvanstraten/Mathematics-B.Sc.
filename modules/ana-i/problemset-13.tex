\documentclass{problemset}

\usepackage{mathrsfs}

\Lecture{Analysis I}
\Problemset{13}
\DoOn{6.2.2024}
\author{Michael van Straten}

\begin{document}
\maketitle

\setlist[enumerate, 1]{label=\alph*)}

\begin{problem}[Riemann-Summen]{9 Punkte}
Berechne die folgenden Integrale mittels Riemannscher Summen:

\begin{enumerate}
    \item $\int_{0}^{a} e^x \,dx$
    \item $\int_{0}^{a} \sin x \,dx$
    \item $\int_{1}^{a} \log x \,dx$ für $a > 1$
\end{enumerate}

\begin{proof} \leavevmode

    Es sei vorgemerkt das die selbe definition von $\mu$ sowie $\mathscr{Z}$ wie in der Vorlesung verwendet wird
    merklich \[
        \mathscr{Z} \coloneqq ((x_k)_{0 \le k \le n}, (\xi_k)_{1 \le k \le n})
    \] und \[
        \mu(\mathscr{Z}) \coloneqq \max_{1 \le k \le n} (x_k - x_{k - 1}).
    \]

    Zudem bezeichnen das Symbol $\operatorname{H}$ die die Regel von l'Hôpital.

    \begin{enumerate}
        \item Wählen wir zunächst \[
                  x_k = \frac{ak}{n}, \xi_k = x_{k-1}
              \] somit folgt $0 = x_0 < x_1 < \dots < x_a = a$.

              Wählen wir nun \[
                  x_k - x_{k - 1} = \frac{a}{n}
              \] so folgt \[
                  \lim_{n \to \infty} \mu(\mathscr{Z}) = 0.
              \]

              Definieren wir nun unsere Riemann-Summe \[
                  a_n \coloneqq \sum_{k=1}^{n} \frac{a}{n} e^{\frac{a(k-1)}{n}}
              \] somit folgt \begin{align*}
                  a_n = \sum_{k=1}^{n} \frac{a}{n} e^{\frac{a(k-1)}{n}} & = \sum_{k=1}^{n} \frac{a}{n} e^{\frac{a(k-1)}{n}}                 \\
                                                                        & = \sum_{k = 0}^{n} \frac{a}{n} e^{\frac{ak}{n}}                   \\
                                                                        & = \frac{a}{n} \sum_{k = 0}^{n} e^{\frac{ak}{n}}                   \\
                                                                        & = \frac{a}{n} \sum_{k = 0}^{n} {\left(e^{\frac{a}{n}}\right)}^{k} \\
                                                                        & = \frac{a(1 - e^\frac{a(n+1)}{n})}{a(1 - e^{\frac{a}{n}})}        \\
                                                                        & =  \frac{a(1 - e^{a + \frac{a}{n}})}{n(1 - e^{\frac{a}{n}})}      \\
                                                                        & =  \frac{a(e^{a + \frac{a}{n}} - 1)}{n(e^{\frac{a}{n}} - 1)}.
              \end{align*}

              Zeigen wir nur zunächst das \[
                  \lim_{n \to \infty} n(e^\frac{a}{n} - 1) \neq 0
              \] $\Longrightarrow$ \[
                  \lim_{n \to \infty} n(e^\frac{a}{n} - 1) = \lim_{n \to \infty} \frac{e^\frac{a}{n} - 1}{\frac{1}{n}} \overset{\operatorname{H}}{=} \lim_{n \to \infty} ae^\frac{a}{n} = a. \checkmark
              \]

              Somit lässt sich die Quotientenregel anwenden um $\lim_{n \to \infty} a_n$ zu bestimmen $\Longrightarrow$ \[
                  \lim_{n \to \infty} a_n = \frac{\lim_{n \to \infty} a(e^{a + \frac{a}{n}} - 1)}{\lim_{n \to \infty} n(e^{\frac{a}{n}} - 1)} = \frac{a(e^a - 1)}{a} = e^a - 1.
              \]

              Somit folgt \[
                  \int_{0}^{a} e^x \,dx = e^a - 1.
              \]
    \end{enumerate}
\end{proof}
\end{problem}

\begin{problem}[Verknüpfte Treppenfunktion]{3 Punkte}
Sei $f: [a, b] \to [c, d]$ eine Treppenfunktion und $g: [c, d] \to \mathbb{R}$ eine Funktion. Beweisen Sie, dass die Komposition $g \circ f$ ebenfalls eine Treppenfunktion ist.


\begin{proof}
    Für $f$ eine Treppenfunktion wissen wir $\exists x_k$ mit $a = x_0 < \dots < x_n = b$ so das für $\forall \xi_k \in (x_{k-1}, k_n)$ $f(\xi_k) = c_k$ mit $c_k \in [c,d]$.

    Da $g: [c,d] \rightarrow \reals$ gilt $\forall \xi_k \in (x_{k-1}, x_k)$ $g(f(\xi)) = g(c_k)$ mit $c_k \in [c,d]$, somit folgt $g \circ f$ konstant auf den offen Intervallen $x_{k-1}, x_k$ womit folgt $g \circ f$ Treppenfunktion.
\end{proof}


\end{problem}

\begin{problem}[Verhalten unter Verknüpfung]{8 Punkte}
Ziel dieser Übung ist es zu zeigen, dass Verknüpfungen von Riemann-integrierbaren Funktionen im Allgemeinen nicht Riemann-integrierbar sind. Dazu betrachten wir die Riemann-integrierbare Modifizierte Dirichlet-Funktion:

\[ g: [0, 1] \to [0, 1], \quad x \mapsto g(x) = \begin{cases}
        0           & \text{falls } x \text{ irrational},                                                            \\
        \frac{1}{q} & \text{falls } x = \frac{p}{q} \text{ mit teilerfremden } p \in \mathbb{N}, q \in \mathbb{N}^*.
    \end{cases} \]

Finden Sie eine Riemann-integrierbare Funktion $f: [0, 1] \to \mathbb{R}$, so dass $f \circ g$ nicht Riemann-integrierbar ist. Hinweis: Siehe auch Aufgabe 2 der Präsenzübung 14.

\begin{proof}
    Bemerken wir das für alle $M \in \nats$ ein $q \in \ints^*$ mit $q > M$ somit folgt $\frac{1}{q} \in \field{Q}$ sowie
    Betrachten wir zunächst ein $\frac{p}{q} \in \field{Q}$ mit $\frac{p}{q}$ teilerfremden.

    Somit folgt \[
        (g \circ g)\left(\frac{p}{q}\right) = q,
    \] wählen wir nun ein $M \in \nats$.
\end{proof}

\end{problem}

\begin{problem}[Basis für Treppenfunktionen*]{4 Sonderpunkte}
In der LAAG I* haben Sie gelernt:

a) Eine Basis ist eine Teilmenge eines Vektorraumes, mit deren Hilfe sich jeder Vektor des Raumes eindeutig als endliche Linearkombination darstellen lässt.

Sie haben zudem gelernt:

Jeder Vektorraum besitzt eine Basis.

Betrachten Sie nun den in der Vorlesung eingeführten Vektorraum aller Treppenfunktionen auf dem Intervall $[a, b]$. Finden Sie eine Basis dieses Vektorraumes (mit Beweis).
\end{problem}

\end{document}
