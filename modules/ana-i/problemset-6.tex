\documentclass{problemset}

\Lecture{Analysis I}
\Problemset{6}
\DoOn{5.12.2023}
\author{Michael van Straten}

\begin{document}
\maketitle

\setlist[enumerate, 1]{label=\alph*)}

\begin{problem}[Konvergenz und absolute Konvergenz]{10 Punkte}
Untersuchen Sie folgende Reihe auf Konvergenz und absolute Konvergenz:
\begin{enumerate}
    \item \[
              \sum_{n=1}^{\infty} (-1)^n(\sqrt{n + 1} - \sqrt{n}),
          \]
    \item \[
              \sum_{n=1}^{\infty} \frac{n^2}{{(3 + \frac{1}{n})}^n},
          \]
    \item \[
              \sum_{n=2}^{\infty} \frac{2^{n+1}}{5 \cdot 3^n},
          \]
    \item \[
              \sum_{n=1}^{\infty} \frac{2^n \cdot n!}{n^n},
          \]
    \item \[
              \sum_{n=0}^{\infty} \frac{n + 5}{n^2 - 4n + 1}.
          \]
\end{enumerate}
\begin{proof}
    $ $

    \begin{enumerate}
        \item
              \begin{align*}
                  \sum_{n=1}^{\infty} (-1)^n(\sqrt{n + 1} - \sqrt{n}) & = \sum_{n=1}^{\infty} (-1)^n(\sqrt{n + 1} - \sqrt{n}) \left(\frac{\sqrt{n + 1} + \sqrt{n}}{\sqrt{n + 1} + \sqrt{n}}\right) \\
                                                                      & = \sum_{n=1}^{\infty} (-1)^n \left(\frac{n + 1 - n}{\sqrt{n + 1} + \sqrt{n}}\right) \tag {$(a - b)(a+b) = a^2-b^2$}        \\
                                                                      & = \sum_{n=1}^{\infty} (-1)^n \left(\frac{1}{\sqrt{n + 1} + \sqrt{n}}\right)
              \end{align*}

              Da $a_n = \frac{1}{\sqrt{n + 1} + \sqrt{n}}$ eine monoton
              fallende Nullfolge ist, konvergiert die Reihe nach dem
              Leibnitzkriterium, jedoch nicht notwendigerweise absolut. \item
              \begin{align*}
                  \sum_{n \to \infty} \frac{\frac{{(n+1)}^2}{{(3 + \frac{1}{n+1})}^{n+1}}}{\frac{n^2}{{(3 + \frac{1}{n})}^n}} & < \sum_{n \to \infty} \frac{n^2 \cdot {(3 + \frac{1}{n})}^n}{{(3 + \frac{1}{n+1})}^{n+1} \cdot n^2} \\
                                                                                                                              & = \sum_{n \to \infty} \frac{1}{3 + \frac{1}{n+1}}                                                   \\
                                                                                                                              & < \frac{1}{3}
              \end{align*}

              Daher konvergiert die Reihe absolut nach dem Quotientenkriterium.

        \item Da $\left|\frac{2^{n+1}}{5 \cdot 3^n} \right| = \frac{2^{n+1}}{5
              \cdot 3^n}$ (Nenner und Zähler sind immer größer null),
              konvergiert $\sum_{n=2}^{\infty} \frac{2^{n+1}}{5 \cdot 3^n}$
              auch absolut, wenn sie konvergiert.

              \begin{align*}
                  \sum_{n=2}^{\infty} \frac{2^{n+1}}{5 \cdot 3^n} & = \frac{2}{5} \sum_{n=2}^{\infty} \frac{2^n}{3^n}                                                \\
                                                                  & = \frac{2}{5} \sum_{n=2}^{\infty} {\left(\frac{2}{3}\right)}^{n}                                 \\
                                                                  & = \frac{2}{5} \left(\sum_{n=0}^{\infty} {\left(\frac{2}{3}\right)}^{n} - 1 - \frac{2}{3} \right) \\
                                                                  & = \frac{2}{5} \left(\frac{1}{1 - \frac{2}{3}} - 1 - \frac{2}{3}\right)                           \\
                                                                  & = \frac{2}{5} \left(3 - 1 - \frac{2}{3}\right)                                                   \\
                                                                  & = \frac{2}{5} \left(2 - \frac{2}{3}\right)
              \end{align*}

              Daher konvergiert die Reihe.

        \item
              \begin{align*}
                  \lim_{n \to \infty} \frac{{\frac{2^{n+1} \cdot (n+1)!}{n^{n+1}}}}{\frac{2^n \cdot n!}{n^n}} & = \lim_{n \to \infty} \frac{2^{n+1} \cdot (n+1)! \cdot n^n}{n^{n+1} \cdot 2^n \cdot n!}                \\
                                                                                                              & = \lim_{n \to \infty} \frac{2 \cdot 2^n \cdot (n+1)\cdot n! \cdot n^n}{n \cdot n^n \cdot 2^n \cdot n!} \\
                                                                                                              & = \lim_{n \to \infty} \frac{2 \cdot (n+1)}{n}                                                          \\
                                                                                                              & > \lim_{n \to \infty} \frac{2 \cdot (n)}{n}                                                            \\
                                                                                                              & = 2
              \end{align*}

              Da $2 > 1$, divergiert die Reihe nach dem Quotientenkriterium.
    \end{enumerate}
\end{proof}
\end{problem}

\begin{problem}[Binomialreihe]{6 Punkte}
\begin{enumerate}
    \item Sei $a \in \mathbb{R}$ und $x \in \mathbb{R}$ mit $|x| < 1$.
          Untersuchen Sie die folgende Binomialreihe auf Konvergenz und
          absolute Konvergenz:
          \[
              B_a(x) := \sum_{n=0}^{\infty} \binom{a}{n}x^n.
          \]
    \item Zeigen Sie für die Binomialreihe aus a), dass für alle $x \in
          \mathbb{R}$ mit $|x| < 1$ und alle $a, b \in \mathbb{R}$ die folgende
          Funktionalgleichung gilt:
          \[
              B_a(x) \cdot B_b(x) = B_{a+b}(x).
          \]
\end{enumerate}
\end{problem}

\begin{problem}[Sinus und Cosinus hyperbolicus]{4 Punkte}
Beweisen Sie die folgenden Funktionalgleichungen für die Grenzwerte der in der Vorlesung definierten unendlichen Reihen von Sinus hyperbolicus und Cosinus hyperbolicus.
\begin{enumerate}
    \item $\cosh(x + y) = \cosh(x) \cosh(y) + \sinh(x) \sinh(y)$,
    \item $\sinh(x + y) = \cosh(x) \sinh(y) + \sinh(x) \cosh(y)$,
    \item $(\cosh(x))^2 - (\sinh(x))^2 = 1$.
\end{enumerate}
Zusatzfrage: In welchem Sinne gelten diese Funktionalgleichungen auch für die formalen Reihen (als Folgen von Partialsummen)?
\begin{proof}
    $ $

    \begin{enumerate}
        \item $\cosh(x + y) = \cosh(x) \cosh(y) + \sinh(x) \sinh(y)$

              Verwenden wir \(\cosh x + \sinh x = e^x\) und \(\cosh x - \sinh x
              = e^{-x}\).
              \begin{align*}
                  2\cosh(x+y) & = e^{x+y} + e^{-x -y}                                                            \\
                              & = e^xe^y + e^{-x}e^{-y}                                                          \\
                              & = (\cosh x + \sinh x)(\cosh y + \sinh y) +(\cosh x - \sinh x)(\cosh y - \sinh y) \\
                              & = 2\left(\cosh x\cosh y+\sinh x \sinh y\right)
              \end{align*}

              Teilen durch 2:
              \[
                  \cosh(x+y)= \cosh x\cosh y+\sinh x \sinh y\
              \]
        \item $\sinh(x + y) = \cosh(x) \sinh(y) + \sinh(x) \cosh(y)$

              Analog zum vorherigen Beweis:
              \begin{align*}
                  2\sinh(x+y) & = e^{x+y} - e^{-x -y}                                                            \\
                              & = e^xe^y - e^{-x}e^{-y}                                                          \\
                              & = (\cosh x + \sinh x)(\cosh y + \sinh y) -(\cosh x - \sinh x)(\cosh y - \sinh y) \\
                              & = 2\left(\sinh x\cosh y+\sinh y \cosh x\right)
              \end{align*}
              Teilen durch 2:
              \[
                  \sinh(x+y)= \sinh x\cosh y+\sinh y \cosh x
              \]

        \item $(\cosh(x))^2 - (\sinh(x))^2 = 1$

              Verwenden wrr die Tatsachen \(\cosh x + \sinh x = e^x\) und
              \(\cosh x - \sinh x = e^{-x}\).
              \begin{align*}
                  (\cosh x)^2 - (\sinh x)^2 & = (\cosh x + \sinh x)(\cosh x - \sinh x) \\
                                            & = e^x e^{-x}                             \\
                                            & = 1
              \end{align*}
    \end{enumerate}
\end{proof}
\end{problem}

\begin{problem}[Umordnungen der alternierenden Reihe]{4 Sonderpunkte}
Betrachten Sie die alternierende harmonische Reihe, die gegen $a \in \mathbb{R}$ konvergiert:
\[
    \sum_{n=1}^{\infty} \frac{{(-1)}^{n-1}}{n}.
\]
\begin{enumerate}
    \item Finden Sie eine Umordnung dieser Reihe, so dass diese gegen
          $\frac{3}{2}a$ konvergiert.
    \item Finden Sie eine Umordnung, so dass diese bestimmt gegen $-\infty$
          divergiert.
    \item Finden Sie eine Umordnung, so dass diese weder konvergiert noch
          bestimmt gegen $\pm\infty$ divergiert.
\end{enumerate}
\begin{proof}
    \begin{enumerate}
        \item
        \item
        \item
    \end{enumerate}
\end{proof}
\end{problem}

\end{document}

