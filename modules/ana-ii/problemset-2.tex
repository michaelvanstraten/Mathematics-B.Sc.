\documentclass{problemset}

\Lecture{Analysis II}
\Problemset{2}
\DoOn{30.05.2024}
\author{Michael van Straten}

\setlist[enumerate, 1]{label=\alph*)}

\begin{document}
\maketitle

\begin{problem}[Erzeugende Funktion für Catalan-Zahlen]{9 Punkte}
Sei $f : \left[-\frac{1}{2}, \frac{1}{2}\right] \to [-1, 1]$, $x \mapsto f(x) := \begin{cases} \frac{1}{2x} - \frac{1}{2x\sqrt{1 - 4x^2}} & \text{für } x \neq 0, \\ 0 & \text{für } x = 0. \end{cases}$
\begin{enumerate}
    \item Zeigen Sie, dass $f$ eine auf ganz $\left[-\frac{1}{2},
                  \frac{1}{2}\right]$ stetige und auf ganz $]\,-\frac{1}{2},
              \frac{1}{2}[\,$ stetig differenzierbare Funktion ist.
    \item Beweisen Sie, dass sich $f$ um den Entwicklungspunkt 0 in eine
          konvergente Taylor-Reihe entwickeln lässt. Bestimmen Sie den
          Konvergenzradius. Konvergiert die Taylor-Reihe auf ganz
          $\left[-\frac{1}{2}, \frac{1}{2}\right]$ gegen $f$?
    \item Zeigen Sie, dass die Taylor-Reihe von $f$ um 0 gegeben ist durch
          $\sum_{n=0}^{\infty} C_n x^{2n+1}$, wobei $C_n =
              \frac{(2n)!}{n!(n+1)!}$ die sogenannten Catalan-Zahlen sind.
\end{enumerate}
\end{problem}

\begin{problem}[Skalarprodukt für reelle Folgen]{6 Punkte}
Sei $F$ die Menge aller reellen Folgen: $F = \{(x_n)_{n\in\mathbb{N}}\}$.
\begin{enumerate}
    \item Zeigen Sie, dass $F$ mit der Addition $(x_n) + (y_n) := (x_n + y_n)$
          und der skalaren Multiplikation $\lambda \cdot (x_n) := (\lambda
              x_n)$ zum reellen Vektorraum wird.
    \item Sei $\ell^2 := \left\{(x_n) \in F \mid \sum_{n=0}^{\infty} |x_n|^2 <
              \infty\right\}$. Zeigen Sie, dass $\ell^2$ ein Untervektorraum des
          Vektorraums $F$ ist.
    \item Skalarprodukte für reelle Vektorräume genügen den gleichen Axiomen
          (Def 22.7) wie im komplexen Fall. Allerdings darf man dann die
          Semilinearität (Def 22.7 1.a.) im ersten Argument einfach durch
          Linearität ersetzen, und die Hermitizität (Def 22.7 2.) durch
          Symmetrie: $\forall u, v \langle u, v \rangle = \langle v, u
              \rangle$. Für $(x_n), (y_n) \in \ell^2$ sei nun $\langle (x_n), (y_n)
              \rangle := \sum_{n=0}^{\infty} x_n \cdot y_n$. Zeigen Sie, dass
          $\langle \cdot, \cdot \rangle$ ein Skalarprodukt auf $\ell^2$ ist.
\end{enumerate}

\begin{proof}
    \leavevmode
    \begin{enumerate}
        \item Um zu zeigen das \(F\) einen reellen Vektorraum abbildet müssen
              die folgenden Eigenschaft erfüllt sein.

              \textbf{Assoziativität:} Für \(x_n,y_n,z_n
              \in F\) folgt, \[
                  ((x_n) + (y_n)) + (z_n) = (x_n + y_n) + (z_n) = (x_n + y_n + z_n) = (x_n) (y_n + z_n) = (x_n) + ((y_n) + (z_n)).
              \]

              \textbf{Existenz des neutralen Elements:} Für \((0), x_n \in F\) folgt \((0) + (x_n) = (0 + x_n) = (x_n)\).

              \textbf{Existenz von Inversen Elementen:} Für alle \((x_n) \in F\) existiert eine Folge \((-x_n) \in F\) mit \((x_n) + (-x_n) = (0)\).

              \textbf{Kommutativität:} Für \((x_n), (y_n) \in F\) folgt aus der Kommutativität über \(\reals\) das \((x_n) + (y_n) = (y_n) + (x_n)\).

              \textbf{S1:}Für \(\lambda \in \reals, (x_n), (y_n) \in F\) folgt, \[
                  \lambda \cdot ((x_n) + (y_n)) = \lambda \cdot (x_n +_\reals y_n) = (\lambda x_n) + (\lambda y_n) = \lambda \cdot (x_n) + \lambda \cdot (y_n).
              \]

              \textbf{S2:} Folgt analog zu S1.

              \textbf{S3:} Für \(\lambda_1, \lambda_2 \in \reals, (x_n) \in F\) folgt, \[
                  (\lambda_1 +_\reals \lambda_2) \cdot (x_n) = (\lambda_1 x_n +_\reals \lambda_2 x_n) = (\lambda_1 x_n) + (\lambda_2 x_n) = \lambda_1 \cdot (x_n) + \lambda_2 \cdot (x_n)
              \]

              \textbf{Multiplikative Identität:} Für alle \(x_n \in F\) folgt das \(1 \in \reals\) die Identität zur Multiplikation über \(F\) ist da \[
                  1 \cdot (x_n) = (1 x_n) = (x_n)
              \]

        \item Damit \(l^2\) einen Untervektorraum bildet müssen die folgenden
              Bedingungen gelten.

              \begin{enumerate}
                  \item[\(l^2 \neq \emptyset\):] Da \((0) \in F\) und \[
                            \sum_{n = 0}^\infty \abs{0}^2 = 0
                        \] folgt das \((0) \in l^2\), was die nicht leere des
                        Untervektorraums beweist.

                  \item[Geschlossenheit:] Für \(x_n, y_n \in l^2\) folgt aus

                  \item[\(\):] Für \(x_n, y_n \in l^2\) folgt aus
              \end{enumerate}
    \end{enumerate}

\end{proof}

\end{problem}

\begin{problem}[Fourier-Reihe]{5 Punkte}
Berechnen Sie die Fourier-Reihe der Funktion $f(x) = |\sin x|$.
\end{problem}

\begin{problem}[Restglied der Taylor-Formel*]{4 Sonderpunkte}
Sei $p \in \mathbb{N}$ mit $1 \leq p \leq n + 1$. Man beweise für das Restglied $R_{n+1}(x)$ der Taylor-Formel Satz 21.2: Es gibt ein $\xi$ zwischen $a$ und $x$, so dass
\[ R_{n+1}(x) = \frac{f^{(n+1)}(\xi)}{p \cdot n!} (x - \xi)^{n+1-p} (x - a)^p. \]
\end{problem}

\end{document}
