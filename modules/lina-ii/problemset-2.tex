\documentclass{problemset}

\Lecture{Lineare Algebra und Analytische Geometrie II}
\Problemset{2}
\DoOn{28. April 2024 23:59 Uhr}
\author{Michael van Straten}

\begin{document}
\maketitle

\begin{problem}[Eigenwerte und Kompositionen]{5 Punkte}
Sei $K$ ein Körper, $V$ ein $K$-Vektorraum und $F, G \in \text{End}(V)$. Zeigen Sie:
\begin{enumerate}
    \item Ist $v \in V$ ein Eigenvektor von $F \circ G$ und $G(v) \neq 0$, so
          ist $G(v)$ ein Eigenvektor von $G \circ F$ zum gleichen Eigenwert.
    \item Ist $\text{dim}(V) < \infty$, so haben $F \circ G$ und $G \circ F$
          die selben Eigenwerte.
    \item Aussage ii) gilt im Allgemeinen nicht, falls $\text{dim}(V) =
              \infty$.
\end{enumerate}
\end{problem}

\begin{problem}{5 Punkte}
Berechnen Sie das charakteristische Polynom, die Eigenwerte, Eigenvektoren und
Eigenräume für \[
    A = \begin{pmatrix}
        0 & -3 & 3 & 0 \\ -3 & 0 & 3 & 0 \\ -1 & -1 & 2 & 2 \\ -2 & -2 & 4 & 1
    \end{pmatrix} \in \reals^{4 \times 4}.
\]
\begin{proof}
    Sei \(\lambda \in R\) ein Eigenwerte von \(A\) mit dem dazugehörigen
    Eigenvektor \(v \in \reals^4\). So folgt aus \(Av = \lambda v\), \[
        0 = \det(A - \lambda I)
        \annotated{Laplacescher Entwicklungssatz}{=} \sum_{i=1}^{4} (-1)^{i+4} \cdot a_{i4} \cdot \det(A_{i4})
        = \lambda (\lambda - 3)^2(\lambda + 3)
    \] das charakteristische Polynom der Matrix \(A\).

    Die Eigenwerte der Matrix \(A\) lassen sich somit auf \(\lambda_1 = 0,
    \lambda_2 = 3, \lambda_3 = 3, \lambda_4 = -3\) bestimmen.

    Substituieren wir nun \(\lambda_i\) für \(\lambda\) in, \[
        (A - \lambda I)(v) = 0
    \] erhalten wir \[
        v_1 = \begin{pmatrix}
            1 \\ 1 \\ 0 \\ 1
        \end{pmatrix},
        v_2 = \begin{pmatrix}
            1 \\ 0 \\ 1 \\ 1
        \end{pmatrix},
        v_3 = \begin{pmatrix}
            -1 \\ 1 \\ 0 \\ 0
        \end{pmatrix},
        v_4 = \begin{pmatrix}
            1 \\ 1 \\ 1 \\ 0
        \end{pmatrix}
    \] als die respektieren Eigenvektor für \(\lambda_i\).

    Die Eigenräume für \(\lambda_i\) folgen somit mit, \begin{align*}
        \Eig(0, A)  & = \Span( v_1 )      \\
        \Eig(3, A)  & = \Span( v_2, v_3 ) \\
        \Eig(-3, A) & = \Span( v_4 ).
    \end{align*}
\end{proof}
\end{problem}

\begin{problem}[Eigenwerte reeller Matrizen in $\mathbb{C}$]{5 Punkte}
Sei $n \in \mathbb{N}$ und $A \in \mathbb{C}^{n \times n}$ eine Matrix mit
reellen Einträgen. Sei weiter $\lambda \in \mathbb{C} \setminus \mathbb{R}$ ein
Eigenwert von $A$ mit Eigenvektor $v \in \mathbb{C}^n$. Zeigen Sie: $\lambda$
ist ebenfalls ein Eigenwert von $A$ zum Eigenvektor $v = (v_1, \dots, v_n)$.

\begin{definition}[Konjugierte Matrize]
    Ist \(A = (a_{ij}) \in \field{C}^{m \times n}\) eine komplexe Matrize,
    \[
        A=
        \begin{pmatrix}
            a_{11} & \dots & a_{1n} \\
            \vdots &       & \vdots \\
            a_{m1} & \dots & a_{mn}
        \end{pmatrix}
    \]
    dann ist die zugehörige konjugierte Matrize \(\bar{A} \in \field{C}^{m
        \times n}\) definiert als \[
        \bar{A} = (\bar{a}_{ij}) = \begin{pmatrix}
            \bar{a}_{11} & \dots & \bar{a}_{1n} \\
            \vdots       &       & \vdots       \\
            \bar{a}_{m1} & \dots & \bar{a}_{mn}
        \end{pmatrix}.
    \]

    Ist \(a_{ij} \in \reals\) folgt sofort \(\bar{A} = A\).
\end{definition}

\begin{definition}[Real und Imaginär teil einer komplexen Matrize]
    Ist \(A = (a_{ij}) \in \field{C}^{m \times n}\) eine komplexe Matrize,
    dann ist der real beziehungweise imaginärt teil der Matrize wiefolgt definiert, \[
        \Re(A) = (\Re(a_{ij})) = \begin{pmatrix}
            \Re(a_{11}) & \dots & \Re(a_{n})  \\
            \vdots      &       & \vdots      \\
            \Re(a_{m1}) & \dots & \Re(a_{mn})
        \end{pmatrix}, \Im(A) = (\Im(a_{ij})) = \begin{pmatrix}
            \Im(a_{11}) & \dots & \Im(a_{n})  \\
            \vdots      &       & \vdots      \\
            \Im(a_{m1}) & \dots & \Im(a_{mn})
        \end{pmatrix}.
    \]

    Es ist somit leicht zu sehen das \(A = \Re(A) + i \cdot \Im(A)\).
\end{definition}

\begin{proof}
    Sei \(A \in \field{C}^{m \times n}, v \in C^{n \times 1}\). Wir definieren nun
    \(A_\Re \coloneq \Re(A), A_\Im \coloneq \Im(A)\) sowie \(v_\Re \coloneq \Re(v),
    v_\Im \coloneq \Im(v)\) somit folgt, \begin{align}
        \overline{Av} & = \overline{A \cdot (v_\Re + i \cdot v_\Im)}                                                                                                                \\
                      & = \overline{A \cdot v_\Re} + \overline{A \cdot i \cdot v_\Im}                                                                                               \\
                      & = \overline{(A_\Re + i \cdot A_\Im) \cdot v_\Re} + \overline{(A_\Re + i \cdot A_\Im) \cdot i \cdot v_\Im}                                                   \\
                      & = \overline{A_\Re \cdot v_\Re} + \overline{i \cdot A_\Im \cdot v_\Re} + \overline{A_\Re \cdot i \cdot v_\Im} + \overline{i \cdot A_\Im \cdot i \cdot v_\Im} \\
                      & = A_\Re \cdot v_\Re  - i \cdot A_\Im \cdot v_\Re - A_\Re \cdot i \cdot v_\Im + i \cdot A_\Im \cdot i \cdot v_\Im                                            \\
                      & = A_\Re \cdot v_\Re  - A_\Re \cdot i \cdot v_\Im - (i \cdot A_\Im \cdot v_\Re - i \cdot A_\Im \cdot i \cdot v_\Im)                                          \\
                      & = A_\Re \cdot \bar{v} - i \cdot A_\Im \cdot \bar{v}                                                                                                         \\
                      & = \bar{A}\bar{v}.
    \end{align}

    Für \(\lambda \in \field{C}^{1 \times 1}\) und \(A \in \reals^{m,n}\) folgt
    sofort \(Av = \lambda v \Rightarrow \overline{Av} = \overline{\lambda v}
    \Rightarrow A\bar{v} = \bar{\lambda}\bar{v}\).
\end{proof}
\end{problem}

\end{document}
