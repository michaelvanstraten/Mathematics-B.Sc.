\documentclass{problemset}

\Lecture{Lineare Algebra und analytische Geometrie II}
\Problemset{4}
\DoOn{12. Mai 2024}
\author{Michael van Straten}

\begin{document}
\maketitle

\begin{problem}[Eindeutigkeit der Lösung; Rechenregeln für das Matrixexponential]{6 Punkte}
Sei $n \in \mathbb{N}$, $A \in \mathbb{R}^{n,n}$ diagonalisierbar. Zeigen Sie:
\begin{enumerate}
    \item Für $r, s \in \mathbb{R}$ gilt $e^{rA}e^{sA} = e^{(r+s)A}$.
    \item Sei $y_0 \in \mathbb{R}^n$, $T \in \mathbb{R}$ mit $T > 0$, $I := [0,
              T]$ und $D := C^\infty(I; \mathbb{R}^n)$ der Vektorraum der unendlich
          oft stetig differenzierbaren Funktionen von $I$ nach $\mathbb{R}^n$.
          Dann ist $y(t) := e^{At}y_0 \in D$ die einzige Lösung der Gleichungen
          \[ \frac{d}{dt} y(t) = Ay(t), \quad \text{für alle } t \in I, \]
          \[ y(0) = y_0. \]
          Hinweis: Für eine weitere Lösung $w \in D$ betrachten Sie die
          Funktion $z(t) := e^{-At}w(t)$. Sie können die Produktregel für
          Ableitungen für vektor- und matrixwertige Funktionen ohne Beweis
          verwenden.
    \item Ist $B \in \mathbb{R}^{n,n}$ und es gilt $AB = BA$ und alle
          Eigenwerte von $A$ und $B$ haben jeweils die algebraische
          Vielfachheit 1, dann stimmen die Eigenvektoren von $A$ und $B$
          überein.
    \item Ist $B \in \mathbb{R}^{n,n}$ zusätzlich zu den Anforderungen aus iii)
          diagonalisierbar, dann gilt auch $e^{A+B} = e^Ae^B$. Hinweis:
          Betrachten Sie die Funktion $z(t) := e^{(A+B)t}v - e^{At}e^{Bt}v$ für
          einen beliebigen Vektor $v \in \mathbb{R}^n$.
\end{enumerate}
\end{problem}

\begin{problem}[Aufgabe 2]{6 Punkte}

Sei $n \in \mathbb{N}$, $A \in \mathbb{R}^{n,n}$, $T \in \mathbb{R}$ mit $T >
    0$, $I := [0, T]$ und $D := C^\infty(I; \mathbb{R}^n)$ der Vektorraum der
unendlich oft stetig differenzierbaren Funktionen von $I$ nach $\mathbb{R}^n$.
Weiter sei $\mathcal{L} \subset D$ die Menge der Lösungen der
Differenzialgleichung
\[ \frac{d}{dt} y(t) = Ay(t), \quad \text{für alle } t \in I. \]
Zeigen Sie:
\begin{enumerate}
    \item Die Menge der Lösungen $L$ ist ein Untervektorraum von $D$.
    \item Für $\lambda \in \mathbb{R}$ und $v \in \mathbb{R}^n \setminus \{0\}$
          gilt $e^{\lambda t v} =: y(t) \in \mathcal{L}$ genau dann, wenn
          $(\lambda, v)$ ein Eigenpaar von $A$ ist (d.h., $\lambda$ ist ein
          Eigenwert von $A$ zum Eigenvektor $v$).
    \item \label{prob:2:iii} Für $k \in \mathbb{N}$, $\lambda_1, \dots, \lambda_k \in \mathbb{R}$
          und $v_1, \dots, v_k \in \mathbb{R}^n \setminus \{0\}$ mit $y_j(t) :=
              e^{\lambda_j t} v_j$, $j \in \{1, \dots, k\}$, sind folgende Aussagen
          äquivalent:
          \begin{itemize}
              \item[a)] Die Vektoren $v_1, \dots, v_k$ sind linear unabhängig
                    in $\mathbb{R}^n$.
              \item[b)] Für alle $t \in I$ sind die Vektoren $y_1(t), \dots,
                        y_k(t)$ linear unabhängig in $\mathbb{R}^n$.
          \end{itemize}
    \item \label{prob:2:iv} Gilt eine der Bedingungen aus iii), dann sind die Funktionen $y_1,
              \dots, y_k$ linear unabhängig in $D$. Sei nun $A$ diagonalisierbar
          und für $j \in \{1, \dots, n\}$ die Funktionen $y_j$ wie in iii) für
          unterschiedliche Eigenpaare $(\lambda_j, v_j)$ von $A$ definiert.
          Zeigen Sie:
    \item Sind $\alpha_1, \dots, \alpha_n \in \mathbb{R}$ und $y_0 := \alpha_1
              v_1 + \dots + \alpha_n v_n$, dann erfüllt $y := \alpha_1 y_1 + \dots
              + \alpha_n y_n \in L$ die Anfangsbedingung $y(0) = y_0$ und es gilt
          $y(t) = e^{At} y_0$ mit dem Matrixexponential $e^{At}$ aus den
          Übungen.
\end{enumerate}

\begin{proof}
    \leavevmode
    \begin{enumerate}
        \item Da die \(0 \in D\) eine Lösung der Gleichungen
              \begin{equation}
                  \frac{d}{dt} y(t) = A y(t)
              \end{equation}
              ist, folgt das \(L \neq \emptyset\).

              Sei indessen \(y, z \in \mathcal{L}\) sowie \(\lambda \in R\) so
              folgt, dass
              \begin{align*}
                  \lambda (\frac{d}{dt} y(t) + \frac{d}{dt} z(t))                   & = \lambda (A y(t) + A z(t))       \\
                  \Rightarrow \frac{d}{dt} \lambda y(t) + \frac{d}{dt} \lambda z(t) & = A \lambda y(t) + A \lambda z(t) \\
                  \Rightarrow \frac{d}{dt} \lambda (y(t) + z(t))                    & = A \lambda (y(t) + z(t))         \\
                  \Rightarrow \lambda (y(t) + z(t))                                 & \in \mathcal{L}.
              \end{align*}
              Was impliziert das \(\mathcal{L}\) geschlossen und somit
              einen Untervektorraum des Vektorraumes \(D\) ist.

        \item Für \(y(t) \coloneq e^{\lambda t}v \in \mathcal{L}\) folgt, dass
              \begin{equation}
                  \frac{d}{dt} e^{\lambda t}v = \lambda e^{\lambda t}v = A e^{\lambda t}v
              \end{equation} ist, was impliziert das
              \begin{equation}
                  e^{\lambda t} \lambda v = e^{\lambda t} Av.
              \end{equation}

              Da für alle \(t \in I \quad e^{\lambda t} > 0\) ist, folgt somit
              das \((\lambda, v)\) ein Eigenpaar von \(A\) ist.

              Sei nun \((\lambda, v)\) ein Eigenpaar von \(A\), so folgt dass
              \begin{equation}
                  \lambda v = Av.
              \end{equation}

              Wählen wir nun ein beliebigen \(t \in I\) so folgt, dass
              \begin{equation}
                  e^{\lambda t} \lambda v = e^{\lambda t} Av
              \end{equation} ist, was impliziert das \(\lambda e^{\lambda t} v = A e^{\lambda t} v\) ist.

              Definieren wir nun \(y(t) \coloneq e^{\lambda t} v\) so folgt das
              für alle \(t \in I\)
              \begin{equation}
                  \lambda y(t) = A y(t)
              \end{equation} ist, was impliziert das \(y \in \mathcal{L}\).

        \item \leavevmode
              \begin{enumerate}
                  \item [\(a) \Rightarrow b)\)]
                        Angenommen die Vektoren \(v_1, \ldots, v_k\) sind
                        linear unabhängig und wir betrachten eine lineare
                        Kombination aus Vektoren \(y_1(t), \ldots, y_k(t)\) mit
                        \(t \in I\)

                        \begin{equation}
                            \sum_{i=1}^{k} a_i y_i(t) = \sum_{i=1}^{k} a_i e^{\lambda_i t} v_i = 0 \quad \text{für } a_i \in \reals.
                        \end{equation}

                        Da \(\set{v_i}_{1 \le i \le k}\) linear unabhängig sind
                        und \(e^{\lambda_i t} > 0 \quad \forall t \in I\) folgt
                        das \(a_i = 0\) sein muss, woraus Aussage \(b)\) folgt.

                  \item [\(b) \Rightarrow a)\)]
                        Angenommen für alle \(t \in I\) sein die Vektoren
                        \(y_1(t), \ldots, y_k(t)\) linear unabhängig in \(\reals^n\).
                        So folgt das
                        \begin{equation*}
                            \sum_{i=1}^{k} a_i y_i(0) = \sum_{i=1}^{k} a_i e^{\lambda_i 0} v_i = \sum_{i=1}^{k} a_i v_i = 0 \quad \text{für } a_i \in \reals
                        \end{equation*} impliziert das \(a_i = 0\), woraus Aussage \(a)\) folgt.
              \end{enumerate}
        \item Aus der Linearität der Vektoren \(y_i(t)\) über alle \(t \in I\)
              und der Äquivalenz der Aussagen im \autoref{prob:2:iii} folgt
              sofort die lineare Unabhängigkeit der Funktion \(y_i\) über
              \(D\).

        \item Für \(y_i(t) \coloneq e^{\lambda_it}v_i\) folgt mit \(y_i(0) =
              v_i\) sofort, dass
              \begin{equation*}
                  y(0) = y_0.
              \end{equation*}

              Betrachten wir nun \(e^{At}y_0\), so ergibt sich
              \begin{align*}
                  e^{At}y_0 & = \sum_{k=0}^{\infty} \frac{A^k t^k}{k!} y_0                                                                                  \\
                            & = \sum_{k=0}^{\infty} \sum_{i=1}^{n} \frac{A^k t^k a_i v_i}{k!}                                                               \\
                            & = \sum_{k=0}^{\infty} \sum_{i=1}^{n} \frac{t^k a_i A^k v_i}{k!}                                                               \\
                            & \annotated{\((\lambda_i, v_i)\) Eigenpaar von \(A\)}{=} \sum_{k=0}^{\infty} \sum_{i=1}^{n} \frac{t^k a_i \lambda_i^k v_i}{k!} \\
                            & = \sum_{k=0}^{\infty} \sum_{i=1}^{n} \frac{t^k \lambda_i^k a_i v_i}{k!}                                                       \\
                            & \annotated{\(e^x\) konvergiert Absolut}{=} \sum_{i=1}^{n} \sum_{k=0}^{\infty} \frac{t^k \lambda_i^k a_i v_i}{k!}              \\
                            & = \sum_{i=1}^{n} a_i e^{\lambda_i t} v_i                                                                                      \\
                            & = y(t).
              \end{align*}
    \end{enumerate}
\end{proof}

\end{problem}

\begin{problem}[Stark gedämpfte Schwingungen]{3 Punkte}
Es sei $y_0 \in \mathbb{R}^2$, $\omega, \mu \in \mathbb{R}$ mit $\mu > \omega >
    0$ und
\[ A := \begin{pmatrix} 0 & 1 \\ -\omega^2 & -2\mu \end{pmatrix} \in \mathbb{R}^{2,2}. \]
Weiter sei $T \in \mathbb{R}$ mit $T > 0$, $I := [0, T]$ und $D :=
    C^\infty(I; \mathbb{R}^2)$. Gesucht sind Lösungen $y \in D$ der Gleichungen

\begin{align}
    \frac{d}{dt} y(t) = Ay(t) \quad \text{für } t \in I, \quad \\
    y(0) = y_0. \quad
\end{align}

\begin{enumerate}
    \item Bestimmen Sie eine Basis aus Eigenvektoren von $A$ und geben Sie eine
          Basis des Lösungsraums von (1) an.
    \item Bestimmen Sie die Lösung des Gleichungssystems (1), (2).
\end{enumerate}

\begin{proof}
    \leavevmode

    \begin{enumerate}
        \item Für Eigenpaar \((\lambda, v)\) der Matrix \(A\) gilt, dass
              \begin{equation*}
                  \lambda v = Av
              \end{equation*} woraus folgt das
              \begin{equation*}
                  \det(A - \lambda I) = \lambda^2 + 2\mu\lambda + w^2 = 0.
              \end{equation*}

              Somit ergeben sich konkrete Eigenwerte werte
              \begin{equation*}
                  \lambda_1 =  -\mu + \sqrt{\mu^2 - w^2}, \quad \lambda_2 =  -\mu - \sqrt{\mu^2 - w^2},
              \end{equation*} sowie Eigenvektoren
              \begin{equation*}
                  v_1 = \begin{pmatrix}
                      \frac{-\mu + \sqrt{\mu^2 - w^2}}{w^2} \\ 1
                  \end{pmatrix}, \quad
                  v_2 = \begin{pmatrix}
                      \frac{-\mu - \sqrt{\mu^2 - w^2}}{w^2} \\ 1
                  \end{pmatrix},
              \end{equation*} der Matrix \(A\).

              Da \(\lambda_1\) und \(\lambda_2\) paarweise verschieden sind,
              folgt das die Vektoren \(v_1\) und \(v_2\) eine Basis aus
              Eigenvektoren Bilden. Mit \autoref{prob:2:iii} und
              \autoref{prob:2:iv} der Aufgabe 2 folgt das
              \begin{equation*}
                  B_\mathcal{L} \coloneq \set{e^{\lambda_1 t}v_1, e^{\lambda_2 t}v_2}
              \end{equation*} eine Basis des Lösungsraumes von (1) bildet.

        \item Betrachten wir einen beliebigen Vektor \(y_0 \in \reals^2\)
              \begin{equation*}
                  y_0 = \begin{pmatrix}
                      x_1 \\ x_2
                  \end{pmatrix} \quad \text{mit } x_1, x_2 \in \reals
              \end{equation*} sowie einen allgemeinen Vektor des Lösungsraumes von (1)
              \begin{equation}
                  y(t) = \alpha_1 e^{\lambda_1 t} v_1 + \alpha_2 e^{\lambda_2 t} v_2
              \end{equation} so erhalten wir durch das Einsetzen in (2) die folgende Gleichung
              \begin{equation}
                  y(0) = \alpha_1 v_1 + \alpha_2 v_2 = y_0.
              \end{equation}

              Lösen wir besagte Gleichung für \(\alpha_1\) sowie \(\alpha_2\)
              so erhalten wir
              \begin{equation*}
                  \alpha_1 = \frac{x_2 (\sqrt{u^2 - w^2} - u) - w^2 x_1}{2\sqrt{u^2 -w^2}},
                  \quad \text{und} \quad
                  \alpha_2 = - \alpha_1.
              \end{equation*}

              Somit ergibt sich eine Lösung des Gleichungssystems von (1) und
              (2).
    \end{enumerate}
\end{proof}

\end{problem}

\end{document}

