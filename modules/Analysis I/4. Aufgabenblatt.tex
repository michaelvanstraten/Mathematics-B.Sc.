\documentclass{problemset}

\Lecture{Analysis I}
\Problemset{4}
\DoOn{21.11.2023}
\author{Michael van Straten}

\setlist[enumerate, 1]{label=\alph*)}

\begin{document}
\maketitle

\begin{problem}[Ungleichungen mit Wurzeln und zwei Anwendungen]{4 Punkte}
\begin{enumerate}
    \item Beweisen Sie: Für alle natürlichen Zahlen $n \geq 2$ gilt
          \[
              \sum_{k=1}^{n} \frac{1}{\sqrt{k}} > \sqrt{n}.
          \]

    \item Zeigen Sie mithilfe von \textbf{a)}, dass \[
              \sum_{k=1}^{\infty} \frac{1}{\sqrt{k}}
          \] bestimmt gegen $+\infty$ divergiert.

    \item Beweisen Sie: Für alle natürlichen Zahlen $n \geq 1$ gilt
          \[
              \sqrt[n]{n} \leq 1 + \frac{2}{\sqrt{n}}.
          \]

    \item Zeigen Sie mithilfe von c), dass
          \[
              \lim_{{n\to\infty}} \sqrt[n]{n} = 1.
          \]
\end{enumerate}

\begin{proof}
    \begin{enumerate}
        \item \textit{Beweis durch Induktion}, angenommen \textbf{a)} ist bereits für $n=2$ bewiesen, so folgt: \begin{align*}
                  \sum_{k=1}^{n+1} \frac{1}{\sqrt{k}} = \sum_{k=1}^{n} \frac{1}{\sqrt{k}} + \frac{1}{\sqrt{n + 1}} & > \sqrt{n} + \frac{1}{\sqrt{n + 1}}                                              \\
                                                                                                                   & = \sqrt{n+1} \frac{1}{\sqrt{n+1}} \left(\sqrt{n} + \frac{1}{\sqrt{n + 1}}\right) \\
                                                                                                                   & = \sqrt{n+1} \left(\frac{\sqrt{n}\sqrt{n+1} + 1}{n + 1}\right)                   \\
                                                                                                                   & > \sqrt{n+1} \left(\frac{\sqrt{n}\sqrt{n} + 1}{n + 1}\right)                     \\
                                                                                                                   & = \sqrt{n+1} \left(\frac{n + 1}{n + 1}\right)                                    \\
                                                                                                                   & = \sqrt{n+1} (1)                                                                 \\
                                                                                                                   & = \sqrt{n+1} \tag*{\checkmark}.                                                  \\
              \end{align*}
        \item Da $\sum_{k=1}^{n} \frac{1}{\sqrt{k}} > \sqrt{n}$ reicht es zu zeigen das
              $\sqrt{n}$ divergiert und somit transitive auch $\sum_{k=1}^{n}
                  \frac{1}{\sqrt{k}}$. Es ist zu zeigen das für alle $M > 0$ ein $N \in
                  \mathbb{N}$ existiert sodas $\forall n \ge N$ $\mid \sqrt{n} \mod > M$. Setze
              $N = {(M + 1)}^2$, somit ist $\forall n \ge N$ $\sqrt{n} \ge \sqrt{{(M + 1)}^2}
                  = M+1 > M$. Somit divergiert $\sqrt{n}$ bestimmt gegen $+\infty$ und transitive
              auch $\sum_{k=1}^{n} \frac{1}{\sqrt{k}}$. \checkmark
        \item In Satz 4.14 haben wir die Bernoullische Ungleichungen definiert, welche
              aussagt das für alle $x \ge -1$ gilt ${(1+x)}^k \ge 1 + kx$. Setzen wir $k =
                  \frac{n}{2}$, folgt
              \begin{align*}
                  \sqrt[n]{n} \leq 1 + \frac{2}{\sqrt{n}} \Longrightarrow 1 + \frac{2}{\sqrt{n}} \ge \sqrt[n]{n} \\
                  \Longrightarrow {(1 + \frac{2}{\sqrt{n}})}^{\frac{n}{2}} \ge 1 + \frac{n}{2}\frac{2}{\sqrt{n}} \\
                  \Longrightarrow {(1 + \frac{2}{\sqrt{n}})}^{\frac{n}{2}} \ge 1 + \sqrt{n}                      \\
                  \Longrightarrow {(1 + \frac{2}{\sqrt{n}})}^n \ge {(1 + \sqrt{n})}^{2}                          \\
                  \Longrightarrow {(1 + \frac{2}{\sqrt{n}})}^n \ge 1 + 2\sqrt{n} + n                             \\
                  \Longrightarrow {(1 + \frac{2}{\sqrt{n}})}^n > n                                               \\
                  \Longrightarrow 1 + \frac{2}{\sqrt{n}} > \sqrt[n]{n}.
              \end{align*}
              \checkmark
        \item Aus c) wissen wir das $\sqrt[n]{n} < 1 + \frac{2}{\sqrt{n}}$, und da die $n$-te
              Wurzel aus einer Zahlen $\ge 1$ ebenfalls $\ge 1$ sein muss, muss somit $1 \le
                  \sqrt[n]{n} < 1 + \frac{2}{\sqrt{n}}$. Es ist somit nur noch zu zeigen, dass $1
                  + \frac{2}{\sqrt{n}}$ für $n \to \infty$ gegen 1 konvergiert. Bemerke
              \begin{align*}
                  \lim_{n \to \infty} 1 + \frac{2}{\sqrt{n}} & < \lim_{n \to \infty} {( 1 + \frac{2}{\sqrt{n}})}^2                                                                                 \\
                                                             & = \lim_{n \to \infty} 1 + 2\frac{2}{\sqrt{n}} + \frac{4}{n} \tag{Da aus a) $\sqrt{n}$ divergiert muss $ 2\frac{2}{\sqrt{n}} \to 0$} \\
                                                             & = \lim_{n \to \infty} 1 + 0 + \frac{4}{n} \tag{[Konvergenz Kriterien] $\frac{4}{n} \to 0$}                                          \\
                                                             & = \lim_{n \to \infty} 1 + 0 + 0                                                                                                     \\
                                                             & = \lim_{n \to \infty} 1
              \end{align*}
              Somit haben wir $1 \le \sqrt[n]{n} < 1$ was impliziert $\lim_{n \to \infty} \sqrt[n]{n} = 1$
              \checkmark
    \end{enumerate}
\end{proof}
\end{problem}

\begin{problem}[Cauchy-Folgen]{6 Punkte}
Sei $0 < q < 1$ und $(a_n)$ eine Folge mit $|a_{n+1} - a_n| \leq q |a_n - a_{n-1}|$ für alle $n > n_0$.

\begin{enumerate}
    \item Zeigen Sie, dass dann $(a_n)$ eine Cauchy-Folge ist.
    \item Zeigen Sie, dass aus $|a_{n+1} - a_n| < |a_n - a_{n-1}|$ ab einem $n_0$ nicht
          notwendigerweise folgt, dass $(a_n)$ eine Cauchy-Folge ist.
\end{enumerate}
\end{problem}

\begin{problem}[Eine Folge mit Wurzeln]{6 Punkte}
Seien $c > 1$ und $x_0 = 1$ und $(x_n)$ die folgende rekursiv definierte Folge:
\[
    x_{n+1} := \sqrt{c} \cdot x_n, \quad n \in \mathbb{N}.
\]

Zeigen Sie, dass die Folge $(x_n)$ gegen $c$ konvergiert. Hinweis: Der Beweis
kann ähnlich zu dem Beweis von Stz 8.8 geführt werden.
\end{problem}

\begin{problem}[$\sqrt{2}$ als dyadischer Bruch]{4 Sonderpunkte}
\begin{enumerate}
    \item Finden Sie die ersten vier Terme $x_0$, $x_1$, $x_2$, $x_3$ unserer Folge
          $(x_n)$ aus 8.8 (siehe auch 1.6 im Skript) für die Quadratwurzel aus Zwei als
          exakte dyadische (=2-adische) Brüche. Einen Beweis brauchen Sie nur für $x_0$,
          $x_1$, $x_2$ zu führen; für $x_3$ genügt das Resultat.
    \item Kleine Forschungsaufgabe: Finden Sie im Internet oder berechnen Sie mit einem
          Computer \(\sqrt{2}\) als dyadischen Bruch mit einer Genauigkeit von 20 binären
          Stellen hinter dem Punkt.
\end{enumerate}

\begin{proof}
    \begin{enumerate}
        \item \textit{Wurzel Zwei als dyadischer Bruch} \\
              Aus dem Skript 1.6 wissen wir die werte von $x_0, x_1, x_2, x_3$ als dezimal Bruch.
              Es ist nun Lediglich die dezimal Brüche in dyadische Brüche umzuwandeln.
              In Vorlesung 7.10 haben wir die b-adischen Brüche als $\pm \sum_{n=-k}^{\infty}z_nb^{-n}$ definiert.
              Setzen wir $b=2$ erhalten wir für die folgenden Werte: \begin{align*}
                  x_0 = 1 = \frac{1}{1}_2                                  \\
                  x_1 = \frac{2}{3} = \frac{10}{11}_2                      \\
                  x_2 = \frac{17}{12} =_2 \frac{10001}{1100}_2             \\
                  x_3 = \frac{577}{408} =_2 \frac{1001000001}{110011000}_2 \\
              \end{align*}
        \item \textit{$\sqrt{2}$ als dyadischen Bruch}
              Aus \href{https://en.wikipedia.org/wiki/Square_root_of_2}{\textit{la Wikipedia}} ist $\sqrt(2)$ zu zwanzig Stellen $1.01101010000010011110$.

    \end{enumerate}
\end{proof}

\end{problem}

\end{document}
