\documentclass{problemset}

\Lecture{Analysis I}
\Problemset{8}
\DoOn{19.12.2023}
\author{Michael van Straten}

\begin{document}
\maketitle

\setlist[enumerate, 1]{label=\alph*)}

\begin{problem}[Logarithmus, Folgen und Reihen]{6 Punkte}
Sei $(a_k)$ eine Folge positiver reeller Zahlen und bezeichne für jede natürliche Zahl $k > 1$
\[ L_k := \frac{-\log a_k }{\log k}. \]
Zeigen Sie:
\begin{enumerate}
    \item Gilt $\lim_{k\to\infty} \inf L_k > 1$, so konvergiert die Reihe
          $\sum_{k=1}^{\infty} a_k$.
    \item Existiert ein $k_0 \in \mathbb{N}$ so dass $L_k \leq 1$ für alle $k \geq k_0$,
          so divergiert die Reihe $\sum_{k=1}^{\infty} a_k$.
\end{enumerate}
\begin{proof}
    \leavevmode
    \begin{enumerate}
        \item
        \item
    \end{enumerate}
\end{proof}
\end{problem}

\begin{problem}[Reihen und Logarithmen]{8 Punkte}
Es sei $\alpha$ eine positive reelle Zahl. Untersuchen Sie die folgenden Reihen auf Konvergenz
und absolute Konvergenz:
\begin{enumerate}
    \item $\sum_{k=1}^{\infty} \alpha^{\log k}$,
    \item $\sum_{k=2}^{\infty} \alpha^{\log \log k}$,
    \item $\sum_{k=2}^{\infty} \frac{1}{k \cdot (\log k)^\alpha}$,
    \item $\sum_{k=1}^{\infty} (-1)^k \frac{\log(k+1) - \log k}{k}$.
\end{enumerate}
\end{problem}

\begin{problem}[Gleichungen mit Potenzen und Logarithmen]{6 Punkte}
Bestimmen Sie alle reellen Zahlen $x \in \mathbb{R}$, die die folgenden Gleichungen lösen:
\begin{enumerate}
    \item $2^{(3^x)} = 3^{(4^x)}$,
    \item $2 (\log_5 x)^2 + \log_5 (x^3) = 2$. Hierbei ist $\log_a$ die Umkehrfunktion zu $\exp_a$.
\end{enumerate}
\begin{proof}
    \leavevmode
    \begin{enumerate}
        \item
        \item
    \end{enumerate}
\end{proof}
\end{problem}

\begin{problem}[Logarithmus und Stetigkeit*]{4 Sonderpunkte}
Wir betrachten die Logarithmusfunktion $\log : \mathbb{R}^+ \rightarrow \mathbb{R}, x \mapsto \log x$. Zeigen Sie:
\begin{enumerate}
    \item $\log : \mathbb{R}^+ \rightarrow \mathbb{R}$ ist nicht gleichmäßig stetig.
    \item Für jede positive reelle Zahl $a$ ist die Einschränkung $\log [a, \infty[ : [a,
              \infty[ \rightarrow \mathbb{R}$ gleichmäßig stetig.
    \item Machen Sie eine Literaturrecherche, und finden Sie die Definition von
          Lipschitzstetigkeit für reelle Funktionen heraus. Schreiben Sie diese nieder.
    \item Ist $\log [a, \infty[ : [a, \infty[ \rightarrow \mathbb{R}$ lipschitzstetig?
\end{enumerate}
\begin{proof}
    \leavevmode
    \begin{enumerate}
        \item
        \item
        \item
        \item
    \end{enumerate}
\end{proof}
\end{problem}

\end{document}
