\documentclass{problemset}

\author{Michael van Straten}
\Lecture{Analysis I*}
\Problemset{1}
\DoOn{14. November 2023}

\begin{document}
\maketitle

\begin{problem}[Summenformeln]{6 Punkte}
Beweisen Sie folgende Summenformeln für alle $n \in \mathbb{N}$:
\begin{enumerate}
    \item[a)]
          \[
              \sum_{k = 1}^{n} k^3 = \left(\sum_{k = 1}^{n}k\right)^2,
          \]
    \item[b)]
          \[
              \sum_{k = 1}^{n-1} k^2(n-k)^2 = \frac{n(n^4-1)}{30}.
          \]
\end{enumerate}
\begin{proof}
    Für alle $n \in \mathbb{N}$ gilt $\sum_{k = 1}^{n} k^3 = \left(\sum_{k = 1}^{n}k\right)^2$.
    \begin{enumerate}

        \item[a)] \underline{Anfang}: $A(0)$ wahr: $Leere\ Summe = 0 = (Leere\ Summme)^2$
        \item[b)] \underline{Schritt}: Sei $n \ge n_0$. Angenommen $A(n)$ sei schon bewiesen.
              Dann ist \begin{align}
                  \sum_{k = 1}^{n + 1} k^3 & = \sum_{k = 1}^{n}k^3 + (n + 1)^3                                           \\
                                           & = \left(\sum_{k = 1}^{n}k\right)^2 + (n + 1)^3 \tag{Gausische summenformel} \\
                                           & = \left(\frac{n(n+1)}{2}\right)^2 + (n + 1)^3                               \\
                                           & = \frac{n^2(n+1)^2}{4} + (n + 1)^3                                          \\
                                           & = \frac{n^2(n+1)^2}{4} + \frac{4(n + 1)^3}{4}                               \\
                                           & = \frac{n^2(n+1)^2 + 4(n + 1)^3}{4}                                         \\
                                           & = \frac{(n+1)^2(n^2 + 4(n + 1))}{4}                                         \\
                                           & = \frac{(n+1)^2(n^2 + 4n + 4))}{4}                                          \\
                                           & = \frac{(n+1)^2(n +2)^2}{4}                                                 \\
                                           & = \left(\frac{(n+1)(n +2)}{2}\right)^2                                      \\
                                           & = \left(\sum_{k=1}^{n+1}k\right)^2.
              \end{align}
    \end{enumerate}
    Dann folgt aus a), b), dass $A(n)$ für alle $n \ge n_0$ wahr ist.
\end{proof}

\begin{proof}
    Für alle $n \in \mathbb{N}$ gilt $\sum_{k = 1}^{n - 1} k^2(n - k)^2 = \frac{n(n^4 - 1)}{30}$.
    \begin{enumerate}
        \item[a)] \underline{Anfang}: $A(1)$ wahr: $Leere\ Summe = \frac{1(1^4 - 1)}{30} = 0$
        \item[b)] \underline{Schritt}: Sei $n \ge n_0$. Angenommen $A(n)$ sei schon beweisen:
              Dann ist
              \begin{align}
                  \sum_{k = 1}^{n} k^2(n + 1 - k)^2 & = \sum_{k = 1}^{n} k^2(n - k + 1)^2                                                                       \\
                                                    & = \sum_{k = 1}^{n} k^2((n - k)^2 + 2(n-k)+ 1)                                                             \\
                                                    & = \sum_{k = 1}^{n} k^2(n - k)^2 + 2k^2(n-k) + k^2                                                         \\
                                                    & = \sum_{k = 1}^{n} k^2(n - k)^2 + \sum_{k=1}^{n} 2k^2(n-k) + k^2                                          \\
                                                    & = \frac{n(n^4 - 1)}{30} + \sum_{k=1}^{n} 2k^2(n-k) + k^2                                                  \\
                                                    & = \frac{n(n^4 - 1)}{30} + \sum_{k=1}^{n} 2nk^2 - 2\sum_{k=1}^{n} k^3 + \sum_{k=1}^{n} k^2                 \\
                                                    & = \frac{n(n^4 - 1)}{30} + 2n\sum_{k=1}^{n} k^2 - 2\sum_{k=1}^{n} k^3 + \sum_{k=1}^{n} k^2                 \\
                                                    & = \frac{n(n^4 - 1)}{30} + (2n+1)\sum_{k=1}^{n} k^2 - 2\sum_{k=1}^{n} k^3 \tag{aus 1.a}                    \\
                                                    & = \frac{n(n^4 - 1)}{30} + (2n+1)\sum_{k=1}^{n} k^2 - 2\left(\frac{n(n+1)}{2}\right)^2 \tag{aus Vorlesung} \\
                                                    & = \frac{n(n^4 - 1)}{30} + (2n+1)\left(\frac{n(n+1)(2n+1)}{6}\right) - 2\left(\frac{n(n+1)}{2}\right)^2    \\
                                                    & = \frac{n(n^4 - 1)}{30} + (2n+1)\left(\frac{n(n+1)(2n+1)}{6}\right) - 2\frac{n^2(n+1)^2}{4}               \\
                                                    & = \frac{n(n^4 - 1)}{30} + \frac{n(n+1)(2n+1)^2}{6} - \frac{2n^2(n+1)^2}{4}                                \\
                                                    & = \frac{n(n^4 - 1)}{30} + \frac{n(n+1)(2n+1)^2 - 3n^2(n+1)^2}{6}                                          \\
                                                    & = \frac{n(n^4 - 1) + 5n(n+1)(2n+1)^2 - 15n^2(n+1)^2}{30}                                                  \\
                                                    & = \frac{n^5 - n + (5n^2+5n)(2n+1)(2n+1) - 15n^2(n+1)(n+1)}{30}                                            \\
                                                    & = \frac{n^5 - n + (5n^2+5n)(4n^2 + 4n +1) - 15n^2(n^2 + 2n +1)}{30}                                       \\
                                                    & = \frac{n^5 - n + 20n^4 + 20n^3 + 5n^2 + 20n^3 + 20n^2 + 5n - 15n^4- 30n^3 - 15n^2}{30}                   \\
                                                    & = \frac{n^5 + 5n^4 + 10n^3 + 10n^2 + 4n}{30}                                                              \\
                                                    & = \frac{(n+1)(n^4+4n^3+6n^2+4n)}{30} \tag{Polynomdivision}                                                \\
                                                    & = \frac{(n+1)(n^4+4n^3+6n^2+4n + 1 - 1)}{30}                                                              \\
                                                    & = \frac{(n+1)((n^2 + 2n +1)^2 - 1)}{30}                                                                   \\
                                                    & = \frac{(n+1)((n + 1)^4 - 1)}{30}
              \end{align}
    \end{enumerate}
\end{proof}
\end{problem}

\begin{problem}{4 Punkte}
Wir ergänzen nun die Definition 2.9 des Binomialkoeffizienten $n \choose k$ mit
$n \in \mathbb{N}$ und $k \in \mathbb{Z}$ aus der Vorlesung zu $x \choose v$
mit $x \in \mathbb{R}$ und $v \in \mathbb{N}$:
\[
    \binom{x}{v} := \prod_{j=1}^{v}\frac{x-j+1}{j} = \frac{x(x-1)\cdot\ldots\cdot(x-v+1)}{v!}.
\]
Es sein $x \in \mathbb{R}$, $v,l,k \in \mathbb{N}$ und $l \le k$. Beweisen Sie
die folgenden Formeln für die Binomialkoeffizienten $\binom{x}{v}$:
\begin{proof}
    a)
    \begin{displaymath}
        \binom{-x}{v} = (-1)^v\binom{x+v-1}{v}
    \end{displaymath}
    \begin{align}
        \binom{-x}{v} & = \frac{(-x)(-x-1)\cdot\ldots\cdot(-x-v+1)}{v!}                                                 \\
                      & = \frac{(-1)(x)(-1)(x+1)\cdot\ldots\cdot(-1)(x+v-1)}{v!}                                        \\
                      & = \frac{\overbrace{(-1)\cdot\ldots\cdot(-1)}^{v\text{ mal}}(x)(x+1)\cdot\ldots\cdot(x+v-1)}{v!} \\
                      & = \frac{(-1)^v(x)(x+1)\cdot\ldots\cdot(x+v-1)}{v!}                                              \\
                      & = (-1)^v\frac{(x)(x+1)\cdot\ldots\cdot(x+v-1)}{v!}                                              \\
                      & = (-1)^v\frac{(x+v-1)\cdot\ldots\cdot(x+1)(x)}{v!}                                              \\
                      & = (-1)^v\binom{x+v-1}{v}
    \end{align}
\end{proof}
\begin{proof}
    b)
    \begin{displaymath}
        \binom{x+1}{v+1} = \binom{x}{v}\frac{x+1}{v+1}
    \end{displaymath}
    \begin{align}
        \binom{x+1}{v+1} & = \frac{(x+1)x(x-1)\cdot\ldots\cdot(x-v+1)}{(v+1)!}        \\
                         & = \frac{(x+1)x(x-1)\cdot\ldots\cdot(x-v+1)}{(v+1)v!}       \\
                         & = \frac{x+1}{v+1} \frac{x(x-1)\cdot\ldots\cdot(x-v+1)}{v!} \\
                         & = \frac{x+1}{v+1} \binom{x}{v}                             \\
                         & = \binom{x}{v} \frac{x+1}{v+1}
    \end{align}

\end{proof}
\begin{proof}
    c)
    \begin{displaymath}
        \binom{x}{v+1} = \binom{x}{v}\frac{x-v}{v+1}
    \end{displaymath}
    \begin{align}
        \binom{x}{v+1} & = \prod_{j=1}^{v+1}\frac{x-j+1}{j}                    \\
                       & = \prod_{j=1}^{v}\frac{x-j+1}{j}\frac{x-(v+1)+1}{v+1} \\
                       & = \binom{x}{v}\frac{x-v-1+1}{v+1}                     \\
                       & = \binom{x}{v}\frac{x-v}{v+1}
    \end{align}
\end{proof}
\begin{proof}
    d)
    \begin{displaymath}
        \binom{x}{k}\binom{k}{l} = \binom{x}{l}\binom{x-l}{k-l}
    \end{displaymath}
    \begin{align}
        \binom{x}{k}\binom{k}{l} & = \frac{x(x-1)\cdot\ldots\cdot(x-k+1)}{k!}\frac{k(k-1)\cdot\ldots\cdot(k-l+1)}{l!}
    \end{align}
\end{proof}
\end{problem}

\begin{problem}{6 Punkte}
Beweisen Sie 3.14 a) und b) aus der Vorlesung:
\newline\newline
Für alle $n,m \in \mathbb{Z}$ und $x \in \mathbb{R}$ bzw. $x \in \mathbb{R}^* = \mathbb{R}\setminus\{0\}$ im Falle negativer Potenzen gilt
\begin{proof}
    a)
    \begin{displaymath}
        x^nx^m = x^{n+m}
    \end{displaymath}
    Potenzen werden in 3.13 im Skript wie folgt definiert: \\
    Für alle $x \in K$, $n \in \mathbb{N}$ gilt
    \[
        x^0 := 1 \text{ und } x^{n+1}:=x^n \cdot n.
    \]
    Für alle $x \in K^*=K\setminus\{0\}$, $n \in \mathbb{N}$ gilt \[
        x^{-n} := (x^{-1})^n.
    \]
    Somit kann $x^n$ für $n \ge 0$ als: \[
        \underbrace{x \cdot \ldots \cdot x}_{n\text{ mal}}
    \] geschrieben werden oder für $x < 0$ als: \[
        \underbrace{x^{-1} \cdot \ldots \cdot x^{-1}}_{n\text{ mal}}.
    \]
    Dasselbe gilt natürlich auch für $x^m$. \\ Somit kann man $x^nx^m$ auch als: \[
        \underbrace{x \cdot \ldots \cdot x}_{n\text{ mal}} \cdot \underbrace{x \cdot \ldots \cdot x}_{m\text{ mal}}
    \] oder: \[
        \underbrace{x \cdot \ldots \cdot x \cdot x \cdot \ldots \cdot x}_{n+m\text{ mal}}
    \] schreiben, somit gilt $x^nx^m = x^{n+m}$. \\ \\ Es ist zu bemerken das, das
    obere immer noch hält, auch wenn einer der beiden Exponenten negative ist. Da \[
        x^{-n}x^m = {(x^{-1})}^nx^m = \underbrace{x^{-1} \cdot\ldots\cdots x^{-1}}_{n\text{ mal}} \cdot \underbrace{x \cdot \ldots \cdot x}_{m\text{ mal}}
    \] oder vereinfacht \[
        x^{-n}x^m = \underbrace{x \cdot \ldots \cdot x}_{\mathclap{m - n\text{ mal da } x^{-1} \cdot x = 1}} = x^{m-n}
    \] gilt. \\ \\ Wenn beide Exponenten negative sind folgt: \[
        x^{-n}x^{-m} = {(x^{-1})}^n{(x^{-1})}^m = {(x^{-1})}^{n+m} = x^{(-1)(n+m)} = x^{-n+(-m)} = x^{-n-m}.
    \]
\end{proof}
\begin{proof}
    b)
    \begin{displaymath}
        {(x^n)}^m = x^{nm}:
    \end{displaymath}
    Aus 3.a wissen wir bereits das wir $x^n$ als $\underbrace{x \cdot \ldots \cdot x}_{n\text{ mal}}$ schreiben können.
    Somit können wir ${(x^n)}^m$ auch als \[
        (\underbrace{x \cdot \ldots \cdot x}_{n\text{ mal}})^m
    \] schreiben. Dasselbe lässt sich natürlich auch für $z^m$ mach und somit erhalten
    wir \[
        \underbrace{
        \underbrace{
            x \cdot \ldots \cdot x
        }_{n\text{ mal}
        } \cdot \ldots \cdot \underbrace{
            x \cdot \ldots \cdot x
        }_{n\text{ mal}}
        }_{m\text{ mal}}
    \] oder vereinfacht \[
        \underbrace{
            x \cdot \ldots \cdot x \cdot \ldots \cdot x \cdot \ldots \cdot x.
        }_{n \cdot m\text{ mal}}
    \]
    Es gilt also ${(x^n)}^m = x^{nm}$. \\ \\ Das obere gilt auch dann noch wenn
    einer oder beider der oberen Exponenten negative ist: \[
        {(x^{-n})}^m = {({(x^{-1})}^n)}^m = {(x^{-1})}^{x \cdot m}
    \] oder \[
        {(x^{n})}^{-m} = {({(x^n)}^{-1})}^m = {({(x \cdot \ldots \cdot x)}^{-1})}^m = {(x^{-1} \cdot \ldots \cdot x^{-1})}^m = {({(x^{-1})}^n)}^m = {(x^{-1})}^{n \cdot m}
    \] und auch \[
        {(x^{-n})}^{-m} = {({({(x^{-1})}^n)}^{-1})}^m = {({(\underbrace{{(x^{-1})}^{-1}}_{\mathclap{{(x^{-1})}^{-1} = x \text{ laut Definition des inversen Elements}}})}^n)}^m = {({x}^n)}^m = x^{nm}.
    \]
\end{proof}
\end{problem}

\begin{problem}{4 Punkte}
In der LAAG I* haben Sie kürzlich eine Einführung in die Mengenlehre erhalten. Siehe
auch unser Analysis I* Skript Z1, und insbesondere Z1.10. \\
Seien $X$, $Y$ Mengen und $A$, $A^\prime$ Teilmengen von $X$ und $B$, $B^\prime$ Teilmengen von $Y$.
\begin{enumerate}
    \item[a)]
          \begin{proof}
              Zeigen Sie die Formel
              \begin{displaymath}
                  (A \times B) \cap (A^\prime \times B^\prime) = (A \cap A^\prime) \times (B \cap B^\prime).
              \end{displaymath}
              Um das obere zu beweisen, müssen wir die Gleichheit der beiden folgenden Mengen Zeigen.
              \[
                  (A \times B) \cap (A^\prime \times B^\prime) = \{(x,y)\,|\,( x \in A \, \land \, y\in B) \land (x \in A^\prime \, \land \, y \in B^\prime)\}
              \]
              \[
                  (A \cap A^\prime) \times (B \cap B^\prime) = \{(x,y)\,|\,(x \in A \land x \in A^\prime) \land (x \in B \land x \in B^\prime)\}.
              \]
              Somit folgt:
              \begin{align}
                  (A \times B) \cap (A^\prime \times B^\prime) & = \{(x,y)\,|\,( x \in A \, \land \, y\in B) \land (x \in A^\prime \, \land \, y \in B^\prime)\} \\
                                                               & = \{(x,y)\,|\,x \in A \, \land \, y\in B \land x \in A^\prime \, \land \, y \in B^\prime\}      \\
                                                               & = \{(x,y)\,|\,x \in A \land x \in A^\prime \land x \in B \land x \in B^\prime\}                 \\
                                                               & = \{(x,y)\,|\,(x \in A \land x \in A^\prime) \land (x \in B \land x \in B^\prime)\}             \\
                                                               & =  (A \cap A^\prime) \times (B \cap B^\prime).
              \end{align}
          \end{proof}
    \item[b)] Gibt es eine ähnliche Formel für die Vereinigung? \\ Ja es gibt eine
          ähnliche Formel: \[
              (A \times B) \cup (A^\prime \times B^\prime) \subseteq (A \cup A^\prime) \times (B \cup B^\prime).
          \]
          Der Beweis dazu ist ähnliche aber nicht gleich zu dem Beweis in a).
\end{enumerate}
\end{problem}

\end{document}
