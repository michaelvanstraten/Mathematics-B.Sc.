\documentclass{problemset}

\Lecture{Analysis I}
\Problemset{7}
\DoOn{12.12.2023}
\author{Michael van Straten}

\begin{document}
\maketitle

\setlist[enumerate, 1]{label=\alph*)}

\begin{problem}[Infimum von beschränkten Mengen]{4 Punkte}
Man beweise: Sei $A \subset \mathbb{R}$ mit $A \neq \emptyset$, und sei $A$ nach unten beschränkt. Dann gilt: $\exists \inf A$.
Hinweis: Dies ist ein Teil des Satzes 11.7 der Vorlesung, bei dem wir die analoge Aussage für $\sup A$ bewiesen hatten. Sie dürfen ,,spicken", falls nötig. Passen Sie den Beweis der Vorlesung also an den Fall des Infimums an.
\begin{proof}
    $ $

    Angenommen, das Axiom der Vollständigkeit sei gegeben,
    welches im Wesentlichen besagt, dass für jede nach oben beschränkte Menge $A$ das Supremum $\sup A$ existiert.

    Betrachten wir nun eine nach unten beschränkte Menge $A$,
    und sei $-A = \{-a | a \in A\}$.

    \textbf{Behauptung}: Die Menge $-A$ ist nach oben beschränkt.
    Zur Beweisführung beachten wir,
    dass für $A$ eine untere Schranke $M$ existiert, so dass $a \geq M$ für alle $a \in A$.
    Folglich gilt $-a \leq -M$ für alle $-a \in -A$, was darauf hinweist, dass $-A$ nach oben beschränkt ist.

    Gemäß dem Axiom der Vollständigkeit existiere $s = \sup -A$.

    \textbf{Behauptung}: $-s = \inf A$.

    Da $s \geq -a$ für alle $-a \in -A$ gilt, folgt, dass $-s \leq a$ für alle $a \in A$ ist, was $-s$ zu einer unteren Schranke von $A$ macht. Falls $k > -s$, dann ist $-k < s$, was bedeutet, dass $-k$ keine obere Schranke von $-A$ ist. Daraus folgt, dass es ein $e \in -A$ gibt, so dass $-k < e$. Dies führt zu $k > -e$. Da $e \in -A$ bedeutet, dass $-e \in A$ ist, und somit ist $-k$ keine untere Schranke von $A$. Daher ist $-s$ die größte untere Schranke und bestätigt $-s = \inf A$.
\end{proof}
\end{problem}

\begin{problem}[Funktionsgrenzwerte]{8 Punkte}
Bestimmen Sie folgende Funktionsgrenzwerte mit den Methoden der bisherigen Vorlesung:
\begin{enumerate}
    \item \[
              \lim_{x \to 1} \frac{x^n - 1}{x^m - 1} \quad \text{mit} \quad n, m \in \mathbb{N}^*.
          \]
    \item \[
              \lim_{x \to -\infty} \frac{a_nx^n + \ldots + a_1x + a_0}{b_mx^m + \ldots + b_1x + b_0} \quad \text{mit} \quad n, m \in \mathbb{N}, a_j, b_k \in \mathbb{R} \quad \text{und} \quad a_n, b_m \neq 0.
          \]
\end{enumerate}
\begin{proof}
    $ $

    \begin{enumerate}
        \item Sei $f: D \rightarrow \field{R}$ definiert durch $f(x) = \frac{x^n - 1}{x^m - 1}$ mit $n, m \in \mathbb{N}^*$.
              Wir möchten zeigen, dass für jede Folge $x_{k \in \nats}$ mit $\lim_{k \to \infty} x_{k} = 1$ die Funktionsgrenzwerte von $f(x_{k})$ gegen einen konstanten Wert $c$ konvergieren.

              Beobachten wir, dass die Polynome $x^n - 1$ und $x^m - 1$ beide eine Nullstelle bei $x = 1$ für alle $n, m \in \nats$ besitzen. Daher können wir den Linearfaktor $(x - 1)$ ausklammern:

              \[
                  \frac{x^n - 1}{x^m - 1} = \frac{(x - 1)(x^{n-1} + x^{n-2} + \ldots + x + 1)}{(x - 1)(x^{m-1} + x^{m-2} + \ldots + x + 1)} = \frac{x^{n-1} + x^{n-2} + \ldots + x + 1}{x^{m-1} + x^{m-2} + \ldots + x + 1}
              \]

              Für die zusammengesetzte Folge $\frac{{x_k}^{n-1} + {x_k}^{n-2} + \ldots + x_k + 1}{{x_k}^{m-1} + {x_k}^{m-2} + \ldots + x_k + 1}$ ergibt sich, basierend auf den in der Vorlesung behandelten Methoden:

              \[
                  \lim_{k \to \infty} \frac{{x_k}^{n-1} + {x_k}^{n-2} + \ldots + x_k + 1}{{x_k}^{m-1} + {x_k}^{m-2} + \ldots + x_k + 1} = \frac{\overbrace{1 + 1 + \ldots + 1 + 1}^{n \;\text{mal}}}{\underbrace{1 + 1 + \ldots + 1 + 1}_{m \;\text{mal}}}
              \]

              Somit können wir folgern:

              \[
                  \lim_{x \to 1} \frac{x^n - 1}{x^m - 1} = \frac{n}{m}
              \]

        \item
    \end{enumerate}
\end{proof}
\end{problem}

\begin{problem}[Stetigkeit, Injektivität und Monotonie]{8 Punkte}
Sei $I \subset \mathbb{R}$ ein Intervall und sei $f : I \to \mathbb{R}$ stetig. $f$ heiße streng monoton wachsend (bzw. fallend), falls für alle $x, y \in I$ mit $x < y$ gilt $f(x) < f(y)$ (bzw. $f(x) > f(y)$). $f$ heiße streng monoton, falls $f$ entweder streng monoton wachsend oder streng monoton fallend ist. Beweisen Sie:
\[f \text{ injektiv} \iff f \text{ streng monoton}.\]
\begin{proof}
    $ $

    \begin{itemize}
        \item [,,$\Longrightarrow$'']:

              Sei $f$ eine injektive Funktion.

              Wir möchten zeigen, dass entweder für alle $x, y \in I$ mit $x < y$ gilt:
              \[
                  \forall x, y \in I; x < y \Rightarrow f(x) < f(y)
              \]
              oder
              \[
                  \forall x, y \in I; x < y \Rightarrow f(x) > f(y).
              \]

              Nehmen wir an, $x, y, z \in I$ seien beliebig mit $x < y < z$ und $f(x) \leq f(y)$ sowie $f(z) \leq f(y)$.
              Betrachten wir ein $c \in \text{Im}(f)$ mit $f(x) < c < f(y)$ und $f(z) < c < f(y)$.

              Da $f$ stetig ist, existieren $m, n \in I$ mit $x < m < y$ und $y < n < z$, sodass $f(m) = c = f(x)$.
              Aufgrund der Injektivität von $f$ folgt $m = n$.
              Somit haben wir $m < y < n \Rightarrow m < n$ und $m = n$, was zu einem Widerspruch führt ($\Rightarrow\Leftarrow$).

              Folglich kann, wenn $x < y < z$, nicht gelten $x < z$ und $f(x) \leq f(y)$, woraus $f(y) \geq f(z)$ folgen würde.
              Daher ergibt sich, dass $f(y) \leq f(z) \Rightarrow f(x) \leq f(z)$.
              Für den Fall $x < z$ und $f(x) \leq f(z)$ kann die Gleichheit von $f(x)$ und $f(z)$ durch die Injektivität von $f$ ausgeschlossen werden, da sonst $x = z$ folgen würde. Somit ist in diesem Fall gezeigt, dass $x < z \Rightarrow f(x) < f(z)$ (streng monoton wachsend).

              Es bleibt zu beachten, dass der Fall $x, y, z \in I$ mit $x < y < z$ und $f(x) \geq f(y)$ sowie $f(z) \geq f(y)$ betrachtet werden muss. Hier führen jedoch dieselben Argumente dazu, dass $f$ streng monoton fallend sein muss.
        \item [,,$\Longleftarrow$'']:

              Sei $f$ streng monoton wachsend (bzw. fallend) somit ist zu zeigen das $\forall x, y \in I$ mit $x \neq y \Rightarrow f(x) \ne f(y)$.

              Sein $x,y \in I$ beliebig mit $x \ne y$, für den fall das $f$ streng monoton wachsend ist.

              \begin{itemize}
                  \item [\textbf{Fall 1:}] $x < y \annotated{$f$ streng monoton wachsend}{\Rightarrow} f(x) < f(y) \Rightarrow f(x) \ne f(y)$ \checkmark
                  \item [\textbf{Fall 1:}] $x > y \annotated{$f$ streng monoton wachsend}{\Rightarrow} f(x) > f(y) \Rightarrow f(x) \ne f(y)$ \checkmark
              \end{itemize}

              Für den fall $f$ streng monoton fallend, folgt dasselbe argument.
    \end{itemize}
\end{proof}
\end{problem}

\begin{problem}[Supremum stetiger Funktionen*]{4 Sonderpunkte}
Sei $D \subset \mathbb{R}$, $K \in \mathbb{R}$ und für alle $n \in \mathbb{N}$ sei $f_n : D \to \mathbb{R}$ stetig und auf $D$ durch $K$ beschränkt. Zeigen Sie, dass die durch $f(x) := \sup\{f_n(x) \mid n \in \mathbb{N}\}$ definierte Funktion $f : D \to \mathbb{R}$ im Allgemeinen nicht stetig ist.
\begin{proof}
    $ $

    Wähle $f_n: \closedinterval{0}{\infty} \rightarrow \closedinterval{1}{0} := x \mapsto 1 - \frac{1}{x^n}$. Es ist leicht zu erkennen, dass $f$ von oben durch $1$ beschränkt ist.

    Betrachten wir nun die Funktion $f(x) := \sup\{f_n(x) \mid n \in \mathbb{N}\}$ an den Stellen $x = 1$ sowie $x > 1$.

    Für $x = 1$ ist $f(x)$ für alle $n \in \mathbb{N}$ gleich $0$. Für jedes $x > 1$ ist $f(x)$ jedoch gleich $1$, da $\lim_{{n \to \infty}} (1 - \frac{1}{x^n})$ für $x > 1$ gleich $1$ ist.

    Somit ,,springt'' die Funktion am Punkt $x = 1$ von $0$ auf $1$ und ist daher nicht stetig.
\end{proof}
\end{problem}

\end{document}
