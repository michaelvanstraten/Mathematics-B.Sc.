\documentclass{problemset}

\Lecture{Analysis I}
\Problemset{9}
\DoOn{9.1.2024}
\author{Michael van Straten}

\begin{document}
\maketitle

\setlist[enumerate, 1]{label=\alph*)}

\begin{problem}[Abzählbarkeit und Überabzählbarkeit]{10 Punkte}
\begin{enumerate}
	\item Wir betrachten $n$ abzählbare Mengen $A_1, A_2, \ldots, A_n$ und setzen
	      \[ A := A_1 \times A_2 \times \ldots \times A_n := \set{(a_1, a_2, \ldots, a_n) \,|\, a_j \in A_j, j = 1, 2, \ldots, n}. \]
	      Entscheiden Sie, ob $A$ abzählbar oder überabzählbar ist (mit Beweis).
	\item Es seien $B_k$, $k \in \nats$, abzählbare Mengen. Entscheiden Sie, ob
	      \[ B := \bigcup_{k \in \mathbb{N}} B_k \]
	      abzählbar oder überabzählbar ist (mit Beweis).
	\item Sei $M$ eine unendliche Menge. Zeigen Sie, dass eine abzählbar unendliche Menge
	      $M_0 \subset M$ existiert, so dass $M \setminus M_0$ zu $M$ gleichmächtig ist.
\end{enumerate}
\end{problem}

\begin{problem}[Komplexe Zahlen]{4 Punkte}
\begin{enumerate}
	\item Beweisen Sie, dass für zwei komplexe Zahlen $z$ und $w$ das folgende Gesetz
	      gilt:
	      \[ |z + w|^2 + |z - w|^2 = 2(|z|^2 + |w|^2) \]
	      Was bedeutet diese Formel geometrisch?

	      \begin{proof}
		      $ $

		      Seien $z$ und $w$ komplexe Zahlen mit $z = a + bi$ und $w = c + di$, wobei $a,
			      b, c, d \in \mathbb{R}$. Dann gilt:

		      \begin{align*}
			      |z + w|^2 + |z - w|^2 & = |(a + bi) + (c + di)|^2 + |(a + bi) - (c + di)|^2 \\
			                            & = |(a + c) + (b + d)i|^2 + |(a - c) + (b - d)i|^2   \\
			                            & = (a + c)^2 + (b + d)^2 + (a - c)^2 + (b - d)^2     \\
			                            & = 2(a^2 + b^2 + c^2 + d^2)                          \\
			                            & = 2(|a + bi|^2 + |c + di|^2)                        \\
			                            & = 2(|z|^2 + |w|^2).
		      \end{align*}
		      Dies zeigt $|z + w|^2 + |z - w|^2 = 2(|z|^2 + |w|^2)$.

		      Geometrisch entspricht diese Formel einem Parallelogramm in der komplexen
		      Ebene, dessen Ecken von den Punkten $z$ und $w$ auf einer Diagonalen gebildet
		      werden.
	      \end{proof}

	\item Unter welcher Bedingung liegt in der komplexen Ebene der Punkt $z = x + iy$ im
	      Inneren eines Kreises mit dem Radius $R$ und dem Mittelpunkt $c = a + ib$?

	      Der Punkt $z = x + iy$ liegt innerhalb des Kreises mit Mittelpunkt $c$ und
	      Radius $R$, wenn der euklidische Abstand zwischen den Punkten $z$ und $c$
	      kleiner als $R$ ist, d.h., wenn $\sqrt{(x - a)^2 + (y - b)^2} < R$. Wenn
	      $\sqrt{(x - a)^2 + (y - b)^2} = R$ ist, liegt der Punkt auf dem Rand des
	      Kreises.

	\item Zeigen Sie: $z \in \mathbb{C} \Rightarrow \exists r \in \mathbb{R}^+_0$ und
	      $\exists w \in \mathbb{C}$ mit $|w| = 1$ so dass $z = rw$.

	      Für $z := a + ib$ mit $a,b \in \mathbb{R}$ ist $|z| = \sqrt{a^2 + b^2} \in
		      \mathbb{R}$, daher ist auch $\frac{1}{\sqrt{a^2 + b^2}} \in \mathbb{R}$ und
	      somit existiert $r = \frac{1}{\sqrt{a^2 + b^2}}$. Wir wählen $w$ als $w : = zr
		      = \frac{a}{\sqrt{a^2 + b^2}} + \frac{b}{\sqrt{a^2 + b^2}}i$. Dann ergibt sich:

	      \[
		      |w| = \left|\frac{a}{\sqrt{a^2 + b^2}} + \frac{b}{\sqrt{a^2 + b^2}}i\right| = \sqrt{\left(\frac{a}{\sqrt{a^2 + b^2}}\right)^2 + \left(\frac{b}{\sqrt{a^2 + b^2}}\right)^2} = \sqrt{\frac{a^2 + b^2}{a^2 + b^2}} = 1
	      \]

	      Somit existieren für alle $z \in \mathbb{C}$ ein $w \in \mathbb{C}$ und $r \in
		      \mathbb{R}^+_0$ mit $|w| = 1$ und $z = rw$.

	\item Beweisen Sie, dass für zwei komplexe Zahlen $z$ und $w$ das folgende Gesetz
	      gilt:
	      \[ ||z| - |w|| \leq |z - w| \]

	      \begin{proof}
		      $ $

		      Durch Anwendung der Dreiecksungleichung auf $|z| = |z+w-w|$ und $|w| = |w+z-z|$
		      erhalten wir:

		      $|z| \leq |z+w| + |w| \Leftrightarrow |z| - |w| \leq |z+w|$

		      Analog dazu:

		      $|w| \leq |w+z| + |z| \Leftrightarrow |w| - |z| \leq |z+w|$

		      Was zeigt $||z| - |w|| \leq |z - w|$.
	      \end{proof}

\end{enumerate}
\end{problem}

\begin{problem}[Cauchy-Folgen in $\mathbb{Q}$]{6 Punkte}
Seien $(a_n)$ und $(b_n)$ Cauchy-Folgen in $\mathbb{Q}$. Zeigen Sie:
\begin{enumerate}
	\item $(a_n + b_n)$ ist eine Cauchy-Folge in $\mathbb{Q}$.
	\item $(a_n \cdot b_n)$ ist eine Cauchy-Folge in $\mathbb{Q}$.
	\item Die Folge $(a_n)$ ist beschränkt.
	\item Sei $\forall n \in \mathbb{N},\, a_n \neq 0$. Es gebe $c \in \mathbb{Q}^+$ und
	      $N \in \mathbb{N}$ so dass $\forall n \geq N,\, |a_n| \geq c$. Dann ist auch
	      $(a_n^{-1})$ eine Cauchy-Folge in $\mathbb{Q}$.
\end{enumerate}
\end{problem}

\begin{problem}[Nichtarchimedisch angeordneter Körper*]{8 Weihnachtspunkte}

In der Vorlesung wurde erwähnt, dass für einen Körper im Allgemeinen das
archimedische Axiom unabhängig von den Anordnungsaxiomen ist. Es gibt nämlich
angeordnete Körper, die nichtarchimedisch sind. In dieser Aufgabe sollen Sie
ein Beispiel finden und analysieren, indem Sie ggf. wieder eine kleine
Literaturrecherche machen und die Informationen, die Sie finden, sauber
aufschreiben und gegebenenfalls im Detail ausarbeiten.

\begin{enumerate}
	\item Zeigen Sie zunächst, dass die Menge der rationalen Funktionen über $\mathbb{R}$
	      mit geeigneten Verknüpfungen einen Körper bilden.
	\item Finden Sie heraus, welche Anordnung des Körpers aus a) nicht archimedisch ist.
	      (Hinweis: Betrachten Sie für die Definition der Ordnungsrelation die relative
	      Lage von zwei rationalen Funktionen 'im Unendlichen' - kann es unendlich viele
	      Stellen geben, an denen zwei unterschiedliche rationale Funktionen gleich
	      sind?)
	\item Beweisen Sie nun im Detail, dass Ihr Vorschlag aus b) eine Anordnung des
	      Körpers der rationalen Funktionen liefert.
	\item Beweisen Sie abschließend, dass die Anordnung aus a), b) nichtarchimedisch ist.
\end{enumerate}
\end{problem}

\begin{problem}[Weihnachtsaufgaben*]{4+8 Weihnachtspunkte}
\begin{enumerate}
	\item Ferdi bekommt auch dieses Weihnachten wieder sehr viele Geschenke, nämlich
	      abzählbar unendlich viele. Die Pakete, die alle würfelförmig sind, stellt Ferdi
	      mit dem Größten, das einen Meter hoch ist, beginnend nach Größe geordnet in
	      einer Reihe unter dem Tannenbaum auf. Er stellt dabei fest, dass jedes Paket
	      jeweils ein Drittel so breit ist wie das vorherige. Wie weit müssen die Äste
	      des Tannenbaums mindestens ragen, wenn alle Pakete unterm Baum Platz finden?
	      Beim Auspacken stellt Ferdi fest, dass er beim nachfolgenden Paket immer nur
	      jeweils die Hälfte der Zeit braucht. Wie lange hat Ferdi für das erste Paket
	      gebraucht, wenn er, gierig wie er ist, schon nach 2 Minuten alles ausgepackt
	      hat?
	\item Heini bekommt zu Weihnachten von seinem Patenonkel, der unter Heinis Streichen
	      viel leiden musste, einen Würfel von einem Kubikmeter Größe geschenkt. Heini
	      braucht zum Auspacken eine Minute, und im Allgemeinen hängt die Zeit, die Heini
	      zum Auspacken braucht, proportional von der Oberfläche des Päckchens ab. Als er
	      das Paket geöffnet hat, ist in dem Karton wieder ein eingepackter Würfel und
	      $\frac{7}{8}$ m³ Luft. Und so geht es weiter. Nach dem $n$-ten Auspacken findet
	      Heini wieder ein würfelförmiges Päckchen und
	      $\frac{3n^2+3n+1}{n^3\cdot(n+1)^3}$ m³ gähnende Leere. Heini versucht, die
	      leeren Kartons aufeinander zu stapeln. Gelingt ihm das? Zudem machen die Eltern
	      sich Sorgen, ob Heini denn rechtzeitig zum Abendspaziergang zum Onkel mit dem
	      Auspacken fertig sein wird. Packt Heini noch an Neujahr aus? Und warum ist
	      Heini nachher so enttäuscht, dass er nicht mehr mit zum Onkel will?
\end{enumerate}

\begin{proof}
	$ $

	\begin{enumerate}
		\item Fredi bekommt viele Geschenke

		      1. Um herauszufinden, wie weit die Äste des Tannenbaums mindestens ragen müssen, verwenden wir die geometrische Reihe:

		      \[
			      \sum_{n=0}^{\infty} {\left(\frac{1}{3}\right)}^n
		      \]

		      Diese ist eine geometrische Folge mit einem Verhältnis von \(x = \frac{1}{3}\),
		      die für \(x < 1\) gegen \(\frac{1}{1 - x}\) konvergiert. Daher ergibt sich:

		      \[
			      \sum_{n=0}^{\infty} {\left(\frac{1}{3}\right)}^n = \frac{1}{1 - \frac{1}{3}} = \frac{1}{\frac{2}{3}} = \frac{3}{2}
		      \]

		      Daher müssen die Äste des Tannenbaums mindestens 1,5 Meter lang sein.

		      2. Um herauszufinden, wie lange Ferdi für das erste Paket gebraucht hat, verwenden wir die Tatsache, dass er bereits nach 2 Minuten alles ausgepackt hat und für jedes nachfolgende Paket nur die Hälfte der Zeit benötigt.
		      Wir setzen $t$ als die Zeit für das erste Paket:

		      \[
			      \sum_{n=0}^{\infty} t {\left(\frac{1}{2}\right)}^n = t \sum_{n=0}^{\infty} {\left(\frac{1}{2}\right)}^n = 2t = 2 \Rightarrow t = 1
		      \]

		      Ferdi hat also 1 Minute gebraucht, um das erste Paket auszupacken.
		      \[
			      \sum_{n=0}^{\infty} {\left(\frac{1}{3}\right)}^n
		      \]

		\item Heini's Patenonkel ist ein echter Ganster

		      Da Heini nach dem $n$-ten Auspacken immer noch ein weiteres Paket in der Box
		      findet, handelt es sich um eine unendliche Folge von Paketen.

		      Betrachten wir das Volumen der ersten $n$ Geschenke, die Heini auspackt.

		      Für $n = 0$ haben wir das Volumen bereits aus der Aufgabenstellung vorgegeben
		      mit 1.

		      Für $n = 1$, also die erste Box nach dem Auspacken, erhält Heini eine Box und
		      $\frac{7}{8}$ Luft. Da die vorherige Box ein Volumen von 1 hatte, hat die jetzt
		      ausgepackte Box ein Volumen von $1 - \frac{7}{8} = \frac{1}{8}$. Also erkennen
		      wir, dass die Box $n$ ein Volumen von

		      \[
			      1 - \sum_{k=1}^n \frac{3k^2 + 3k + 1}{k^3{(k+1)}^3}
		      \]

		      hat. Via Partialsummenzerlegung, die wir bereits ausführlich in der Vorlesung
		      behandelt haben, können wir das obige zu

		      \[
			      \frac{1}{{(n+1)}^3}
		      \]

		      vereinfachen und somit das Volumen für Box $n$ erhalten.

		      Um nun zu bestimmen, ob es Heini gelingt, die Kartons zu stapeln, berechnen wir
		      die Seitenlänge des Kartons $n$. Beachte, dass die Seitenlänge eines Würfels
		      die dritte Wurzel seines Volumens ist, also

		      \[
			      \frac{1}{n+1}
		      \]

		      Summieren wir nun die Seitenlängen aller Würfel:

		      \[
			      \sum_{n=0}^{\infty} \frac{1}{n+1} = \sum_{n=1}^{\infty} \frac{1}{n}
		      \]

		      Wir bemerken, dass dies die harmonische Reihe ist, die jedoch bekannterweise
		      konvergiert. Daher gelingt es Heini leider nicht, alle Geschenke zu stapeln.

		      Um nun herauszufinden, ob er rechtzeitig mit dem Auspacken fertig wird,
		      betrachten wir die Oberfläche eines Würfels:

		      \[
			      6 \frac{1}{{(n+1)}^2}
		      \]

		      und summieren sie erneut:

		      \[
			      \sum_{n=0}^{\infty} 6 \frac{1}{{(n+1)}^2} = 6 \sum_{n=1}^{\infty} \frac{1}{{(n)}^2}
		      \]

		      Wie wir bereits aus der Vorlesung wissen, konvergiert $\sum_{n=1}^{\infty}
			      \frac{1}{{(n)}^p}$ für $p \ge 2$. Und mit ein wenig Online-Recherche erhalten
		      wir, dass

		      \[
			      \sum_{n=1}^{\infty} \frac{1}{{(n)}^2} = \frac{\pi^2}{6}
		      \]

		      Also hat Heini in $\pi^2$ Minuten alles ausgepackt, und ist hoffentlich vor dem
		      Abendspaziergang und auf jeden Fall vor Neujahr fertig.

		      Es ist zu bemerken, dass $a_n = \frac{1}{{(n+1)}^3}$ eine Nullfolge bildet, was
		      erklären kann, warum Heini am Ende so enttäuscht ist.

	\end{enumerate}

\end{proof}
\end{problem}

\end{document}
