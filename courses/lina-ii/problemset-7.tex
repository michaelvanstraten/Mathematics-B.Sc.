\documentclass{problemset}

\Lecture{Lineare Algebra und Analytische Geometrie II}
\Problemset{7}
\DoOn{2. Juni 2024}
\author{Michael van Straten}

\begin{document}
\maketitle

\begin{problem}{5 Punkte}
Bestimmen Sie die Jordan-Normalform und eine zugehörige Transformationsmatrix zu
\[
    A = \begin{pmatrix}
        4   & 1  & 0  & 0 & 0 \\
        -4  & 0  & 0  & 0 & 0 \\
        -16 & -8 & 4  & 0 & 1 \\
        8   & 4  & 0  & 2 & 0 \\
        32  & 16 & -4 & 0 & 0
    \end{pmatrix} \in \mathbb{R}^{5 \times 5}.
\]
\end{problem}

\begin{problem}[\textit{[FS20, Seite 280, Aufgabe 6, 7]}]{5 Punkte}
Sei $n \in \mathbb{N}$, $V$ ein $\mathbb{C}$-Vektorraum mit $\dim(V) = n$ und $f \in L(V, V)$.
\begin{enumerate}
    \item Berechnen Sie $J_d(\lambda)^k$ für ein beliebiges $k \in \mathbb{N}$,
          wobei $\lambda \in \mathbb{C}$ einen Eigenwert von $f$ und
          $J_d(\lambda) \in \mathbb{C}^{d \times d}$ ein Jordan-Block der Größe
          $d \in \mathbb{N}$ ist.
    \item Sei $k \in \mathbb{N}$ mit $k \geq 2$ und $f^k = f$. Zeigen Sie, dass
          $f$ diagonalisierbar ist.
\end{enumerate}
\begin{hint}
    Sie können verwenden, dass das charakteristische Polynom eines Endomorphismus
    über $\mathbb{C}$ in Linearfaktoren zerfällt.
\end{hint}

\begin{proof}
    \leavevmode
    \begin{enumerate}
        \item Teilen wir zunächst \(J_d(\lambda)\) in zwei Matrizen \(D, N \in
              C^{d \times d}\), sodass \(D = \lambda I_d\) und \( N =
              J_d(\lambda) - D\) ist. Die Matrixmultiplikation \(A N^n\) für
              \(A \in C^{d \times d}\) und \(n \in \nats\) hat den Effekt die
              Spalten von \(A\) \(n\)-mal nach rechts zu verschieben.
              Definieren wir nun
              \begin{equation*}
                  l : \nats \times \nats \rightarrow \set{0,1} \text{\ gegeben durch }
                  (i,j) \mapsto \left\{
                  \begin{array}{ll}
                      1 & \text{ wenn, } i \le j \\
                      0 & \text{ sonst}
                  \end{array}
                  \right.
              \end{equation*}
              so folgt das
              \begin{align*}
                  J_d(\lambda)^k & = {(D + N)}^k                                                \\
                                 & = \sum_{i=0}^k \binom{k}{i} D^{k-i} N^i                      \\
                                 & = \left(l(i,j) \binom{k}{j - i - 2} \lambda^{k-j+i-2}\right)
              \end{align*}
              ist.

        \item Für jedes \(v \in \ker(f^k)\) gilt das \(f^k(v) = f(v) = 0 v =
              0\) somit ist jedes \(v \in \ker(f^k)\) ein Eigenvektor von \(f\)
              zum Eigenwert \(0\). Für jedes \(v \in \img(f^k)\) gilt, dass ein
              \(w \in V\) existiert mit,
              \begin{equation*}
                  v = f^{k-1}(w) \Rightarrow f(v) = f^k(w) = f(w) = v
              \end{equation*}
              was impliziert das \(v\) ein Eigenvektor von \(f\) zum Eigenwert \(0\) ist.
              Wählen wir nun eine Basis des Kerns sowie des Bildes von \(f^k\) so folgt aus
              deren Direktheit sowie der Dimensionsformel, das deren Vereinigung eine Basis
              des Vektorraum \(V\) bildet. Somit existiert eine Basis aus Eigenvektor von
              \(f\) was Equivalent zu der Aussage ist das \(f\) diagonalisierbar ist.
    \end{enumerate}
\end{proof}

\end{problem}

\begin{problem}[duale Raumpaare und Skalarprodukte \textit{[LM21, Aufgabe 12.3]}]{5 Punkte}
Sei $n \in \mathbb{N}$, $K \in \{\mathbb{R}, \mathbb{C}\}$.
\begin{enumerate}
    \item Sei $\langle \cdot , \cdot \rangle$ ein Skalarprodukt auf $K^n$.
          Zeigen Sie, dass eine hermitesche Matrix $A \in GL_n(K)$ mit reellen
          positiven Eigenwerten existiert, sodass für alle $v, w \in K^n$
          \[
              \langle v, w \rangle = w^H A v.
          \]
    \item Zu $A \in K^{n \times n}$ sei $\beta : K^n \times (K^n)^* \to K$
          definiert durch
          \[
              \beta(v, f) := f(Av).
          \]
          Zeigen Sie, dass $\beta$ genau dann nicht ausgeartet ist, wenn $A \in
          GL_n(K)$.
\end{enumerate}
\end{problem}

\end{document}
