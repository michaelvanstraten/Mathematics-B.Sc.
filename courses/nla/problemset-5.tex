
\documentclass{problemset}

\author{Michael van Straten}
\Lecture{Numerische Lineare Algebra}
\Problemset{5}
\DoOn{25. November 2025}

\setlist[enumerate, 1]{label=(\alph*)}

\begin{document}

\maketitle

\setcounter{problem}{2}

\begin{problem}{5 Punkte}
Sei \( A \in \mathbb{R}^{n \times n} \) eine invertierbare Matrix. Zeigen Sie, dass
\[
    \text{cond}_2(A) \geq \frac{\|A\|_2^n}{|\det(A)|}.
\]
\textit{Hinweis:} Nutzen Sie die Identität aus Aufgabe 1.3.
\end{problem}

\begin{problem}{8 Punkte}
In dieser Aufgabe wollen wir die Rückwärtsanalyse für die Bestimmung einer der
Nullstellen eines quadratischen Polynoms der Form \( x^2 - 2px + q \)
durchführen, wobei \( p, q \in \mathbb{R} \) mit \( p^2 > q \neq 0 \). Dafür
kann die pq-Formel verwendet werden:
\[
    f(p, q) = p + \sqrt{p^2 - q}.
\]

Für die Berechnung mit einem Computerprogramm \( \tilde{f} \) werden
fehlerbehaftete Elementaroperationen \( \tilde{+}, \tilde{-}, \tilde{\cdot},
\tilde{\sqrt{\phantom{x}}} \) verwendet. Dabei gilt, dass für alle Operationen
\( \star \in \{+, -, \cdot, \sqrt{\phantom{x}}\} \) und alle \( x, y \in
\mathbb{R} \) ein \( |\omega_\star| \leq \text{eps} \ll 1 \) existiert mit
\[
    x \tilde{+} y = (x + y)(1 + \omega_+), \quad x \tilde{-} y = (x - y)(1 + \omega_-),
\]
% \[
%     x \tilde{\cdot} y = (x \cdot y)(1 + \omega_\cdot), \quad \tilde{\sqrt{x}} = \sqrt{x}(1 + \omega_\sqrt).
% \]

Hierbei nehmen wir zusätzlich an, dass die Wurzelfunktion im Programm nur für
positive Argumente ausgewertet werden muss, also
\[
    (p^2)(1 + \omega_\cdot) \geq q.
\]

Zeigen Sie, dass \( \Delta p \) und \( \Delta q \) existieren, sodass
\[
    f(p + \Delta p, q + \Delta q) = \tilde{f}(p, q),
\]
mit
\[
    \left|\frac{\Delta p}{p}\right| \leq \text{eps},
\]
\[
    \left|\frac{\Delta q}{q}\right| \leq \left(5 + 4 \left|\frac{p^2}{q}\right|\right) \text{eps} + O(\text{eps}^2).
\]
\end{problem}

\begin{problem}{7 Punkte}
Sei \( C \in \reals^{n \times n} \) mit \( \norm{C} < 1 \) in einer
submultiplikativen Matrixnorm. Zeigen Sie:
\begin{enumerate}
    \item Die Reihe
          \[
              \sum_{k=0}^\infty C^k
          \]
          konvergiert, wobei \( C^0 = I \) die Einheitsmatrix ist.
    \item Die gestörte Einheitsmatrix \( I - C \) ist invertierbar und es gilt
          \[
              \sum_{k=0}^\infty C^k = (I - C)^{-1}.
          \]
    \item Für die Matrixnorm von \( (I - C)^{-1} \) gilt
          \[
              \norm{(I - C)^{-1}} \leq \frac{1}{1 - \norm{C}}.
          \]
\end{enumerate}
\begin{proof} \leavevmode
    \begin{enumerate}
        \item[(a) und (b)] Definieren wir
              \[
                  S_n \coloneq \sum_{k=0}^n C^k,
              \]
              sodass
              \[
                  \lim_{n \to \infty} S_n = \sum_{k=0}^\infty C^k.
              \]
              Es gilt:
              \begin{align*}
                  (I - C) S_n & = \sum_{k=0}^n C^k - \sum_{k=1}^{n + 1} C^k \\
                              & = C^0 - C^{n+1}.
              \end{align*}
              Daraus folgt:
              \[
                  S_n = (I - C)^{-1} (C^0 - C^{n+1}).
              \]

              Für den Grenzwert von \( S_n \) gilt dann:
              \[
                  \lim_{n \to \infty} S_n = (I - C)^{-1} \lim_{n \to \infty} (C^0 - C^{n+1}).
              \]
              Da \( \norm{C} < 1 \), existiert für jedes \( \epsilon > 0 \) ein
              \( N \in \nats \), sodass für alle \( n \geq N \) gilt:
              \begin{align*}
                  \norm{C^0 - C^{n+1} - C^0} & = \norm{C^{n+1}}            \\
                                             & \leq \norm{C}^n < \epsilon.
              \end{align*}

              Somit ergibt sich:
              \[
                  \sum_{k=0}^\infty C^k = (I - C)^{-1}.
              \]

        \item[(c)]
              Da Normen stetig sind und \( S_n \) gegen den oben gezeigten
              Grenzwert konvergiert, folgt:
              \begin{align*}
                  \norm{(I - C)^{-1}} & = \norm*{\sum_{k=0}^\infty C^k}    \\
                                      & \leq \sum_{k=0}^\infty \norm{C^k}  \\
                                      & \leq \sum_{k=0}^\infty \norm{C}^k.
              \end{align*}
              Da \( \norm{C} < 1 \), gilt:
              \[
                  \sum_{k=0}^\infty \norm{C}^k = \frac{1}{1 - \norm{C}}.
              \]
              Somit ist
              \[
                  \norm{(I - C)^{-1}} \leq \frac{1}{1 - \norm{C}}.
              \]
    \end{enumerate}
\end{proof}
\end{problem}

\end{document}

