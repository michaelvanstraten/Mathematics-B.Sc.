\documentclass{problemset}

\author{Michael van Straten}
\Lecture{Numerische Lineare Algebra}
\Problemset{3}
\DoOn{11. November 2024}

\setlist[enumerate, 1]{label=(\alph*)}

\begin{document}

\maketitle

\setcounter{problem}{3}

\begin{problem}{7 Punkte}
Lässt sich die Matrix
\[
    A =
    \begin{bmatrix}
        1 & 2 & 2 \\
        2 & 4 & 7 \\
        3 & 5 & 3 \\
    \end{bmatrix}
\]
mit Hilfe der LR-Zerlegung ohne Pivotisierung zerlegen? Begründen Sie Ihre
  Antwort. Bestimmen Sie anschließend die LR-Zerlegung von \(A\) mit
  Pivotisierung. Geben Sie auch die Permutationsmatrix \(P\) von \(PA = LR\)
  an.
\end{problem}

\begin{problem}{6 Punkte}
Gegeben sei die Matrix
\[
    A =
    \begin{bmatrix}
        4 & 2 & 4 \\
        2 & 2 & 3 \\
        4 & 3 & 9 \\
    \end{bmatrix}.
\]
Berechnen Sie die Cholesky-Zerlegung von \(A\).
\end{problem}

\begin{problem}{7 Punkte}
Sei \( A \in \reals^{n \times n} \) eine symmetrisch positiv definite Matrix.
\begin{enumerate}
    \item Sei \( A \) aufgeteilt wie in (1) mit beliebigem \( p < n \) und sei
          \( S \) das Schur-Komplement von \( A \) bezüglich \( A_{1,1} \).
          Zeigen Sie: \( S \) ist wohldefiniert, und sowohl \( A_{1,1} \) als
          auch \( S \) sind symmetrisch positiv definit.

    \item Zeigen Sie, dass eine Cholesky-Zerlegung \( A = LL^T \) existiert.
\end{enumerate}
\textit{Hinweis:} Gehen Sie ähnlich vor wie in der Hausaufgabe 2.5 und beweisen Sie die Aussage induktiv.

\begin{proof}\leavevmode
    \begin{enumerate}
        \item \label{problem:6:a} Zunächst zerlegen wir \( A \) wie folgt:
              \begin{align*}
                  A & =
                  \begin{bmatrix}
                      A_{1,1} & A_{1,2} \\ A_{2,1} & A_{2,2}
                  \end{bmatrix} \\
                    & =
                  \underbrace{
                      \begin{bmatrix}
                          1                      & 0 \\
                          A_{1,2}^T A_{1,1}^{-1} & 1
                      \end{bmatrix}
                  }_{\eqcolon L}
                  \underbrace{
                      \begin{bmatrix}
                          A_{1,1} & 0                                        \\
                          0       & A_{2,2} - A_{1,2}^T A_{1,1}^{-1} A_{1,2}
                      \end{bmatrix}
                  }_{\eqcolon D}
                  \underbrace{
                      \begin{bmatrix}
                          1 & A_{1,1}^{-1} A_{1,2} \\
                          0 & 1
                      \end{bmatrix}}_{= L^T}.
              \end{align*}

              Da \( A \) sowie \( L \) invertierbare Matrizen sind, ist das
              Schur-Komplement \( S \coloneq A_{2,2} - A_{1,2}^T A_{1,1}^{-1}
              A_{1,2} \) wohldefiniert.

              Da \(\det(L) = 1\), folgt:
              \[
                  D = L^{-1} A {\left(L^{-1}\right)}^T.
              \]
              Für \(v \in \reals^n \setminus \{0\}\) gilt daher:
              \[
                  v^T D v = v^T L^{-1} A {\left(L^{-1}\right)}^T v > 0,
              \]
              da \( A \) positiv definit ist. Daher sind sowohl \( A_{1,1} \)
              als auch das Schur-Komplement \( S \) positiv definit.

        \item Aus Aufgabe 2 wissen wir, dass für den \(2 \times 2\)-Fall eine
              Cholesky-Zerlegung existiert.

              Wir nehmen nun an, dass für ein \( n \in \nats \) und eine Matrix
              \( \tilde{A} \in \reals^{n \times n} \) die Cholesky-Zerlegung \(
              \tilde{A} = \tilde{L} \tilde{L}^T \) existiert.

              Sei
              \begin{equation*}
                  A = \begin{bmatrix}
                      \tilde{A} & v \\
                      v^T       & a
                  \end{bmatrix} \in \mathbb{R}^{(n+1) \times (n+1)}
              \end{equation*}
              mit \( v \in \reals^n \) und \( a \in \reals \), wobei \(
              A \) symmetrisch positiv definit ist.

              Zerlegen wir \( A \) wie folgt:
              \begin{align*}
                  A & =
                  \begin{bmatrix}
                      \tilde{L}                    & 0 \\
                      v^T \tilde{A}^{-1} \tilde{L} & 1
                  \end{bmatrix}
                  \begin{bmatrix}
                      \tilde{L}^T & \tilde{L}^{-1} v \\ 0 & a - v^T \tilde{A}^{-1} v
                  \end{bmatrix} \\
                    & =
                  \begin{bmatrix}
                      \tilde{L}                & 0 \\
                      v^T {(\tilde{L}^{-1})}^T & 1
                  \end{bmatrix}
                  \begin{bmatrix}
                      \tilde{L}^T & \tilde{L}^{-1} v \\ 0 & a - v^T \tilde{A}^{-1} v
                  \end{bmatrix} \\
                    & =
                  \underbrace{
                      \begin{bmatrix}
                          \tilde{L}                & 0                               \\
                          v^T {(\tilde{L}^{-1})}^T & \sqrt{a - v^T \tilde{A}^{-1} v}
                      \end{bmatrix}
                  }_{\eqcolon L}
                  \begin{bmatrix}
                      \tilde{L}^T & \tilde{L}^{-1} v \\ 0 & \sqrt{a - v^T \tilde{A}^{-1} v}
                  \end{bmatrix}.
              \end{align*}
              Aus \ref{problem:6:a} folgt, dass \( a - v^T \tilde{A}^{-1} v > 0
              \), sodass \( A \) durch \( LL^T \) zerlegt werden kann.
    \end{enumerate}
\end{proof}

\end{problem}

\end{document}
