\documentclass{problemset}
\usepackage{hyperref}
\usepackage{multirow}

\author{Michael van Straten}
\Lecture{Numerische Lineare Algebra}
\Problemset{8}
\DoOn{16. Dezember 2024}

\setlist[enumerate, 1]{label=(\alph*)}

\begin{document}

\maketitle

\setcounter{problem}{2}

\begin{problem}{7 Punkte}
Gegeben seien
\[
    A =
    \begin{bmatrix}
        1 & 1  & 2  \\
        2 & -3 & 0  \\
        2 & 4  & -4
    \end{bmatrix}, \quad
    b =
    \begin{bmatrix}
        9  \\
        -4 \\
        -2
    \end{bmatrix}.
\]
Bestimmen Sie eine QR-Zerlegung von \( A \), indem Sie zwei
  Householder-Transformationen \( Q_1 \) und \( Q_2 \) finden, sodass \( Q_2
  Q_1 A \) eine rechte obere Dreiecksmatrix ist. Nutzen Sie anschließend die
  QR-Zerlegung, um das lineare Gleichungssystem \( Ax = b \) zu lösen.

\textit{Hinweis:} Nutzen Sie zuerst Lemma 5.3 für die Elimination in der ersten Spalte. In der zweiten Transformation soll die erste Spalte unverändert bleiben.
\end{problem}

\begin{problem}{7 Punkte}
Seien \( A = Q_1 R_1 = Q_2 R_2 \) zwei QR-Zerlegungen einer regulären Matrix \(
A \in \mathbb{R}^{n \times n} \), d.h. \( Q_1, Q_2 \in \mathbb{R}^{n \times n}
\) sind orthogonale Matrizen und \( R_1, R_2 \in \mathbb{R}^{n \times n} \)
obere Dreiecksmatrizen. Zeigen Sie, dass eine orthogonale Diagonalmatrix \( S
\in \mathbb{R}^{n \times n} \) existiert, sodass
\[
    Q_1 = Q_2 S^T \quad \text{und} \quad R_1 = S R_2.
\]

\textit{Hinweis:} Sie können ohne Beweis annehmen, dass die Inverse einer oberen Dreiecksmatrix wieder eine obere Dreiecksmatrix ist.
\end{problem}

\begin{problem}{6 Punkte}
Sei \( A \in \mathbb{R}^{n \times n} \) eine invertierbare Matrix mit Spalten
\( a_i \in \mathbb{R}^n, \, i = 1, \ldots, n \). Seien \( q_1, \ldots, q_n \)
die normierten Vektoren, die beim Gram-Schmidt-Orthonormalisierungsverfahren
aus der Basis \( a_1, \ldots, a_n \) konstruiert werden. Sei \( Q \) die Matrix
mit Spalten \( q_1, \ldots, q_n \). Zeigen Sie, dass
\[
    R := Q^T A
\]
eine rechte obere Dreiecksmatrix ist und \( A = QR \) somit eine QR-Zerlegung
  ist.

\begin{proof}
    Aus \textbf{Algorithmus 3.11} der Vorlesung \emph{Lineare Algebra und
        Analytische Geometrie 2} von Prof. Dr. Walther \cite{walther2024} ist
    bekannt, dass für eine durch das Gram-Schmidt-Orthonormalisierungsverfahren
    orthogonalisierte Basis, in diesem Fall die Menge \( q_1, \ldots, q_n \),
    die ursprüngliche Basis \( a_1, \ldots, a_n \) die folgende Darstellung
    hat:
    \begin{equation*}
        \begin{pmatrix} a_1, \ldots, a_n \end{pmatrix}
        = \begin{pmatrix} q_1, \ldots, q_n \end{pmatrix}
        \underbrace{%
            \begin{pmatrix}
                \norm{a_1}              & \innerp{a_1}{q_1}      & \cdots & \cdots & \innerp{a_{n-1}}{q_1}     \\
                0                       & \norm{\widetilde{q}_2} & \ddots &        & \multirow{2}{*}{\vdots}   \\
                \multirow{2}{*}{\vdots} & 0                      & \ddots & \ddots &                           \\
                                        & \vdots                 & \ddots & \ddots & \innerp{a_{n-1}}{q_{n-1}} \\
                0                       & 0                      & \cdots & 0      & \norm{\widetilde{q}_n}
            \end{pmatrix}
        }_{\eqcolon R} \in K^{n,n}.
    \end{equation*}
    Dies lässt sich bestätigen, indem wir uns die $i$-te Spalte der rechten
    Matrix ansehen:
    \begin{align*}
        \sum_{j=1}^i (R)_{j,i} \cdot q_j
         & = \norm{\widetilde{q}_i} q_i + \sum_{j=1}^{i-1} \innerp{a_i}{q_j} q_j                  \\
         & = \widetilde{q}_i + \sum_{j=1}^{i-1} \innerp{a_i}{q_j} q_j                             \\
         & = a_i - \sum_{j=1}^{i-1}\innerp{a_i}{q_j} q_j + \sum_{j=1}^{i-1} \innerp{a_i}{q_j} q_j \\
         & = a_i.
    \end{align*}
    Damit folgt die Aussage, dass \(R\) eine obere Dreiecksmatrix ist und \(A =
    QR\) eine QR-Zerlegung darstellt.
\end{proof}

\begin{thebibliography}{1}
    \bibitem{walther2024}
    Prof. Dr. Walther, \emph{Lineare Algebra und Analytische Geometrie 2}.
    Zugriff: \url{https://pages.cms.hu-berlin.de/script-students/lina2/main.pdf#page=58}.
\end{thebibliography}
\end{problem}

\end{document}
