\documentclass{problemset}

\author{Michael van Straten}
\Lecture{Numerische Lineare Algebra}
\Problemset{10}
\DoOn{13. Januar 2025}

\setlist[enumerate, 1]{label=(\alph*)}

\begin{document}

\maketitle

\setcounter{problem}{1}

\begin{problem}[Moore-Penrose-Inverse]{6 Punkte}
Seien \( m, n \in \mathbb{N} \) und \( A \in \mathbb{R}^{m \times n} \). Eine
Matrix \( A^\dagger \in \mathbb{R}^{n \times m} \) heißt Moore-Penrose-Inverse
von \( A \), falls die folgenden Bedingungen erfüllt sind:
\begin{enumerate}
    \item[(a)] \( AA^\dagger A = A \),
    \item[(b)] \( A^\dagger A A^\dagger = A^\dagger \),
    \item[(c)] \( AA^\dagger \) ist symmetrisch,
    \item[(d)] \( A^\dagger A \) ist symmetrisch.
\end{enumerate}
Zeigen Sie:
\begin{enumerate}
    \item Die Moore-Penrose-Inverse \( A^\dagger \) einer Matrix \( A \) ist
          eindeutig durch (a)–(d) bestimmt.

          \textit{Hinweis:}
          Betrachten Sie für eine weitere Matrix \( \tilde{A}^\dagger \),
          welche die Eigenschaften (a)–(d) erfüllt, die Matrix \( D^2 \) mit \(
          D := A \tilde{A}^\dagger - A A^\dagger \).
    \item Es gilt \( A^\dagger = (A^\top A)^{-1} A^\top \), falls \(
          \operatorname{rg}(A) = n \), und \( A^\dagger = A^\top (A
          A^\top)^{-1} \), falls \( \operatorname{rg}(A) = m \).
    \item \( A^\dagger = A^{-1} \), falls \( A \) regulär ist.
\end{enumerate}
\begin{proof}\leavevmode
    \begin{enumerate}
        \item \label{problem:1:i} Zunächst betrachten wir wie im Hinweis vorgeschlagen \( D^2 \):
              \begin{align*}
                  {(A \tilde{A}^\dagger - A A^\dagger)}^2
                   & = A \tilde{A}^\dagger A \tilde{A}^\dagger
                  - A A^\dagger A \tilde{A}^\dagger
                  - A \tilde{A}^\dagger A A^\dagger
                  + A A^\dagger A A^\dagger                      \\
                   & = A \tilde{A}^\dagger - A \tilde{A}^\dagger
                  - A A^\dagger + A A^\dagger                    \\
                   & = 0.
              \end{align*}

              Da \( A \tilde{A}^\dagger \) und \( A A^\dagger \) reelle,
              symmetrische Matrizen sind, ist auch \( D \) symmetrisch und
              somit diagonalisierbar mit Diagonaleinträgen, die den Eigenwerten
              von \( D \) entsprechen.

              Aus \((A \tilde{A}^\dagger - A A^\dagger)^2 = 0\) folgt, dass \(
              D \) nilpotent ist und daher nur Eigenwerte gleich \( 0 \)
              besitzt. Daraus ergibt sich:
              \begin{align*}
                       & A \tilde{A}^\dagger - A A^\dagger = 0                                                   \\
                  \iff & A \tilde{A}^\dagger = A A^\dagger)                                                      \\
                  \iff & \tilde{A}^\dagger = \tilde{A}^\dagger A A^\dagger                                       \\
                  \iff & \tilde{A}^\dagger = \tilde{A}^\dagger A A^\dagger A A^\dagger                           \\
                  \iff & \tilde{A}^\dagger = {(\tilde{A}^\dagger A)}^{}\top A^\dagger A A^\dagger                \\
                  \iff & \tilde{A}^\dagger = A^\top {\tilde{A}}^{\dagger^\top} A^\top A^{\dagger^\top} A^\dagger \\
                  \iff & \tilde{A}^\dagger = A^\top A^{\dagger^\top} A^\dagger                                   \\
                  \iff & \tilde{A}^\dagger = {(A^\dagger A)}^\top A^\dagger                                      \\
                  \iff & \tilde{A}^\dagger = A^\dagger.
              \end{align*}
              Somit ist die Moore-Penrose-Inverse eindeutig bestimmt.

        \item Falls \( \operatorname{rg}(A) = n \), hat \( A^\top A \) den Rang
              \( n \) und ist daher invertierbar. Außerdem erfüllt \( A^\dagger
              = (A^\top A)^{-1} A^\top \) die Bedingungen (a)–(d) (diese können
              durch Einsetzen überprüft werden). Mit der Eindeutigkeit aus
              \ref{problem:1:i} folgt:
              \[
                  A^\dagger = (A^\top A)^{-1} A^\top.
              \]

              Analog gilt für \( \operatorname{rg}(A) = m \), dass \( A A^\top
              \) invertierbar ist. Daraus folgt:
              \[
                  A^\dagger = A^\top (A A^\top)^{-1}.
              \]

        \item Ist \( A \) regulär, so gilt \( \operatorname{rg}(A) = n = m \),
              und \( A^\dagger \) erfüllt die Bedingungen (a)–(d). Da \( A^{-1}
              \) ebenfalls die Bedingungen (a)–(d) erfüllt, folgt mit der
              Eindeutigkeit aus \ref{problem:1:i}, dass:
              \[
                  A^\dagger = A^{-1}.
              \]
    \end{enumerate}
\end{proof}
\end{problem}

\setcounter{problem}{3}

\begin{problem}{7 Punkte}
Seien \( m, n \in \mathbb{N} \) und \( A \in \mathbb{R}^{m \times n} \). Zeigen Sie:
\begin{enumerate}
    \item[i)] \( \ker(A) \oplus \operatorname{img}(A^\dagger) = \mathbb{R}^n
          \),
    \item[ii)] \( \ker(A^\dagger) \oplus \operatorname{img}(A) = \mathbb{R}^m
          \).
\end{enumerate}
\textit{Hinweis:} Nutzen Sie die Projektionen aus Aufgabe 3.
\end{problem}

\begin{problem}{7 Punkte}
Seien \( m, n \in \mathbb{N} \), \( A \in \mathbb{R}^{m \times n} \), und \( b
\in \mathbb{R}^m \). Zeigen Sie, dass für \( \hat{x} := A^\dagger b \) und alle
\( x \in \mathbb{R}^n \) gilt:
\[
    \|A \hat{x} - b\|_2^2 = \|A x - b\|_2^2 - \|A (x - \hat{x})\|_2^2
\]
und
\[
    \|A \hat{x} - b\| = \min_{x \in \mathbb{R}^n} \|A x - b\| = \|P_1 b\|,
\]
wobei \( P_1 \) die Projektion aus Aufgabe 3 ist. \textit{Hinweis:} Zeigen
  Sie gegebenenfalls zuerst, dass \( (A^\top)^\dagger = (A^\dagger)^\top \).
\end{problem}

\end{document}
