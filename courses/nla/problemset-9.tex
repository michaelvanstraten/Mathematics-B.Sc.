\documentclass{problemset}

\author{Michael van Straten}
\Lecture{Numerische Lineare Algebra}
\Problemset{9}
\DoOn{6. Januar 2025}

\setlist[enumerate, 1]{label=(\alph*)}

\begin{document}

\maketitle

\setcounter{problem}{2}

\begin{problem}{7 Punkte}
Gegeben seien
\begin{equation}
    A =
    \begin{bmatrix}
        1        & 1         \\
        1        & 1         \\
        \sqrt{2} & -\sqrt{2}
    \end{bmatrix}, \quad
    b =
    \begin{bmatrix}
        1 \\
        1 \\
        2\sqrt{2}
    \end{bmatrix}.
\end{equation}
Berechnen Sie eine QR-Zerlegung von \( A \). Zeigen Sie, dass das überbestimmte
Gleichungssystem \( Ax = b \) eine Lösung besitzt, und berechnen Sie diese.

\begin{proof}
    Mittels standardmäßiger Householder-Transformationen lässt sich eine
    QR-Zerlegung von \( A \) wie folgt berechnen:
    \begin{align*}
        x & = \begin{pmatrix}
                  1 \\ 1 \\ \sqrt{2}
              \end{pmatrix},                \\
        u & = x - \frac{x_1}{|x_1|} \norm{x} e_1     \\
          & = \begin{pmatrix}
                  1 \\ 1 \\ \sqrt{2}
              \end{pmatrix} - 2 \begin{pmatrix}
                                    1 \\ 0 \\ 0
                                \end{pmatrix}, \\
        v & = \frac{u}{\norm{u}}               \\
          & = \frac{1}{2} u.
    \end{align*}

    Für die erste Householder-Transformation ergibt sich:
    \begin{align*}
        Q_1 & = I_3 - 2 v v^T                                                                              \\
            & = I_3 - \frac{1}{2} u u^T                                                                    \\
            & = I_3 - \frac{1}{2} \begin{bmatrix}
                                      1         & -1       & -\sqrt{2} \\
                                      -1        & 1        & \sqrt{2}  \\
                                      -\sqrt{2} & \sqrt{2} & 2
                                  \end{bmatrix}                                         \\
            & = \begin{bmatrix}
                    \frac{1}{2}         & \frac{1}{2}        & \frac{\sqrt{2}}{2}  \\
                    \frac{1}{2}         & \frac{1}{2}        & -\frac{\sqrt{2}}{2} \\
                    -\frac{\sqrt{2}}{2} & \frac{\sqrt{2}}{2} & 0
                \end{bmatrix}.
    \end{align*}

    Multiplizieren wir nun \( Q_1 \) linksseitig mit \( A \), so erhalten wir:
    \begin{equation*}
        Q_1 A = \begin{bmatrix}
            2 & 0 \\
            0 & 2 \\
            0 & 0
        \end{bmatrix} \eqcolon R.
    \end{equation*}

    Da \( R \) eine obere Dreiecksmatrix ist, folgt, dass \( Q_1^T \) in
    Kombination mit \( R \) die gesuchten Eigenschaften der QR-Zerlegung
    erfüllt.

    Um das Gleichungssystem
    \begin{equation*}
        QRx = b
    \end{equation*}
    zu lösen, berechnen wir zunächst:
    \begin{equation*}
        Q^T b = Q_1 b = \begin{pmatrix}
            3 \\ -1 \\ 0
        \end{pmatrix}.
    \end{equation*}

    Es bleibt die Gleichung
    \begin{equation*}
        R x = \begin{bmatrix}
            2 & 0 \\
            0 & 2 \\
            0 & 0
        \end{bmatrix} x = Q^T b
    \end{equation*},
    welche uns das gesuchte \( x \) liefert:
    \begin{equation*}
        x = \begin{pmatrix}
            \frac{3}{2} \\ -\frac{1}{2}
        \end{pmatrix}.
    \end{equation*}
\end{proof}
\end{problem}

\begin{problem}{6 Punkte}
Sei \( A \in \mathbb{R}^{n \times n} \) eine invertierbare Matrix mit Spalten
\( a_i \in \mathbb{R}^n \). In Aufgabe 8.5 wurde gezeigt, dass eine
QR-Zerlegung mit Hilfe des Gram-Schmidt-Orthonormalisierungsverfahrens
konstruiert werden kann. Berechnen Sie die (gemeinsame) Anzahl der
Multiplikationen und Divisionen bei der Berechnung einer orthonormalen Basis \(
q_1, \ldots, q_n \) mit:
\[
    w_i = a_i - \sum_{j=1}^{i-1} \langle a_i, q_j \rangle q_j, \quad
    q_i = \frac{w_i}{\|w_i\|}.
\]
\textit{Kommentar:} Beachten Sie, dass \( R \) nicht mehr durch eine
Matrix-Matrix-Multiplikation berechnet werden muss, wie in Aufgabe 8.5
beschrieben. Es gilt nämlich \( R_{ji} = \langle q_j, a_i \rangle \), welche
bereits im Gram-Schmidt-Verfahren berechnet wurden.
\end{problem}

\begin{problem}{7 Punkte}
Ist für ein Gleichungssystem \( Ax = b \) mit \( A \in \mathbb{R}^{n \times n}
\) und \( b \in \mathbb{R}^n \) die Matrix schlecht konditioniert oder
singulär, so kann ersatzweise das regularisierte Problem
\[
    \min_{x \in \mathbb{R}^n} \|Ax - b\|_2^2 + \epsilon \|x\|_2^2
\]
(Tikhonov-Regularisierung) mit einem \( \epsilon > 0 \) gelöst werden. Zeigen
Sie, dass die Lösung \( x^* \) des regularisierten Problems gegeben ist durch
\[
    A^\top A x^* + \epsilon x^* = A^\top b.
\]
\end{problem}

\end{document}
