\documentclass{beamer}

\mode<presentation> {
\usepackage{pgfpages}
\setbeameroption{show notes on second screen=right}
}

\usepackage{color}
\usepackage{listings}
\usepackage{tikz}
\usepackage{graphicx}
\usepackage{fontspec}
\setmonofont{JetBrainsMono Nerd Font}

% Lean syntax highlighting
\definecolor{keywordcolor}{rgb}{0.7, 0.1, 0.1}   % red
\definecolor{tacticcolor}{rgb}{0.0, 0.1, 0.6}    % blue
\definecolor{commentcolor}{rgb}{0.4, 0.4, 0.4}   % grey
\definecolor{symbolcolor}{rgb}{0.0, 0.1, 0.6}    % blue
\definecolor{sortcolor}{rgb}{0.1, 0.5, 0.1}      % green
\definecolor{attributecolor}{rgb}{0.7, 0.1, 0.1} % red
\def\lstlanguagefiles{lstlean.tex}

\lstset{
    language=lean,
    basicstyle=\ttfamily\tiny,
    inputpath={./code-examples/}
}

\setbeameroption{show notes on second screen=right}

\graphicspath{{./assets/images/}{./assets/memes/}}

\title{Metaprogramming in Lean 4}
\author{Michael van Straten}
\date{\today}
\institute{Humboldt-Universität zu Berlin}

\AtBeginSection[]{
    \tableofcontents[currentsection]
}

\begin{document}

\NoHyper

\begin{frame}{About Me}
    \begin{columns}
        \begin{column}{0.5\textwidth}
            \begin{itemize}[<+(0)->]
                \item Hi, my name is Michael
                \item I am currently in the 4th semester of my Bachelor's
                      degree in Mathematics
                \item I like to spend my time writing open source software in
                      Rust, but I mostly spend it writing legacy C++ code :-).
            \end{itemize}
        \end{column}
        \begin{column}{0.5\textwidth}
            \begin{center}
                \begin{tikzpicture}
                    \clip (0,0) circle (0.4\textwidth) node {
                            \includegraphics[width=0.8\textwidth]{profile-picture.jpg}
                        };
                \end{tikzpicture}
            \end{center}
        \end{column}
    \end{columns}

    \note<1>[item]{I think you can do this without notes}
\end{frame}

\begin{frame}
    \titlepage
    \note<1>[item]{
        I would like to talk to you about Metaprogramming in Lean4 today. But
        before I would like to thank the Lean Prover community since this talk is
        largely based on the Metaprogramming in Lean4 book which. The book was a
        great resource to get started on and I can now say that I a much better
        understanding of some of the internals of lean.
    }
\end{frame}

\section{Introduction and Overview}

\begin{frame}{What is the goal of this talk?}
    \begin{itemize}[<+->]
        \item Gain insight into how Lean works from an imperative programming
              perspective
        \item Introduce fundamentals of Lean metaprogramming
        \item Understand how Lean's syntax is parsed, elaborated, and
              type-checked
        \item Explore the type theoretical background implementation
        \item Learn how to write your own syntax, macros, and tactics
    \end{itemize}

    \note<1>[item]{I had some difficulties understanding the Lean syntax when
        initially looking at Lean code online. I wanted to understand what a
        concrete expression really meant, which is why I wanted to challenge myself
        to do this talk.}
    \note<1>[item]{Almost everything in Lean is accomplished through metaprogramming,
        which can be challenging to someone coming from a more imperative programming
        background.}
    \note<2,3>[item]{To address this, I will introduce you to the
        fundamentals of Lean metaprogramming. By the end of this talk, you should
        understand how Lean's expressive syntax is parsed, elaborated, and
        type-checked.}
    \note<4>[item]{You can view this as a practical application and extension of
        the previous talk, in which we will explore how the type theoretical
        background is implemented.}
    \note<5>[item]{By the end, you'll be able to extend Lean with your own domain-specific languages.}
\end{frame}

\begin{frame}{Disclaimer}
    \begin{itemize}[<+->]
        \item \textbf{Warning:} Metaprogramming in Lean is an \emph{immense} topic
        \item This 1-hour talk only scratches the surface
        \item We'll focus on core concepts: Syntax, Macros, Elaboration,
              Tactics
        \item \textbf{Read the book:} "Metaprogramming in Lean 4" for the full story
    \end{itemize}

    \note<1>[item]{I want to set realistic expectations - metaprogramming in Lean is incredibly deep.}
    \note<2>[item]{We're covering the essentials in 60 minutes, but there's so much more.}
    % Name them
    \note<3>[item]{These are the core concepts you need to understand how Lean works.}
    \note<4>[item]{Think of this as your entry point - enough to get started and know what to learn next.}
    \note<4>[item]{The Metaprogramming in Lean 4 book is freely available online and goes much deeper.}
\end{frame}

\begin{frame}{About Monads in This Talk}
    \includegraphics[width=0.8\textwidth]{about-monads-in-this-talk.png}

    \begin{itemize}[<+->]
        \item You'll see monads like \lstinline{MetaM}, \lstinline{TacticM},
              \lstinline{MacroM}
        \item \textbf{What we won't cover monads :(}
              \begin{itemize}
                  \item We need monads to manage state (environment,
                        metavariables, local context) between pure function
                        calls
                  \item But I won't define what a monad is — that's a whole
                        other talk probably!
              \end{itemize}
    \end{itemize}

    \note<1>[item]{We'll use monads throughout but won't explain the theory behind them.}
    \note<2>[item]{For practical purposes, think of them as execution contexts with specific capabilities.}
    \note<2>[item]{The theory of monads is beautiful but would take too long to explain properly!}
\end{frame}

\begin{frame}[fragile]{What does it mean to be in meta?}
    \begin{itemize}[<+->]
        \item Most programming languages have fixed syntax
        \item Lean is different - almost no predefined syntax
        \item Everything is defined like regular definitions
        \item Example: rewriting \lstinline{if then else} behavior
              \lstinputlisting{basic/rewrite-if-then-else.lean}
        \item We manipulate Lean's CST (Concrete Syntax Tree) and AST (Abstract
              Syntax Tree)
        \item This is essential for writing tactics
    \end{itemize}

    \note<1>[item]{In most common programming languages like Python, C++, or Java,
        you usually have to stick to a pre-defined syntax; otherwise, the compiler
        (the thing that turns your source files into something computable) won't
        be able to understand what you are saying.}
    \note<2>[item]{In Lean, this is very different. There is in fact almost no
        predefined syntax. Everything you commonly see is basically defined in the
        same way you write a definition in a file.}
    \note<3>[item]{For example, the following code rewrites the meaning of if-then-else
        to flip the two conditions.}
    \note<6>[item]{We will frequently want to directly manipulate Lean's Concrete
        Syntax Tree and Lean's Abstract Syntax Tree (Lean's Expr type) instead of
        using "normal Lean code", especially when we're writing tactics. A large
        chunk of this talk is devoted to studying these types and how we can
        manipulate them.}
\end{frame}

\section{The Lean Compilation Pipeline}

\begin{frame}[fragile]{Lean Compilation Process}
    \begin{center}
        \includegraphics[width=\textwidth]{lean-compilation-procees.png}
    \end{center}

    \vspace{0.5cm}

    \begin{itemize}[<+->]
        \item Lean operates on two main types: \lstinline{Syntax} and
              \lstinline{Expr}
        \item Four-step process: Parsing → Macro Expansion → Elaboration →
              Evaluation
        \item Process is repeated for each top level command in a file
    \end{itemize}

    % Eplain with parsing the header (the imports and then the top level command)
    \note<1>[item]{As a 10000-foot overview of how Lean works: Lean will internally
        act on two main types - the Syntax and Expr types.}
    \note<1>[item]{You have actually seen an external representation of the Expr
        type in the previous talk.}
    \note<2>[item]{Step 1: Parser reads source code and builds Syntax trees using parser tables
        that contain rules from syntax declarations.}
    \note<2>[item]{Step 2: Macro expansion repeatedly applies macro transformations
        (Syntax → Syntax functions) until no more macros can be applied.}
    \note<2>[item]{Step 3: Elaboration converts the final Syntax trees into Expr trees,
        performing type inference, implicit argument insertion, and overload resolution.}
    \note<2>[item]{Step 4: The kernel type-checks Expr trees and they can be evaluated
        or compiled to executable code.}
    \note<3>[item]{This pipeline is run multiple times for each top level
        command in a lean file.}
\end{frame}

\begin{frame}[fragile]{A Concrete Example}
    \lstinputlisting{basic/transitive-command-elab.lean}

    \note<1>[item]{Let me walk through this example step by step to show how syntax flows through the compilation pipeline.}
    \note<1>[item]{Lines 1-2: We import Lean and open the Command elaboration namespace.}
    \note<1>[item]{Lines 4-6: We declare three syntax rules - "red", "green", "blue" - as commands, each pointing to elaborator names xxx, yyy, zzz.}
    \note<1>[item]{Lines 8-14: We define macros that transform syntax. The redMacro transforms xxx into green syntax, and greenMacro transforms yyy into blue syntax.}
    \note<1>[item]{Lines 16-17: We define a command elaborator for zzz that prints "finally, blue!" when executed.}
    \note<1>[item]{Line 19: When Lean sees "red", it follows the chain: red → xxx → green → yyy → blue → zzz → print message.}
\end{frame}

\begin{frame}{From Text to Syntax Trees}
    \begin{itemize}[<+(1)->]
        \item Lean starts by parsing source code into \lstinline{Syntax}
              objects
        \item The parser follows rules defined by \lstinline{syntax}
              declarations
        \item Each \lstinline{syntax} commadn adds a rule
        \item The result is a concrete syntax tree (CST) representing your code
        \item This tree preserves all information: operators, precedence,
              structure
    \end{itemize}

    \note<1>[item]{Before we dive into the Syntax type, let's understand what parsing does.}
    \note<2>[item]{Lean starts by parsing source code into \lstinline{Syntax} objects}
    \note<3>[item]{Lean has a table-driven parser that looks up syntax rules as it encounters tokens.}
    \note<4>[item]{Every syntax declaration we write adds new rules to this parser table.}
    \note<5>[item]{The parser builds as a result a concrete syntax tree that maintains the exact structure of your source code.}
    \note<6>[item]{Unlike abstract syntax trees, CSTs preserve formatting, comments, and other source-level details.}
\end{frame}

\section{Syntax Parsing}

\begin{frame}[fragile]{The Syntax Type}
    \begin{itemize}[<+->]
        \item \lstinline{Lean.Syntax} is an inductive type representing parsed code
        \item Four main constructors represent different syntax elements:
        \item \lstinline{missing}: Used for syntax errors and recovery
        \item \lstinline{node}: Tree nodes with kind and children
        \item \lstinline{atom}: String literals (operators, keywords)
        \item \lstinline{ident}: Identifiers with Lean.Name
    \end{itemize}

    \lstinputlisting{basic/syntax-type.lean}

    \note<1>[item]{Syntax is the fundamental type for representing parsed code in Lean.}
    \note<2>[item]{These four constructors can represent any piece of Lean
        syntax, namely \lstinline{missing}, \lstinline{node}, \lstinline{atom} and
    \lstinline{ident}.}
    \note<3>[item]{missing is used when the parser encounters an error but recovers to continue parsing.}
    \note<4>[item]{node represents syntax tree nodes with a kind (like \lstinline{term_+_}) and child syntax elements.}
    \note<5>[item]{atom represents terminal symbols like operators, keywords, and literals.}
    \note<6>[item]{ident is special - it uses Lean.Name instead of String for hygiene reasons.}
\end{frame}

\begin{frame}[fragile]{Inspecting Syntax}
    \lstinputlisting{basic/inspecting-syntax.lean}

    \note<1>[item]{We can introspect the parsed Syntax for any given term in lean using this
        code snippet. The output shows the complete tree structure, including source information
        and the hierarchical relationship between different syntax elements.}

    \note<1>[item]{We can introspect the parsed Syntax for any given term in lean using this
        code snippet. The output shows the complete tree structure, including source information
        and the hierarchical relationship between different syntax elements.}

    \note<1>[item]{You migth wander what the \lstinline{term} after the colon in
        the syntax command means. It is something called a syntax category.}
\end{frame}

\begin{frame}[fragile]{Syntax Categories}
    \begin{itemize}[<+->]
        \item Syntax categories group related parser rules together
        \item Each \lstinline{syntax} command adds a rule to a category
        \item Built-in categories: \lstinline{term}, \lstinline{command},
              \lstinline{tactic}
        \item Create custom categories with \lstinline{declare_syntax_cat}
        \item Categories enable domain-specific languages within Lean
    \end{itemize}

    \lstinputlisting{basic/syntax-cat.lean}

    \note<1>[item]{Think of syntax categories as grammar non-terminals that group related parsing rules.}
    \note<2>[item]{When you write a syntax command, you're adding a new production rule to that category's grammar.}
    \note<3>[item]{Lean comes with essential categories like term (expressions with values) and command (top-level declarations).}
    \note<4>[item]{Custom categories let you create mini-languages. Here we create an arithmetic expression language.}
    \note<5>[item]{The arith category only accepts numbers, +, -, and parentheses - nothing else. This gives us precise control over what syntax is allowed.}
\end{frame}

\begin{frame}[fragile]{Predefined Syntax: Infix and Notation}
    \begin{itemize}[<+(1)->]
        \item \lstinline{infix}, \lstinline{infixl}, \lstinline{infixr} for binary operators
        \item \lstinline{notation} for flexible mixfix syntax
        \item Precedence controls operator binding strength
        \item Associativity determines parsing of chains
        \item These are syntax sugar for \lstinline{syntax} + \lstinline{macro}
    \end{itemize}

    \lstinputlisting{basic/predefined-notation.lean}

    \note<1>[item]{Before into more custom syntax, let's see Lean's convenient shortcuts for common patterns.}
    \note<2>[item]{infixl creates left-associative operators, infixr creates right-associative ones.}
    \note<3>[item]{notation is more flexible - you can put parameters anywhere in the syntax.}
    \note<4>[item]{These examples come directly from Lean's initialization code in Init/Notation.lean.}
    \note<5>[item]{Notice how precedence numbers determine parsing: \lstinline{&&} (35) binds tighter than \lstinline{||} (30).}
    \note<6>[item]{The \lstinline{::} operator is right-associative
        (\lstinline{infixr}), perfect for building lists from left to right.}
\end{frame}

\section{Macros}

\begin{frame}[fragile]{What are Macros?}
    \begin{itemize}[<+->]
        \item Macros are \lstinline{Syntax → MacroM Syntax} functions
        \item Registered with \lstinline{@[macro name]}
        \item Compiler will apply the macro to the syntax for us
    \end{itemize}

    \lstinputlisting[lastline=14]{macros/basic-macro.lean}

    \note<1>[item]{Macros in Lean are \lstinline{Syntax → MacroM Syntax} functions. \lstinline{MacroM} is the macro
    monad which allows macros to have some static guarenties about your code, you can mostly ignore it for now.}
    \note<2>[item]{Macros are registered as handlers for a specific syntax declaration using the
        \lstinline{macro} attribute.}
    \note<3>[item]{The compiler will take care of applying these function
        to the syntax for us before performing actual analysis of the input.}
    \note<3>[item]{This
        means that the only thing we have to do is declare our syntax with a specific
        name and bind a function of type \lstinline{Lean.Macro} to it. Let's try to reproduce
        the \lstinline{LXOR} notation from the \lstinline{Syntax} chapter:}
\end{frame}

\begin{frame}[fragile]{Multiple Macros for Same Syntax}
    \begin{itemize}[<+->]
        \item Can register multiple macros for the same syntax
        \item Tried in reverse registration order (later = higher priority)
        \item Continue until success or real error
        \item Use \lstinline{Macro.throwUnsupported} to pass to next macro
    \end{itemize}

    \lstinputlisting[firstline=16,lastline=23]{macros/basic-macro.lean}

    \note<1>[item]{Lean allows multiple macros to handle the same syntax pattern.
        This enables sophisticated pattern matching and fallback behavior.}
    \note<2>[item]{The newer macro takes precedence, creating a specialized handler
        for the specific case of \lstinline{true LXOR true}.}
    \note<3>[item]{This demonstrates the power of Lean's macro system but should
        be used carefully to avoid confusing behavior.}
\end{frame}

\begin{frame}[fragile]{Macro Limitations and Best Practices}
    \begin{itemize}[<+->]
        \item Macros work on syntax structure, not semantic meaning
        \item Can lead to surprising behavior with different identifiers
        \item Use macros for simple syntax transformations
        \item Complex logic should use elaborators instead
        \item Keep transformations predictable and intuitive
    \end{itemize}

    \lstinputlisting[firstline=25,lastline=28]{macros/basic-macro.lean}

    \note<1>[item]{This example shows a potential pitfall - the macro only recognizes
        the literal identifiers \lstinline{true}, not variables containing true.}
    \note<2>[item]{The rule of thumb: if you're building real logic rather than
        simple syntax translation, use elaborators instead of macros.}
    \note<3>[item]{Macros should be transparent transformations that a human
        could easily perform manually.}
\end{frame}

\begin{frame}[fragile]{Simplifying Macro Declaration}
    \begin{itemize}[<+->]
        \item \lstinline{macro_rules} provides syntactic sugar for common patterns
        \item \lstinline{macro} offers even more concise syntax for simple cases
        \item These conveniences automatically handle registration and error
              handling
    \end{itemize}

    \lstinputlisting{macros/macro-sugar.lean}

    \note<1>[item]{Now that we know the basics of what a macro is and how to register it,
        we can explore more automated approaches. In fact, all the techniques we're about
        to see are implemented as macros themselves.}
    \note<2>[item]{\lstinline{macro_rules} desugars to functions like the ones we wrote manually.
        It automatically figures out the syntax name, handles the \lstinline{macro} attribute
        registration, and provides the \lstinline{throwUnsupported} wildcard behavior.}
    \note<3>[item]{The \lstinline{macro} syntax is even more concise and perfect for simple
        transformations that don't require complex logic. It combines syntax declaration
        and macro definition in one step.}
    \note<4>[item]{These pattern-matching approaches work like functions with guards,
        allowing arbitrarily complex macro logic on the right-hand side.}
\end{frame}

\subsection{Building a Set Languages with Macros}

\begin{frame}[fragile]{Defining basic set logic}
    \lstinputlisting[lastline=11]{macros/set-comprehensions.lean}

    \note<1>[item]{As a final mini project for this section we will build syntax
        for working with sets, using a macro this time so we can actually fully integrate it into the Lean syntax.}
    \note<2>[item]{Lets define the basics of sets}
\end{frame}

\begin{frame}[fragile]{Design Approach for Set Comprehensions}
    \begin{itemize}[<+->]
        \item \textbf{Goal:} Support both \lstinline{x : X | p x} and
              \lstinline{x ∈ X, p x} notations
        \item Two possible approaches:
              \begin{enumerate}
                  \item Define separate syntax and macro for each binding style
                  \item Define reusable syntax category for binders
              \end{enumerate}
        \item \textbf{We choose approach 2} for reusability across other constructs
    \end{itemize}

    \note<1>[item]{The goal is to support both traditional set-builder notation with type annotations and the more concise membership-based notation.}
    \note<2>[item]{We could create separate syntax for each, but that would lead to duplication when we want similar patterns elsewhere.}
    \note<3>[item]{By creating a reusable binder construct category, we can extend this pattern to other mathematical constructs that bind variables.}
\end{frame}

\begin{frame}[fragile]{Declaring Syntax Categories and Macro}
    \lstinputlisting[firstline=12,lastline=22]{macros/set-comprehensions.lean}

    \note<1>[item]{Syntax categories allow us to define a class of syntax rules that can be used interchangeably in specific contexts.}
    \note<2>[item]{The \lstinline{binder_construct} category will contain different ways to bind variables in our set comprehensions.}
    \note<3>[item]{The main syntax rule creates the familiar set comprehension structure with the vertical bar separator.}
\end{frame}

\begin{frame}[fragile]{Set Comprehensions in Action}
    \lstinputlisting[firstline=23,lastline=33]{macros/set-comprehensions.lean}

    \begin{itemize}[<+->]
        \item Define a simple set: \lstinline{oneSet} contains only the number
              1
        \item Create a set comprehension: \lstinline{x ∈ oneSet | 10 ≤ x}
        \item This expands to a logical contradiction (empty set)
        \item The proof demonstrates the macro's semantic correctness
    \end{itemize}

    \note<1>[item]{This example shows the macro system working end-to-end with actual mathematical reasoning.}
    \note<2>[item]{The set comprehension creates an impossible condition - no element can be both equal to 1 and greater than or equal to 10.}
    \note<3>[item]{The proof by contradiction shows that our macro expansion preserves the intended logical meaning.}
\end{frame}

\section{Elaborationd}

\begin{frame}[fragile]{The Bridge from Syntax to Lambda Calculus}
    \begin{itemize}[<+->]
        \item After macro expansion, we have \lstinline{Syntax} trees
        \item But syntax trees don't have meaning in the typed lambda calculus
              yet
        \item \textbf{Elaboration} converts syntax to meaningful expressions
        \item This is where type inference, implicit arguments, and overload
              resolution happen
        \item The result: \lstinline{Expr} trees ready for the kernel
    \end{itemize}

    \note<1>[item]{Macros give us powerful syntax transformations, but we're still just manipulating trees of tokens.}
    \note<2>[item]{Syntax doesn't know about types, definitions, or mathematical meaning in the typed lambda calculus.}
    \note<3>[item]{Elaboration is the bridge that gives semantic meaning to syntax.}
    \note<4>[item]{This is where the "magic" happens - types are inferred, holes are filled, ambiguities resolved.}
    \note<5>[item]{The output is expressions that the kernel can verify and evaluate.}
\end{frame}

\begin{frame}[fragile]{What is Elaboration?}
    \begin{itemize}[<+->]
        \item Two main types: Command and Term elaboration
        \item \textbf{Command elaboration}: Processes top-level declarations, modifies environment
        \item \textbf{Term elaboration}: Converts \lstinline{Syntax} to \lstinline{Expr}, handles type inference
        \item Command elaborators invoke term elaborators internally
        \item Examples: \lstinline{def} uses \lstinline{declarationElab}
              \leftarrow calls term elaboration
    \end{itemize}

    \lstinputlisting{elaboration/basic-elaboration.lean}

    \note<1>[item]{Elaboration has two distinct phases that work together but serve different purposes.}
    \note<2>[item]{Command elaborators run on every top-level command in a Lean file. They can modify the global environment by adding definitions, axioms, or other declarations.}
    \note<3>[item]{Term elaborators are responsible for converting syntax to expressions and performing type inference and checking.}
    \note<4>[item]{The key insight is that command elaborators often call term
        elaborators internally. For example, when you write \lstinline{def foo := bar}, the declaration elaborator processes the command structure but
        calls term elaboration to handle the expression \lstinline{bar}.}
    \note<5>[item]{This separation allows Lean to handle both the structural aspects of declarations and the semantic aspects of expressions in a modular way.}
\end{frame}

\begin{frame}[fragile]{The Expr Type: Lean's Abstract Syntax Tree}
    \begin{itemize}[<+->]
        \item \lstinline{Expr} represents the abstract syntax tree of Lean programs
        \item Every term in Lean has a corresponding \lstinline{Expr}
        \item Nine main constructors representing different expression types
        \item Foundation for type checking, elaboration, and evaluation
        \item Much simpler than source-level Lean syntax
    \end{itemize}

    \lstinputlisting{elaboration/expr-type.lean}

    \note<1>[item]{After macro expansion, Lean works with expressions - the abstract syntax tree representation.}
    \note<2>[item]{Every piece of Lean code you write gets converted to an Expr during elaboration.}
    \note<3>[item]{The Expr type has 9 constructors - each representing a fundamental building block.}
    \note<4>[item]{This is much simpler than the rich syntax we can write - that's the power of elaboration.}
    \note<5>[item]{Understanding Expr is crucial for metaprogramming because tactics manipulate these directly.}
\end{frame}

\begin{frame}[fragile]{Expression Constructors Explained}
    \begin{itemize}[<+(1)->]
        \item \lstinline{bvar n}: Bound variables using de Bruijn indices
        \item \lstinline{fvar id}: Free variables from local context
        \item \lstinline{mvar id}: Metavariables - holes to be filled
        \item \lstinline{sort u}: Universe sorts like \lstinline{Type}, \lstinline{Prop}
        \item \lstinline{const name levels}: Named constants with universe levels
        \item \lstinline{app f x}: Function application (curried)
        \item \lstinline{lam name type body info}: Lambda abstractions
        \item \lstinline{forallE name type body info}: Dependent function types
        \item \lstinline{letE name type value body nonDep}: Let expressions
    \end{itemize}

    \note<1>[item]{bvar uses de Bruijn indices - numbers that count binders upward from the variable.}
    \note<2>[item]{fvar represents local hypotheses and parameters - variables in the current context.}
    \note<3>[item]{mvar are placeholders - they represent goals in tactic proofs and unification variables.}
    \note<4>[item]{sort represents the type universe hierarchy - Type 0, Type 1, Prop, etc.}
    \note<5>[item]{const represents everything declared in the environment - functions, theorems, types.}
    \note<6>[item]{app is function application - multiple arguments use repeated application.}
    \note<7>[item]{lam creates lambda functions - the foundation of functional programming.}
    \note<8>[item]{forallE creates dependent function types - the foundation of type theory.}
    \note<9>[item]{letE allows local definitions within expressions.}
\end{frame}

\begin{frame}[fragile]{De Bruijn Indices: Bound Variables}
    \begin{itemize}[<+(1)->]
        \item Bound variables use numbers instead of names
        \item \lstinline{\#n} means "the binder n levels up"
        \item Avoids variable capture during substitution
        \item Essential for correct beta reduction
        \item Enables "locally nameless" representation
    \end{itemize}

    \lstinputlisting{elaboration/de-bruijn-indices.lean}

    \note<1>[item]{De Bruijn indices solve the variable capture problem in lambda calculus.}
    \note<2>[item]{Instead of names, we use numbers counting binders from the variable location.}
    \note<3>[item]{This ensures that substitution never accidentally captures variables.}
    \note<4>[item]{Beta reduction becomes simple substitution without name concerns.}
    \note<5>[item]{Names in binders are just suggestions for pretty-printing.}
\end{frame}

\begin{frame}[fragile]{Metavariables: Holes and Goals}
    \begin{itemize}[<+->]
        \item Metavariables represent unresolved parts of expressions
        \item Two perspectives: holes in expressions, goals in proofs
        \item Each has a unique \lstinline{MVarId}, target type, and local
              context
        \item Can be assigned values during elaboration/proof construction
        \item Essential for unification and type inference
    \end{itemize}

    \lstinputlisting{elaboration/metavariables-example.lean}

    \note<1>[item]{Metavariables are placeholders - they represent unknown parts of expressions.}
    \note<2>[item]{In proofs, metavariables are goals. In type inference, they're unknown types.}
    \note<3>[item]{Each metavariable has its own context of available hypotheses.}
    \note<4>[item]{Assignment propagates through expressions - solving one goal can affect others.}
    \note<5>[item]{This is how Lean does unification - by creating and solving metavariable constraints.}
\end{frame}

\begin{frame}[fragile]{The MetaM Monad: Managing Metavariables}
    \begin{itemize}[<+->]
        \item \lstinline{MetaM} provides access to metavariable context
        \item Create metavariables with \lstinline{mkFreshExprMVar}
        \item Assign metavariables with \lstinline{MVarId.assign}
        \item Update expressions with \lstinline{instantiateMVars}
        \item Essential for elaboration and tactic programming
    \end{itemize}

    \lstinputlisting{elaboration/metam-example.lean}

    \note<1>[item]{MetaM is the core monad for metaprogramming - it manages metavariable state.}
    \note<2>[item]{mkFreshExprMVar creates new goals/holes with specified types.}
    \note<3>[item]{Assignment updates the global metavariable context.}
    \note<4>[item]{instantiateMVars substitutes assigned metavariables throughout expressions.}
    \note<5>[item]{This is how tactics work - they manipulate the metavariable context to build proofs.}
\end{frame}

\begin{frame}[fragile]{Command Elaboration Example}
    \lstinputlisting{elaboration/command-elaboration.lean}

    \note<1>[item]{This example shows a useful diagnostic command that tells us
        what elaborators would handle a given piece of syntax.}
    \note<1>[item]{It demonstrates how to interact with Lean's environment and
        query the registered elaborators.}
\end{frame}

\begin{frame}[fragile]{Term Elaboration with Type Checking}
    \lstinputlisting{elaboration/term-elaboration.lean}

    \note<1>[item]{This example shows a more complex term elaborator that creates
        anonymous constructor syntax for single-constructor types.}
    \note<1>[item]{Notice how we use tryPostponeIfNoneOrMVar to handle cases
        where type information isn't available yet - this is crucial for
        Lean's elaboration algorithm.}
\end{frame}

\section{Introduction to Tactics}

\begin{frame}[fragile]{Metavariables as Tactic Goals}
    \begin{itemize}[<+->]
        \item Each tactic goal is internally a metavariable
        \item Metavariables store the goal's local context (available
              hypotheses)
        \item The metavariable's type is the goal we need to prove
        \item Solving a goal = assigning the metavariable
        \item Tactic communication happens through metavariable manipulation
    \end{itemize}

    \lstinputlisting{tactics/tactic-goals.lean}

    \note<1>[item]{This is the key insight - every goal you see is a metavariable underneath.}
    \note<2>[item]{The local context stores all available hypotheses and their types.}
    \note<3>[item]{The target type is what we need to construct a proof term for.}
    \note<4>[item]{When we "solve" a goal, we're actually assigning a proof term to the metavariable.}
    \note<5>[item]{Tactics coordinate by creating, assigning, and passing around metavariables.}
\end{frame}

\begin{frame}[fragile]{Tactic State and Metavariable Flow}
    \begin{itemize}[<+->]
        \item Tactic state = list of unassigned metavariables (open goals)
        \item Tactics receive current goal metavariable as input
        \item Can split goals (create new metavariables)
        \item Can solve goals (assign metavariables)
        \item Can transform goals (change the type/context)
    \end{itemize}

    \lstinputlisting{tactics/tactic-state-flow.lean}

    \note<1>[item]{The tactic state is just the list of metavariables that still need proof terms.}
    \note<2>[item]{Each tactic gets the current goal and must decide what to do with it.}
    \note<3>[item]{constructor creates new metavariables for the subgoals.}
    \note<4>[item]{exact assigns a concrete proof term to close the goal.}
    \note<5>[item]{At the end, all metavariables are assigned and we get the final proof term.}
\end{frame}

\begin{frame}[fragile]{What are Tactics?}
    \begin{itemize}[<+->]
        \item Programs that manipulate proof goals
        \item Type: \lstinline{TacticM Unit}
        \item Can be implemented via macros or elaboration
        \item Transform goal states step by step
        \item Essential for interactive theorem proving
    \end{itemize}

    \lstinputlisting{tactics/basic-tactics.lean}

    \note<1>[item]{Tactics are programs that help construct proofs by manipulating
        the goal state. They're essential for interactive theorem proving.}
    \note<2>[item]{Simple tactics can be implemented as macros that expand to
        existing tactics, while more complex ones require elaboration.}
\end{frame}

\begin{frame}[fragile]{Building an Assumption Tactic}
    \lstinputlisting{tactics/assumption-tactic.lean}

    \note<1>[item]{This tactic searches through the local context (hypotheses)
        to find one that exactly matches the goal type.}
    \note<1>[item]{It demonstrates key tactic programming concepts: accessing the goal,
        iterating through hypotheses, and closing goals.}
\end{frame}

\begin{frame}{Resources and Next Steps}
    \begin{itemize}[<+->]
        \item \textbf{Book:} "Metaprogramming in Lean 4" (free online)
        \item \textbf{Source Code:} Lean 4 repository on GitHub
        \item \textbf{The Lean Language Reference:} \url{https://lean-lang.org/doc/reference/}
    \end{itemize}

    \note<1>[item]{The Metaprogramming in Lean 4 book is the definitive resource
        and goes much deeper than this talk. It also has some exersizes.}
    \note<2>[item]{To view the actual implementation you can read through the
        lean prover source repository, but be warned, it is over 200k lines of
        lean code.}
    \note<3>[item]{I would not recommend the lean 4 language reference until you reviewed the basics from the Metaprogramming in
        Lean 4 book.}
\end{frame}

\begin{frame}{Summary}
    \begin{itemize}[<+->]
        \item Lean's metaprogramming enables extensible syntax
        \item \lstinline{Syntax} → Macros → Elaboration pipeline
        \item Macros for simple transformations, elaborators for complex logic
        \item Tactics manipulate proof goals programmatically
        \item Powerful foundation for domain-specific languages
        \item Essential skill for advanced Lean usage
    \end{itemize}

    \vspace{1cm}
    \begin{center}
        \Large Thank you for your attention!\\
        \vspace{0.5cm}
        Questions?

        \textbf{The source for the slides and examples can be found here:}
        \url{https://github.com/michaelvanstraten/Mathematics-B.Sc./tree/master/seminars/theorem-proving-in-lean}
    \end{center}

    \note<1>[item]{Lean's metaprogramming system is what makes it so extensible
        and powerful. Almost everything you use in Lean was built using these
        same techniques.}
    \note<2>[item]{The key insight is understanding the pipeline from syntax through
        macros to elaboration, and knowing which tool to use when.}
    \note<6>[item]{Mastering metaprogramming opens up advanced Lean techniques and
        makes you much more effective at both using and contributing to the ecosystem.}
\end{frame}
\endNoHyper
\end{document}
